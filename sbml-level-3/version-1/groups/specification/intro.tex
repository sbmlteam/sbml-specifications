% -*- TeX-master: "main" -*-

\section{Introduction}
\label{intro}
 





\subsection{Proposal corresponding to this package specification}

This specification for Groups in SBML Level~3 Version~1 is based on the proposal located at the following URL:

\begin{center}
  \vspace*{1ex}\small
  \url{https://sbml.svn.sf.net/svnroot/sbml/trunk/specifications/sbml-level-3/version-1/groups/proposal}
  \vspace*{1ex}
\end{center}

The tracking number in the SBML issue tracking system~\citep{tracker} for Groups package activities is 2847474.  The version of the proposal used as the starting point for this specification is the version of June, 2012.


\subsection{Package dependecies}

The Groups package has no dependencies on other SBML Level~3 packages.  It is also designed with the goal of being able to work seamlessly with other SBML Level~3 packages.  (If you find incompatibilities with other packages, please contact the author of this package.  Contact information is shown on the front page of this document.)


\subsection{Document conventions}
\label{conventions}

Following the precedent set by the SBML Level~3 Core specification document, the class diagrams in this specification follow UML~1.0 (Unified Modeling Language; \citealt{eriksson:1998,oestereich:1999}) and are used to define the constructs provided by this package.  Color in the diagrams covey additional information for the benefit of those viewing the document on media that can display color.  The following are the colors used and their meanings:

\begin{itemize}

\item[\raisebox{2.75pt}{\colorbox{black}{\rule{0.8pt}{0.8pt}}}]
  \emph{Black}: Items colored black in the UML diagrams are components
  taken unchanged from their definition in the SBML Level~3 Core
  specification document.

\item[\raisebox{2.75pt}{\colorbox{mediumgreen}{\rule{0.8pt}{0.8pt}}}]
  \emph{\textcolor{mediumgreen}{Green}}: Items colored green are
  components that exist in SBML Level~3 Core, but are extended by this
  package.  Class boxes are also drawn with dashed lines to further
  distinguish them.

\item[\raisebox{2.75pt}{\colorbox{darkblue}{\rule{0.8pt}{0.8pt}}}]
  \emph{\textcolor{darkblue}{Blue}}: Items colored blue are new
  components introduced in this package specification.  They have no
  equivalent in the SBML Level~3 Core specification.

\end{itemize}

In this document, the following typographical conventions distinguish the names of objects and data types from other entities; these are identical to the conventions used in the SBML Level~3 Core specification document:

\begin{description}
  
\item \abstractclass{AbstractClass}: Abstract classes are classes that
  are never instantiated directly, but rather serve as parents of other
  object classes.  Their names begin with a capital letter and they are
  printed in a slanted, bold, sans-serif typeface.  In electronic
  document formats, the class names defined within this document are
  also hyperlinked to their definitions; clicking
  on these items will, given appropriate software, switch the view to
  the section in this document containing the definition of that class.
  (However, for classes that are unchanged from their definitions in
  SBML Level~3 Core, the class names are not hyperlinked because they
  are not defined within this document.)
  
\item \class{Class}: Names of ordinary (concrete) classes begin with a
  capital letter and are printed in an upright, bold, sans-serif
  typeface.  In electronic document formats, the class names are also
  hyperlinked to their definitions in this specification document.
  (However, as in the previous case, class names are not hyperlinked if
  they are for classes that are unchanged from their definitions in the
  SBML Level~3 Core specification.)

\item \token{SomeThing}, \token{otherThing}: Attributes of classes, data
  type names, literal XML, and generally all tokens \emph{other} than
  SBML UML class names, are printed in an upright typewriter typeface.
  Primitive types defined by SBML begin with a capital letter; SBML also
  makes use of primitive types defined by XML
  Schema~1.0~\citep{biron:2000,fallside:2000,thompson:2000}, but
  unfortunately, XML~Schema does not follow any capitalization
  convention and primitive types drawn from the XML~Schema language may
  or may not start with a capital letter.

\end{description}

For other matters involving the use of UML and XML, the same conventions used in the SBML Level~3 Core specification document are used in this specification as well.




