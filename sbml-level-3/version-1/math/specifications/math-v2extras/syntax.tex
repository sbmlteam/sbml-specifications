% -*- TeX-master: "main" -*-

\section{Package syntax and semantics}
\label{sec:syntax}

This section contains a definition of the syntax and semantics of the \sbmlthreepkg.  The \ThisPackage involves the following \mathmltwo \cite{w3c:2000b} constructs.

%\TODO{list mathml elements used}

\begin{itemize}\setlength{\parskip}{-0.3ex}

\item \emph{arithmetic operators}: \token{quotient}, \token{max}, \token{min}, \token{rem}

\item \emph{logical operators}: \token{implies}

\item \emph{csymbol}: \token{\uri{http://www.sbml.org/sbml/symbols/rateOf}}

\end{itemize}



% --------------------------------------------------------------------------
\subsection{Namespace URI and other declarations necessary for using this package}
\label{xml-namespace}

Every SBML Level~3 package is identified uniquely by an XML namespace URI.  For an SBML document to be able to use a given Level~3 package, it must declare the use of that package by referencing its URI.  The following is the namespace URI for this version of the \ThisPackage for \sbmlthreecorenoversion:
\begin{center}
\PackageURL
\end{center}

In addition, SBML documents using a given package must indicate whether the package can be used to change the mathematical interpretation of a model.  This is done using the attribute \token{required} on the \token{<sbml>} element in the SBML document.  For the \ThisPackage, the value of this attribute must be \val{true}, because the use of the \ThisPackage will change the mathematical meaning of a model.

The following fragment illustrates the beginning of a typical SBML model using \sbmlthreecore and this version of the \ThisPackage:

%\TODO{adjust example}

\begin{example}
<?xml version="1.0" encoding="UTF-8"?>
<sbml xmlns="http://www.sbml.org/sbml/level3/version1/core" level="3" version="1"
      xmlns:math_xxx="http://www.sbml.org/sbml/level3/math/v2extras/version1"
      math_xxx:required="true">
\end{example}

The use of the package namespace differs from other Level~3 packages. In this package it is only be used to indicate the use of additional constructs. The MathML constructs themselves remain in the MathML namespace in accordance with the current use of MathML within SBML. 


\MYTODO{what about the number <-> boolean issue}
%------------------------------------------------------------------------------
\subsection{The quotient operator}
\label{sec_quotient}

The \token{quotient} element is a binary arithmetic operator used for division modulo a particular base. When the quotient operator is applied to integer arguments a and b, the result is the "quotient of a divided by b". That is, quotient returns the unique integer q such that a = q b + r. (In common usage, q is called the quotient and r is the remainder.)

\begin{example}
<math xmlns="http://www.w3.org/1998/Math/MathML">
    <apply>
        <quotient/>
        <ci> a </ci>
        <ci> b </ci>
    </apply>
</math>
\end{example}

\subsubsection{Properties of \token{quotient}}

\begin{itemize}\setlength{\parskip}{-0.3ex}

\item binary function

\item arguments must be type \primtype{int}

\item result is \primtype{int}

\end{itemize}


%------------------------------------------------------------------------------
\subsection{The rem operator}
\label{sec_rem}

The \token{rem} element is a binary arithmetic operator that returns the "remainder" of a division modulo a particular base. When the rem operator is applied to integer arguments a and b, the result is the "remainder of a divided by b". That is, rem returns the unique integer, r such that a = q b+ r, where r < q. (In common usage, q is called the quotient and r is the remainder.)

\begin{example}
<math xmlns="http://www.w3.org/1998/Math/MathML">
    <apply>
        <rem/>
        <ci> a </ci>
        <ci> b </ci>
    </apply>
</math>
\end{example}

\subsubsection{Properties of \token{rem}}

\begin{itemize}\setlength{\parskip}{-0.3ex}

\item binary function

\item arguments must be type \primtype{int}

\item result is \primtype{int}

\item rem(a, 0) is undefined

\end{itemize}


%------------------------------------------------------------------------------
\subsection{The max operator}
\label{sec_max}

The \token{max} element is an n-ary arithmentic function used to compare the values of the arguments. It returns the maximum of these values.

\begin{example}
<math xmlns="http://www.w3.org/1998/Math/MathML">
    <apply>
        <max/>
        <ci> a </ci>
        <ci> b </ci>
        <ci> c </ci>
    </apply>
</math>
\end{example}

\subsubsection{Properties of \token{max}}

\begin{itemize}\setlength{\parskip}{-0.3ex}

\item nary function

\item max() is undefined

\item arguments must all be comparable if the result is to be well defined

\end{itemize}

\subsubsection{Other package considerations}

\begin{itemize}

\item \textbf{arrays}: When the arguments to the \token{max} function represent an array of values, the result is the maximum value of the values in the array.

 
\end{itemize}


%------------------------------------------------------------------------------
\subsection{The min operator}
\label{sec_min}

The \token{min} element is an n-ary arithmentic function used to compare the values of the arguments. It returns the minimum of these values.

\begin{example}
<math xmlns="http://www.w3.org/1998/Math/MathML">
    <apply>
        <min/>
        <ci> a </ci>
        <ci> b </ci>
        <ci> c </ci>
    </apply>
</math>
\end{example}

\subsubsection{Properties of \token{min}}

\begin{itemize}\setlength{\parskip}{-0.3ex}

\item nary function

\item min() is undefined

\item arguments must all be comparable if the result is to be well defined

\end{itemize}

\subsubsection{Other package considerations}

\begin{itemize}

\item \textbf{arrays}: When the arguments to the \token{min} function represent an array of values, the result is the minimum value of the values in the array.

 
\end{itemize}
%------------------------------------------------------------------------------
\subsection{The implies operator}
\label{sec_implies}

The \token{implies} element is a binary logical function used to represent the boolean relational operator "implies".

\begin{example}
<math xmlns="http://www.w3.org/1998/Math/MathML">
    <apply>
        <implies/>
        <ci> a </ci>
        <ci> b </ci>
    </apply>
</math>
\end{example}

\subsubsection{Properties of \token{implies}}

\begin{itemize}\setlength{\parskip}{-0.3ex}

\item binary function

\item arguments must be type \primtype{boolean}

\item result is \primtype{boolean}


\end{itemize}



%------------------------------------------------------------------------------
\subsection{The csymbol rateOf}
\label{sec_rateOf}

SBML uses the MathML \token{csymbol} element to denote certain built-in mathematical entities without introducing reserved names into the component identifier namespace.
\ThisPackage introduces the \token{rateOf} \token{csymbol}.


The intent of this \token{csymbol} is to provide a means for models to refer to quantities that must naturally be computed as part of doing a dynamical analysis of a model.  The \emph{rateOf} \token{csymbol} is not intended to provide full numerical differentiation capabilities but to represent the instantaneous rate of change, with respect to time, of an entity in the model. 

\subsubsection{Attributes on csymbol rateOf}

\begin{itemize}
 
\item The \token{encoding} attribute of \token{csymbol} must be set to \val{text}.  

\item The \token{definitionURL} should be set to \uri{http://www.sbml.org/sbml/symbols/rateOf}.

\end{itemize}

\subsubsection{Properties of \token{rateOf}}

\begin{itemize}\setlength{\parskip}{-0.3ex}

\item unary function

\item argument must be type \primtype{SId}

\item result is \primtype{double}

\item rateOf(a) = 0;  if \token{a} is declared \val{constant}


\end{itemize}

\subsubsection{Determining the \token{rateOf} for a symbol }

\begin{enumerate}
\item The \emph{rateOf} for a symbol whose \primtype{SId} appears as the \token{variable} of a \RateRule is the numerical value of that \RateRule, using the current values of all symbols referenced in the rule's formula.


\item The \emph{rateOf} for the \emph{amount} of a \Species having attribute \token{boundaryCondition}=\val{true} and appearing in one or more reactions can be calculated from the stoichiometries and \KineticLaw \Math of every \Reaction in which the species appears, plus appropriate \token{conversionFactor} values.  If the \emph{species quantity} is in terms of its \emph{concentration}, the rate must be converted by the \token{size} of the \Compartment in which it appears, which may itself be changing in time.  This can be calculated as follows, where $[x]$ is the concentration of species X, $x$ the amount of species X, and $V$ the size of the compartment in which species X is located:

    \begin{larray*}
      \frac{d[x]}{dt} &=& \frac{d(x/V)}{dt} \\
      &=& \frac{1}{V} \cdot \frac{dx}{dt} + x \cdot \frac{d(1/V)}{dt} \\
      &=& \frac{1}{V} \cdot \frac{dx}{dt} - \frac{x}{V^2} \cdot \frac{dV}{dt}
    \end{larray*}
    \vspace*{0.5ex}
   When $dV/dt$ is equal to zero, the final term in the last line is zero.

\end{enumerate}

In simulations that progress in a stepwise fashion, such as stochastic simulations, the \emph{rateOf} \token{csymbol} is still calculated as above, from any appropriate \RateRule or \KineticLaw.  This effectively means that for stepwise simulations, the \emph{rateOf} indicates the expected average rate of change of the corresponding symbol over time, even when the actual rate of change may be zero or discontinuous.

\subsubsection{Special considerations of \token{rateOf} in \sbmlthreecorenoversion}

\begin{itemize}

\item The allowable identifiers for use with \emph{rateOf} in \sbmlthreecorenoversion are restricted to those of \Compartment, \LocalParameter, \Parameter, \Species, and \SpeciesReference objects in the enclosing model.  Note that \emph{rateOf} is not allowed for \Reaction objects, because their identifiers already represent the rate of change of the reaction, and calculating second derivatives is beyond the scope of this construct.  Likewise, there is no sensible meaning to be given to the \emph{rateOf} of a \FunctionDefinition, \Event, \Priority, \Delay, or other SBML entities.



\item An object whose \primtype{SId} identifier appears as the \token{variable} of an \AssignmentRule, or which is calculated from an \AlgebraicRule, may not be referenced by the \emph{rateOf} \token{csymbol}.  Similarly, it is also not valid to use the \emph{rateOf} \token{csymbol} to reference a \Species with a \token{hasOnlySubstanceUnits} attribute value of \val{false} and whose \token{compartment} appears as the \token{variable} of an \AssignmentRule or whose \token{size} is calculated from an \AlgebraicRule.  In other words, anything whose value is directly or indirectly determined by an algebraic rule or an assignment rule is excluded.

\item In the event of a discontinuity, such as might happen due to an \Event, a \token{piecewise} function, the beginning of a time course simulation (i.e., at $t=0$), or due to a new construct defined in a package, the rate of change is defined as the right-handed \emph{rateOf} for the symbol, that is, the derivative with respect to time of the symbol moving forward in time from the current time, and not the derivative with respect to time from the recent past up until the current time.  Thus, the \emph{rateOf} of a symbol will always be calculable from the set of current values of symbols in the model.  No \Event can affect the \emph{rateOf} for a symbol except indirectly.


\end{itemize}

\MYTODO{any packages that have things that might use rateOf}
