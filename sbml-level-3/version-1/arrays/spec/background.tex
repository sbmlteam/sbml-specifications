% -*- TeX-master: "main"; fill-column: 72 -*-

\section{ Background }
\label{background}

\subsection{ Problems with current SBML approaches }

All mathematical operations in SBML are currently restricted to operations on scalar values.  There is currently no way to express an array of values.  If one desires such a structure, they must first flatten it into individual scalar values.  This approach is not appealing for expressing things such as a probability mass function for a distribution.  A second problem in SBML is that regular structures cannot be represented efficiently.  While the hierarchical modeling package has made this somewhat easier, it is still necessary to instantiate each submodel individually and make connections between the submodels explicitly.  

\subsection{Package history}

The idea of adding arrays to SBML has been around for more than 10 years when it when it was originally proposed by Andrew Finney, Victoria Gor, Ben Bornstein, and Eric Mjolsness in September 2003.   Bruce Shapiro, Victoria Gor, and Eric Mjolsness revised this proposal in December 2004 to support ``dynamic'' arrays.  This specification adopts several of the ideas described in these proposals with some exceptions.  In particular, this specification restricts arrays to be statically sized, and it requires all arrays to be explicitly defined.  

This specification evolved through discussions at various COMBINE and HARMONY workshops beginning with a presentation by Chris Myers at COMBINE 2012 at the University of Toronto.  Over the past several years, Sarah Keating, Lucian Smith, Mike Hucka, Nicolas LeNovere, and Stuart Moodie, among others, made significant contributions to these discussions.  As part of the 2014 Google Summer of Code, Leandro Watanabe under the direction of Nicolas Rodriguez implemented a JSBML version of the arrays package, and during this time made significant revisions to this specification document.  Around the same time, Sarah Keating implemented this package within libSBML, as well.  Support for the arrays package has been incorporated into the iBioSim software tool with the help of Leandro Watanabe, Scott Glass, and Chris Myers.

TODO: IS THERE ANYTHING MORE THAT SHOULD BE SAID HERE?  ANYONE LEFT OUT?

\subsection{Array notation used in this document}

Elements of 1-dimensional arrays are referenced as $X[i]$ in this document to refer to the  $i^{th}$ element of the 1-dimensional array $X$ while $X[i][j]$ is used to refer to an individual element in a 2-dimensional array where $i$ selects the highest dimension and $j$ the lowest dimension.
Higher dimension arrays can be indexed in a similar fashion.
This notation is used to describe the meaning of certain features, but it is not intended to be used explicitly anywhere in SBML. 
In some cases, this specification may refer to the $i^{th}$ element of an array of objects (such as rules) that do not have object ids and cannot be referenced in SBML. 
%Furthermore, this specification sometimes uses the same notation $x(i,j)$ to refer to the array itself with indices i, j,... rather than a specific array element; the correct interpretation should be clear from the context in which it is used. 
%The notation A[i..j, k..m, ..., p..q] will be used to refer to the array 
%i.e, the array whose first dimension ranges from i to j, second dimension ranges from k to m, and whose last dimension ranges from p to q. For example A[0..5,0..7] refers to 6 by 8 array whose indices both start at 0. 


%\subsection{ Design goals of the current \FBC Package }
%\label{design-goals}

