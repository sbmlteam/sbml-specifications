% -*- TeX-master: "main"; fill-column: 72 -*-

\section{Validation of SBML documents}
\label{apdx-validation}

An important issue for software systems is being able to determine the
validity of a given SBML document that uses constructs from the
Hierarchical Model Composition package.  This section describes
operational rules for assessing validity.  


\subsection{Validation procedure}
\label{validation-procedure}


\subsection{Validation and consistency rules}
\label{validation-rules}

This section summarizes all the conditions that must (or in some cases,
at least \emph{should}) be true of an SBML Level~3 Version~1 model that
uses the Hierarchical Model Composition package.  We use the same
conventions as are used in the SBML Level~3 Version~1 Core specification
document.  In particular, there are different degrees of rule
strictness.  Formally, the differences are expressed in the statement of
a rule: either a rule states that a condition \emph{must} be true, or a
rule states that it \emph{should} be true.  Rules of the former kind are
strict SBML validation rules---a model encoded in SBML must conform to
all of them in order to be considered valid.  Rules of the latter kind
are consistency rules.  To help highlight these differences, we use the
following three symbols next to the rule numbers:

\begin{description}

\item[\hspace*{7.8pt}\vSymbol] A \vSymbolName indicates a
  \emph{requirement} for SBML conformance. If a model does not follow
  this rule, it does not conform to the Hierarchical Model Composition
  specification.  (Mnemonic intention behind the choice of symbol:
  ``This must be checked.'')

\item[\hspace*{8.7pt}\cSymbol] A \cSymbolName indicates a
  \emph{recommendation} for model consistency.  If a model does not
  follow this rule, it is not considered strictly invalid as far as
  the Hierarchical Model Composition specification is concerned;
  however, it indicates that the model contains a physical or
  conceptual inconsistency.  (Mnemonic intention behind the choice of
  symbol: ``This is a cause for warning.'')

\item[\hspace*{7.5pt}\mSymbol] A \mSymbolName indicates a strong
  recommendation for good modeling practice.  This rule is not
  strictly a matter of SBML encoding, but the recommendation comes
  from logical reasoning.  As in the previous case, if a model does
  not follow this rule, it is not strictly considered an invalid SBML
  encoding.  (Mnemonic intention behind the choice of symbol: ``You're
  a star if you heed this.'')

\end{description}

The validation rules listed in the following subsections are all stated
or implied in the reset of this specification document.  They are
enumerated here for convenience.  Unless explicitly stated, all
validation rules concern objects and attributes specifically defined in
the Hierarchical Model Composition package.


\subsubsection*{General rules about this package} \begin{sbmlenum}

\validRule{comp10101}{An SBML file with elements conforming to this
  specification must use the namespace \\
  \val{http://www.sbml.org/sbml/level3/version1/comp/version1}}

\validRule{comp10102}{All new classes and attributes described in this
  document must be defined using the \token{comp} namespace, either
  explicitly or via an XML Namespace prefix.}


\end{sbmlenum} \subsubsection*{General rules about identifiers} \begin{sbmlenum}

\validRule{comp10201}{Within an SBMLDocument, the value of the
  attribute \token{id} and \token{comp:id} on every instance of all
  \Model and \ExternalModelDefinition objects must
  be unique across the set of all \token{id} and \token{comp:id}
  attribute values of such identifiers in the SBML document to which
  they belong.}

\validRule{comp10202}{(Extending L3V1 Core validation rule 10301) Within
  a \Model or \ExternalModelDefinition object, the value of the
  attribute \token{id} and \token{comp:id} on every instance of the
  following classes of objects must be unique across the set of all
  \token{id} and \token{comp:id} attribute values of all such objects in
  a model: the \Model itself, plus all contained \FunctionDefinition,
  \Compartment, \Species, \Reaction, \SpeciesReference,
  \ModifierSpeciesReference, \Event, and \Parameter objects, plus the
  newly-defined objects \Submodel and \Deletion.}

\validRule{comp10203}{Within a \Model or \ExternalModelDefinition
  object, the value of the attribute \token{comp:id} on every instance
  of all \Port objects must be unique across the set of all
  \token{comp:id} attribute values of all such objects in the model.}

\validRule{comp10204}{The value of a \token{comp:id} attribute must
  always conform to the syntax of the SBML data type \primtype{SId}.}

\validRule{comp10205}{The value of model attributes on
  \ExternalModelDefinition objects, \token{submodelRef},
  \token{deletion}, and \token{conversionFactor} attributes on
  \ReplacedElement objects, \token{modelRef},
  \token{lengthConversionFactor}, \token{areaConversionFactor},
  \token{volumeConversionFactor}, \token{substanceConversionFactor},
  \token{timeConversionFactor}, and \token{extentConversionFactor}
  attributes on \Submodel objects, and \token{port} and \token{idRef}
  attributes on \SBaseRef objects must always conform to the syntax of
  the SBML data type \primtype{SId}.}

\validRule{comp10206}{The value of the \token{unitRef} attribute on
  \SBaseRef objects must always conform to the syntax of the SBML data
  type \primtype{UnitSId}.}

\validRule{comp10207}{The value of the \token{metaIdRef} attributes on
  \SBaseRef objects must always conform to the syntax of the XML data
  type \primtype{ID}.}

\validRule{comp10208}{The value of the \token{source} attribute on
  \ExternalModelDefinition objects must always conform to the syntax of
  the XML Schema 1.0 data type \primtype{anyURI}.}

\validRule{comp10209}{The value of the \token{md5} attribute on
  \ExternalModelDefinition objects must always conform to the syntax of
  type \primtype{string}.}


\end{sbmlenum} \subsubsection*{General rules about \class{SBaseRef} class objects and subclasses} \begin{sbmlenum}

\validRule{comp10301}{No \Port object may use the optional \token{port}
  attribute, as this would cause either a circular reference, or would
  cause two port objects in the same model to point to the same
  object.}

\validRule{comp10302}{No two \Port objects in the same \Model may
  reference the same XML element.  That is, the element pointed to
  through the use of the \token{idRef}, \token{unitRef}, or
  \token{metaIdRef} attributes, in conjunction with any child \SBaseRef
  element, may not be the same element pointed to by a \Port object with
  the same parent \ListOfPorts, whether it uses the same attribute to
  point to that object or not.}

\validRule{comp10303}{No two \ReplacedElement objects in the same \Model
  may reference the same XML element unless that element is a \Deletion.
  That is, the element pointed to through the use of the \token{port}, \token{idRef},
  \token{unitRef}, or \token{metaIdRef} attributes, in conjunction with any child
  \SBaseRef element, may not be the same element pointed to by any other
  \ReplacedElement in the same \Model, whether it uses the same attribute
  to point to that object or not.} 


\end{sbmlenum} \subsubsection*{General rules about circular references in models} \begin{sbmlenum}

\validRule{comp10401}{No \ExternalModelDefinition may reference an
  \ExternalModelDefinition in a different SBML document that in turn
  refers to the original \ExternalModelDefinition object, whether
  directly or indirectly through a chain of \ExternalModelDefinition objects.} 

\validRule{comp10402}{No \Model may contain a \Submodel which references
  itself.  That is, the \token{id} attribute of a \Model may not match the
  \token{modelRef} attribute on any of its \Submodel objects.} 

\validRule{comp10403}{No \Model may contain a \Submodel which references
  itself indirectly.  That is, the \token{modelRef} attribute of a \Submodel may
  not point to a \Model, any of whose \Submodel objects point to the original
  \Model, whether directly or indirectly through a chain of
  \Model/\Submodel pairs.} 


\end{sbmlenum} \subsubsection*{General rules about class inheritance} \begin{sbmlenum}

\validRule{comp10501}{The \Deletion, \ExternalModelDefinition,
  \ModelDefinition, \Port, \ReplacedElement, \SBaseRef, \Submodel,
  \ListOfDeletions, \ListOfExternalModelDefinitions,
  \ListOfModelDefinitions, \ListOfPorts, \ListOfReplacedElements, and
  \ListOfSubmodels classes are comp namespace elements that inherit from
  the SBML Level 3 Version 1 class \SBase.  As such, they must follow the
  validation rules for L3v1 core attributes and child elements from the
  \SBase class.} 

\validRule{comp10502}{The \ListOfDeletions,
  \ListOfExternalModelDefinitions, \ListOfModelDefinitions, \ListOfPorts,
  \ListOfReplacedElements, and \ListOfSubmodels classes are comp namespace
  elements that inherit from the SBML Level 3 Version 1 class ListOf.
  As such, they must follow the validation rules for L3v1 core
  attributes and child elements from the \ListOf class.} 

\validRule{comp10501}{The \ModelDefinition class is a comp namespace
  element that inherits from the SBML Level 3 Version 1 class \Model.  As
  such, it must follow the validation rules for L3v1 core attributes and
  child elements from the Model class.} 


\end{sbmlenum} \subsubsection*{Rules for the extended \class{SBML} container object} \begin{sbmlenum}

\validRule{comp20101}{There may be at most one instance of each of the
  following kind of object in an SBML document:  \ListOfModelDefinitions,
  and \ListOfExternalModelDefinitions. } 

\validRule{comp20102}{The \token{required} attribute must be set true if its
  \Model child contains any \Submodel objects with \Species, \Parameter,
  \Reaction, or \Event objects (directly or indirectly) that have not been
  replaced.  [Note:  This may be too hard to implement--maybe go for a
  warning instead?]} 

\validRule{comp20103}{Apart from the general notes and annotation
  subobjects permitted on all SBML components, a \ListOfModelDefinitions
  container object may only contain ModelDefinition objects.} 

\validRule{comp20104}{Apart from the general notes and annotation
  subobjects permitted on all SBML components, a
  ListOfExternalModelDefinitions container object may only contain
  ExternalModelDefinition objects.} 

\validRule{comp20105}{A ListOfModelDefinitions object may define no
  attribute from the comp namespace.} 

\validRule{comp20106}{A ListOfExternalModelDefinitions object may define
  no attribute from the comp namespace.} 


\end{sbmlenum} \subsubsection*{Rules for \class{SBaseRef}, \class{Deletion}, \class{Port} and \class{ReplacedElement} objects} \begin{sbmlenum}

\validRule{comp20201}{Every SBaseRef object must point to an object.
  That is, SBaseRef, Deletion, and Port objects must define one of the
  attributes port, idRef, unitRef, or metaIdRef, and ReplacedElement
  objects must define one of the attributes port, idRef, unitRef,
  metaIdRef, or deletion.} 

\validRule{comp20202}{No SBaseRef object may point to an object using
  more than one method.  That is, SBaseRef, Deletion, and Port objects
  must not define more than one of the attributes port, idRef, unitRef,
  or metaIdRef, and ReplacedElement objects must not define more than
  one of the attributes port, idRef, unitRef, metaIdRef, or deletion.} 

\validRule{comp20203}{The value of a port attribute on an SBaseRef
  object must be the identifier of a Port object from the referenced
  Model. } 

\validRule{comp20204}{The value of an idRef attribute on an SBaseRef
  object must be the identifier of an object from the referenced Model
  within the SId namespace for that model.  This includes elements with
  id attributes which are defined in packages other than Level 3 core or
  this comp package.} 

\validRule{comp20205}{The value of a unitRef attribute on an SBaseRef
  object must be the identifier of a UnitDefinition object from the
  referenced Model. } 

\validRule{comp20206}{The value of a metaIdRef attribute on an SBaseRef
  object must be the value of a metaId attribute on any element
  contained in the referenced Model.  This includes elements with metaId
  attributes which are defined in packages other than Level 3 core or
  this comp package.} 

\validRule{comp20207}{If an SBaseRef object contains an SBaseRef child,
  it must point to a Submodel element.} 

\validRule{comp20208}{The value of a submodelRef attribute on a
  ReplacedElement object must be the identifier of a Submodel object
  from the parent Model of the ReplacedElement. } 

\validRule{comp20209}{The value of a deletion attribute on a
  ReplacedElement object must be the identifier of a Deletion object
  from the parent Model of the ReplacedElement. } 

\validRule{comp20210}{The value of a submodelRef attribute on a
  ReplacedElement object which also defines a deletion attribute must be
  the identifier of the Submodel object to which the referenced Deletion
  belongs. } 

\validRule{comp20211}{The value of a conversionFactor attribute on a
  ReplacedElement object must be the identifier of a Parameter object
  from the parent Model of the ReplacedElement. } 

\validRule{comp20212}{The value of an identical attribute on a
  ReplacedElement object must, if present, have a value of type
  Boolean.} 

\validRule{comp20213}{If the value of the identical attribute on a
  ReplacedElement object is true, the parent element of the
  ListOfReplacedElements to which the ReplacedElement belongs must be
  the same class as the referenced element.} 

\validRule{comp20214}{If the value of the identical attribute on a
  ReplacedElement object is true, the parent element of the
  ListOfReplacedElements to which the ReplacedElement belongs must
  define all of the attributes present on the referenced element.  This
  includes attributes from other namespaces, such as from packages other
  than Level 3 core and this 'comp' package.} 

\validRule{comp20215}{If the value of the identical attribute on a
  ReplacedElement object is true, the parent element of the
  ListOfReplacedElements to which the ReplacedElement belongs must only
  define attributes present on the referenced element, with the
  exception of the id and metaId attributes, which may be added even if
  not present on the referenced element.} 

\validRule{comp20216}{If the value of the identical attribute on a
  ReplacedElement object is true, all attributes of the parent element
  of the ListOfReplacedElements to which the ReplacedElement belongs
  (including attributes from other namespaces) must be identical to the
  corresponding attributes of the referenced element, with the exception
  of the id and metaId attributes, which may be anything, and with the
  exception of attributes of type SIdRef, UnitSIdRef, PortSIdRef, and
  IDREF, which must now reference elements of the parent model which
  themselves are replacements for the original target of the reference
  attribute.  Those referenced replacements need not be flagged with
  'identical=true', and need not be identical to the elements they
  replace.  If any attributes define a numerical value in the submodel
  that would be converted to a new value in the parent model though the
  use of a conversionFactor, that attribute must be set to be equal to
  the new numerical value.} 

\validRule{comp20217}{If the value of the identical attribute on a
  ReplacedElement object is true, the children of the parent element of
  the ListOfReplacedElements to which the ReplacedElement belongs must
  be identical to the corresponding children of the referenced element,
  with the exception of any child ListOfReplacedElements objects (which
  have no restrictions).  'Identical' means these child objects
  themselves must follow validation rules comp20213, comp20214,
  comp20215, comp20216, and comp20217.} 

\validRule{comp20218}{(warning) If the identical attribute on a
  ReplacedElement object is not set, all attributes with defined values
  on the referenced element should be defined on the parent element of
  the ListOfReplacedElements to which the ReplacedElement belongs.} 

\validRule{comp20219}{(warning) If the identical attribute on a
  ReplacedElement object is not set, the parent element of the
  ListOfReplacedElements to which the ReplacedElement belongs should
  contain the same number and type of children as the referenced
  element, with the exception of ListOfReplacedElements children.} 

\validRule{comp20220}{SBaseRef objects may define port, idRef, unitRef,
  and metaIdRef attributes.  SBaseRef objects which are not Port,
  Deletion, or ReplacedElement objects may not define any other
  attributes from the comp namespace.} 

\validRule{comp20221}{Port objects may define an id attribute in
  addition to the port, idRef, unitRef, and metaIdRef attributes.  No
  other attributes from the comp namespace are permitted on a Port
  object.} 

\validRule{comp20222}{Deletion objects may define an id attribute in
  addition to the port, idRef, unitRef, and metaIdRef attributes.  No
  other attributes from the comp namespace are permitted on a Deletion
  object.} 

\validRule{comp20223}{ReplacedElement objects must define a submodelRef
  attribute, and may define deletion, identical, and conversionFactor
  attributes, in addition to the port, idRef, unitRef, and metaIdRef
  attributes.  No other attributes from the comp namespace are permitted
  on a ReplacedElement object.} 


\end{sbmlenum} \subsubsection*{Rules for \class{ModelDefinition} objects} \begin{sbmlenum}

\validRule{comp20301}{ModelDefinition objects inherit from the Model
  class, and must follow the same restrictions present on Model objects.
  This includes any validation rules from the SBML Level 3 Version 1
  core specification as well as this document.} 


\end{sbmlenum} \subsubsection*{Rules for \class{Model} objects} \begin{sbmlenum}

\validRule{comp20401}{There may be at most one instance of each of the
  following kind of object in a Model:  ListOfSubmodels, and
  ListOfPorts. } 

\validRule{comp20402}{Apart from the general notes and annotation
  subobjects permitted on all SBML components, a ListOfSubmodels
  container object may only contain Submodels objects.} 

\validRule{comp20403}{Apart from the general notes and annotation
  subobjects permitted on all SBML components, a ListOfPorts container
  object may only contain Port objects.} 

\validRule{comp20404}{A ListOfSubmodels object may define no attribute
  from the comp namespace.} 

\validRule{comp20405}{A ListOfPorts object may define no attribute from
  the comp namespace.} 


\end{sbmlenum} \subsubsection*{Rules for \class{ExternalModelDefinition} objects} \begin{sbmlenum}

\validRule{comp20501}{ExternalModelDefinition objects must define the id
  and source attributes, and may define the model and md5 attributes.
  No other attributes from the comp namespace are permitted on an
  ExternalModelDefinition object.} 

\validRule{comp20502}{The value of the source attribute on an
  ExternalModelDefinition object must point to a SBML Level 3 document.}

\validRule{comp20503}{The value of the model attribute on an
  ExternalModelDefinition object, if present, must refer to an id in the
  model namespace of the SBML document pointed to by the source
  attribute.} 

\consistencyRule{comp20504}{The value of the md5 attribute on an
  ExternalModelDefinition object, if present, should match the
  calculated md5 hash of the SBML document pointed to by the source
  attribute. [Note: This is almost certainly too vague and perhaps also
  incorrect, since I just made it up without knowing thing one about
  md5's, so consider this a placeholder.]}


\end{sbmlenum} \subsubsection*{Rules for \class{Submodel} objects} \begin{sbmlenum}

\validRule{comp20601}{Submodel objects must define the id and modelRef
  attributes, and may define the lengthConversionFactor,
  areaConversionFactor, volumeConversionFactor,
  substanceConversionFactor, timeConversionFactor, and
  extentConversionFactor attributes.  No other attributes from the comp
  namespace are permitted on a Submodel object.} 

\validRule{comp20602}{There may be at most one instance of the
  ListOfDeletions object in a Submodel.} 

\validRule{comp20603}{The lengthConversionFactor, areaConversionFactor,
  volumeConversionFactor, substanceConversionFactor,
  timeConversionFactor, and extentConversionFactor attributes on a
  Submodel object must, if defined, refer to Parameter objects in the
  same Model as the Submodel.} 

\validRule{comp20605}{The modelRef attribute on a Submodel must refer to
  the id of a Model, ModelDefinition, or ExternalModelDefinition object
  in the same SBMLDocument as the Submodel. } 

\validRule{comp20606}{Apart from the general notes and annotation
  subobjects permitted on all SBML components, a ListOfDeletions
  container object may only contain Deletion objects.} 

\validRule{comp20607}{A ListOfDeletions object may define no attribute
  from the comp namespace.} 


\end{sbmlenum} \subsubsection*{Rules for the extended \class{SBase} class} \begin{sbmlenum}

\validRule{comp20701}{SBase objects (that is, all elements inheriting
  from the SBase class, as defined in the SBML Level 3 Version 1 core
  specification, as defined in this package, and as defined in other
  packages) may contain at most one instance of the
  ListOfReplacedElements object.} 

\validRule{comp20702}{Apart from the general notes and annotation
  subobjects permitted on all SBML components, a ListOfReplacedElements
  container object may only contain ReplacedElement objects.} 

\validRule{comp20703}{A ListOfReplacedElements object may define no
  attribute from the comp namespace.} 


\end{sbmlenum}
