\documentclass{article}


\title{{\scshape XML Standard for Cellular \\ Models (MML)} \\ \  \\ First Release}
\author{ERATO \\}

\newcommand{\versionno}{1.0}
\newcommand{\docdate}{10~May~2000}
\date{Version \versionno\\Last revised \docdate}

\let\App=\appendix
\def\appendix{\App\small} % tighten up the ending pages of the standard.

\hbadness=9999 % Cut down the amount of annoying messages generated...
\parindent=0pt
\setlength{\parskip}{4pt}

\begin{document}
\maketitle

\begin{quotation}
\begin{center}
Intended audience
\end{center}

\medskip\noindent
This specification is intended for use by implementors of software to
interconvert specific storage schemes for metabolic simulation data
into a portable common format.

\medskip\noindent
The text of the specification assumes a basic background in XML and XML
Schemas.
\end{quotation}

\section*{Introduction}

There are a number of excellent cellular modelling packages now available over
the internet. One of the main problems with these packages is that it is not
possible to exchange models developed by one application for use in another,
for example Gepasi cannot read SCAMP files and visa versa. A number of
independent research groups have suggested developing a common specification
for the storage of metabolic models.  The BTK group set up a discussion
mailing list to discuss the possibility of a portable format early in 1999,
and more recently the ERATO project has made similar moves.

The advent of the Extensible Markup Language (XML) has made it possible to
develop a portable format based on an industry standard.

XML provides a number of distinct advantages:

\begin{itemize}
\item XML is a world-wide data exchange standard developed and standardised by
the World Wide Wed Consortium, W3C (http://www.w3.org).
\item Standard parsers, validators and rendering engines are now widely available on all
the major computer platforms.
\item XML files are human readable and in principal can be entered using any
basic ASCII editor.
\end{itemize}

It has been decided therefore to develop an XML standard for exchanging
metabolic models.

\section*{Information for the Format}

Here is a list of the sorts information that we should consider in the format:

\begin{verbatim}
  Version Number
  Model Label Name
  Header Information
     eg Title
        Author
        Date and Time
        Notes
  Species List
        Identification Label (eg Glucose), db reference?
        Fixed or Float Status
        Compartment (?)
        Initial Concentration
  Reaction List
        Identification Label (eg BranchFlux)
        Reference to data base entry if applicable
        List of Reactants
           Reactant Identification Label (eg Glucose)
           Stoichiometry, ratio of two integers
        List of Products
           Product Identification Label (eg Glucose_6_Phosphate)
           Stoichiometry, ratio of two integers
        Rate Law
           Rate Expression
           Parameter Values
  Forcing Function List (?)
\end{verbatim}

Note that there is no process specification, eg the types of algorithm to use
and specific things like time start, time stop, or the number of points and so
on.

\section*{Suggestion for a File Structure}

A metabolic XML file (hence abbreviated to MML) could have the following basic
structure:

\begin{verbatim}
  <?xml version="1.0"?>
  <MMLFILE xmlns="cell:mml.xml" version = "1.0">
    <NETWORK NAME="YeastModel">
     ....Header Information
     ....Species List
     ....Reaction List
     ....Forcing Function List (Optional)
    </NETWORK>
  </MMLFILE>
\end{verbatim}

{\tt MMLFILE} is the root element and the file {\tt mml.xml} is the MML schema
definition (or alternatively a DTD specification if this is more desirable).
Note the version number, this will be required if we intend doing a second
level specification of the format at a later date.


\section*{Header Information - Optional ?}

{\footnotesize
\begin{verbatim}
  <TITLE>Martian Life</TITLE>
  <AUTHOR>A Gladiator</AUTHOR>
  <DATE>12 June 2009</DATE>
  <NOTES>Model of Martian Organism v3.4 as published in Journal
       of Alien Life, vol 1, pp1-24, 2009</NOTES>
\end{verbatim} }

The sorts of information that should go in the header is debatable and the
example above simply gives an idea of the sort of things one might wish it to
contain. The entire header should probably be optional. Note that there is no
organism specific information, again this needs discussion but we should not
limit the format to only `real' models since a lot of research is also carried
out using hypothetical models. One reason why the header should be optional is
that not all simulators specifically store this type of information, if the
header is compulsory then applications must have the option to ignore it.

One other thing that probably needs looking at is whether we need to supply
information on the units for mass, length and time since the various kinetic
constants and concentrations will have units which should be made explicit.
However again this information should perhaps be optional or at least
ignorable by the application if the model is simply a hypothetical model where
the units in such cases will be of no interest.

The next two levels, the species and the reaction list must be present in
all MML files.

\section*{Species List}

The species list stores a list of species in the model, by species I include
entities such as simple ions (protons, calcium etc), simple molecules such as
glucose, ATP etc, large molecules such a RNA, polysaccharides and of course
proteins, be they single monomers or complexes.

The most basic type of information required to define a species include, its
name (possibly a reference to a data base entry might be useful here), it's
initial concentration and whether, within the model, it is a floating species
or a fixed boundary species. Distinguishing between fixed or floating species
is important and the mix of types in a model has a significant impact on the
structural properties of the model, which includes conservation and flux
constraints.

{\bf Compartmentation}

One aspect I've not mentioned is compartmentation. A living cell is not a
homogeneous bag of chemicals but is divided into a number of distinct
compartmental volumes. An obvious example being the cytoplasmic and
mitochondria spaces. A potential problem arises when mass (in the form of
molecular entities) moves from one compartment to another. If the volumes in
the two compartments are different then the concentration changes in the two
compartments will change by differing amounts when a unit of mass moves from
one compartment to another. Although this might seem a straight forward
problem to deal with, on closer inspection it turns out to be non-trivial. I
realize that a number of existing simulators `support' compartmentation but I
know for certain that at least one of them gets it wrong. As for the others I
don't know since I've not tested them. SCAMP takes a more cautious approach
and relies on the modeller to make some decisions on how compartmentation
should be dealt with.

For these reasons I have left out any mention of compartments even though they
are clearly an important aspect to modelling metabolic systems. I would instead
wait for further discussion on this problem before we decide on what should be
done. If we need to do something now then the simplest is to add a compartment
list before the species list where all the compartments in the model are
listed together with their volumes; then in the species list we indicate for
each species which compartment it is associated with and leave the rest to the
simulators to sort out.

For example:

{\footnotesize
\begin{verbatim}
   <COMPARTMENTLIST
     <COMPARTMENT NAME="BLOODVOLUME" VOLUME="0.5" </COMPARTMENT>
   </COMPARTMENTLIST>
\end{verbatim} }

{\footnotesize
\begin{verbatim}
   <SPECIES STATUS="Fixed" COMPARTMENT="BLOODVOLUME"
            VALUE="0.5"> Glucose </SPECIES>
\end{verbatim} }


If we leave out the issue of compartmentation for the moment then we could
write the species list in the following format:

{\footnotesize
\begin{verbatim}
  <SPECIESLIST>
     <SPECIES STATUS="Fixed" VALUE="0.5"> Glucose </SPECIES>
     <SPECIES STATUS="Float" VALUE="5.5"> ATP     </SPECIES>
     <SPECIES STATUS="Float" VALUE="0.0"> Glucose_6_P </SPECIES>
     <SPECIES STATUS="Float" VALUE="0.1"> ADP     </SPECIES>
  </SPECIESLIST>
\end{verbatim} }


Should the status and value fields be optional and if so what should be their
agreed defaults?


\subsection*{Naming Species}

We should agree on what constitutes a legal name for a molecular species. I
have tended to follow the rule that molecular species must start with a letter
or an underscore character. Some people would like to relax this rule and
allow digits at the start of a name, eg 3PG and even to allow spaces and others
characters within a name, eg 2,3 DPG. I think this is a bit of a Pandora's box
and would play havoc with language parsing algorithms.But it needs thinking
about.

\begin{verbatim}
 Species Name   ::= Letter | `_' { Letter | Digit | `_' }
\end{verbatim}


\section*{Reaction List}

The reaction list is the main body of the file and contains all the structural
and kinetic information. One question is whether we should in fact mix
structural with kinetic information and perhaps kinetic information should have
its own separate section. Some models are purely structural and don't need
kinetic data, on the other hand it is an easy matter simply to put in a dummy
kinetic law if the kinetic and structural information share the same section and
specific kinetic data is note available.

A possible structure for a reaction list could be a list of one or more
reaction elements where a reaction element gives details on a particular
reaction.

{\footnotesize
\begin{verbatim}
  <REACTIONLIST>
     <REACTION>

     </REACTION>

     .
     .
     .
  </REACTIONLIST>
\end{verbatim} }


\subsection*{Reaction}

For a particular reaction the minimal information includes, an identification
label, the names of all the reactants and products together with their
stoichiometries, that is the number of molecular species that take part in a
single transformation.

An unbelievable reaction might be:

\begin{verbatim} Glucose + 2ATP -> Glucose-6-P + 3ADP  Rate Law: k1*Glucose*ATP \end{verbatim}

and could be represented using:

{\footnotesize
\begin{verbatim}
  <REACTION NAME="BranchFlux" TYPE="Reversible" COMMENT="Some suitable comment here">

     <!-- The following element should be optional and could possibly be -->
     <!-- better represented with a simple data base reference -->
     <BIOLOGICALINFO ORGANISM="EColi"
                     COMPARTMENT="CYTOPLASM">
                     ECNUMBER="1.2.3.4">
     </BIOLOGICALINFO>

     <REACTANTLIST>
       <REACTANT STOICHIOMETRY="1/1"> Glucose </REACTANT>
       <REACTANT STOICHIOMETRY="2/1"> ATP </REACTANT>
     </REACTANTLIST>

     <PRODUCTLIST>
       <PRODUCT STOICHIOMETRY="1/1"> Glucose_6_P </REACTANT>
       <PRODUCT STOICHIOMETRY="3/1"> ADP </REACTANT>
     <\PRODUCTLIST>

     <!-- The following Element could be Optional -->
     <KINETICLAW>
        k1*Glucose*ATP
        <PARAMETERLIST>
           <P VALUE="3.4"> k1 </P>
        </PARAMETERLIST>
     </KINETICLAW>

  </REACTION>
\end{verbatim} }

All reactions must have an identification label attached to them and we need
to indicate whether they are reversible or not. Strictly speaking all
reactions are reversible so the designation seems pointless and for the
majority of models this is indeed the case, however certain structural
analyses, in particular elementary mode analysis requires this information in
order to carry out its computation. Since this information is quite
specialised and adds nothing to the models other than in the case when
elementary modes need to be computed, the information should be optional. I've
also added a specific reaction (but optional) comment field.


I've also indicated a separate section within a specific reaction which I
called BioliogicalInfo (for want of a better word, perhaps Source would be
better) which would be useful for `real-world' models. Again this needs some
discussion, for example do we need it, if so what information should we
include. We could of course simply make a reference to a remote data base
entry.

We then come to the reactant and product lists. Both of these are structured
in the same way except one refers to the reactants and other the products. The
only point worth making here is the form of the stoichiometry matrix. Note
that we could use a simple integer here but I would recommend a ratio of
integers here because structural analyses such as elementary modes (and any
arithmetic manipulation of the stoichiometry matrix) would benefit from this.


The last section is the kinetic law. At the moment there is only two types of
information here, an infix expression representing the kinetic law (this
assumes that applications that read these files can parse expressions in infix
notation) and the values for any kinetic constants that appear in the rate
expression.

\section*{Forcing Functions}

One thing that might be worth including is a section for forcing functions. I
know quite a few models that employ these, in particular they are very useful
for computing the equilibrium concentrations of rapidly equilibrating
reactions. Useful because it means you can take these fast reactions away from
the integrator and this makes life easier for the integrator.

This probably needs discussion.


\section*{DTD or XML Schema}

There is probably little point to even start defining a DTD or XML scheme
until we've decided on the basic structure. But when that time arrives we then
have to decide whether to go for a DTD or an XML Schema approach.

\section*{Implementation}

As an experiment I have implemented most of the ideas in this document into
Jarnac. Programming on the Windows platform makes life a bit easier for XMLers
because IE5 comes with a built-in XML parser with a fully implemented DOM
interface so its quite straight forward to read XML. Jarnac itself has built
into it what I call a NOM (Network Object Model) from which it is easy to
write out an XML file.

\section*{What's to be done?}

Here are some things that I think need to be considered, there are bound to be
other issues I've not even considered.

\begin{itemize}
\item Distribute an initial document to participating parties to collect
opinions and suggestions on the spec so far.
\item Ideas that need sorting out or require more discussion include:
\begin{itemize}
\item Versioning
\item Header information
\item DTD or XML Schema
\item Compartmentation
\item An agreement on what consitutes a legal species name
\item Structural info separate from kinetic?
\item Forcing Functions
\item Format for kinetic rate laws, infix, reverse-polish etc?
\item Other things I've not thought of
\end{itemize}
\end{itemize}


\end{document}
