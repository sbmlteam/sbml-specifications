\newcounter{arraysCtr}
\newcommand{\printValid}{\validRule{arrays-\arabic{arraysCtr}\addtocounter{arraysCtr}{1}}}
\section{Validation Rules}
\label{validation}

This section summarizes all the conditions that must (or in some cases,
at least \emph{should}) be true of an SBML Level~3 Version~1 model that
uses the Arrays package.  We use the same
conventions as are used in the SBML Level~3 Version~1 Core specification
document.  In particular, there are different degrees of rule
strictness.  Formally, the differences are expressed in the statement of
a rule: either a rule states that a condition \emph{must} be true, or a
rule states that it \emph{should} be true.  Rules of the former kind are
strict SBML validation rules---a model encoded in SBML must conform to
all of them in order to be considered valid.  Rules of the latter kind
are consistency rules.  To help highlight these differences, we use the
following three symbols next to the rule numbers:

\begin{description}

\item[\hspace*{6.5pt}\vSymbol\vsp] A \vSymbolName indicates a
  \emph{requirement} for SBML conformance. If a model does not follow
  this rule, it does not conform to the Arrays
  specification.  (Mnemonic intention behind the choice of symbol:
  ``This must be checked.'')

\item[\hspace*{6.5pt}\cSymbol\csp] A \cSymbolName indicates a
  \emph{recommendation} for model consistency.  If a model does not
  follow this rule, it is not considered strictly invalid as far as the
  Arrays specification is concerned; however, it
  indicates that the model contains a physical or conceptual
  inconsistency.  (Mnemonic intention behind the choice of symbol:
  ``This is a cause for warning.'')

\item[\hspace*{6.5pt}\mSymbol\msp] A \mSymbolName indicates a strong
  recommendation for good modeling practice.  This rule is not strictly
  a matter of SBML encoding, but the recommendation comes from logical
  reasoning.  As in the previous case, if a model does not follow this
  rule, it is not considered an invalid SBML encoding.  (Mnemonic
  intention behind the choice of symbol: ``You're a star if you heed
  this.'')

\end{description}

The validation rules listed in the following subsections are all stated
or implied in the rest of this specification document.  They are
enumerated here for convenience.  Unless explicitly stated, all
validation rules concern objects and attributes specifically defined in
the Arrays package.

For \notice convenience and brievity, we use the shorthand
``\token{arrays:x}'' to stand for an attribute or element name \token{x}
in the namespace for the Arrays package, using
the namespace prefix \token{comp}.  In reality, the prefix string may be
different from the literal ``\token{arrays}'' used here (and indeed, it
can be any valid XML namespace prefix that the modeler or software
chooses).  We use ``\token{arrays:x}'' because it is shorter than to
write a full explanation everywhere we refer to an attribute or element
in the Arrays package namespace.

\subsubsection*{General rules about the Arrays package}
\setcounter{arraysCtr}{10101} 
\printValid{To conform to Version 1 of the Arrays package
  specification for SBML Level~3, an SBML document must declare the
  use of the following XML Namespace: \\ \textls[-25]{\uri{http://www.sbml.org/sbml/level3/version1/arrays/version1}}. (References:
  SBML Level~3 Package Specification for Arrays, Version~1, \sec{xml-namespace}.)}
  
\printValid{Wherever they appear in an SBML document,
  elements and attributes from the Arrays
  package must be declared either implicitly or explicitly to be in the
  XML namespace
     \\ \textls[-25]{\uri{http://www.sbml.org/sbml/level3/version1/arrays/version1}}.
  (References: SBML Level~3 Package Specification for Arrays,
  Version~1, \sec{xml-namespace}.) }

\subsubsection*{Rules for the extended \class{SBML} class} 
\setcounter{arraysCtr}{10201}

\printValid{In all SBML documents using the Arrays package, the \SBML object must include a value for
  the attribute \token{arrays:required} attribute.  (References: SBML Level~3 Version~1 Core, Section~4.1.2.)}
  
\printValid{The value of attribute \token{arrays:required} on
  the \SBML object must be of the data type \token{boolean}.
  (References: SBML Level~3 Version~1 Core, Section~4.1.2.) }
  
\printValid{The value of attribute \token{arrays:required} on
  the \SBML object must be set to \val{false} (References: \sbmlthreearrays, \sec{xml-namespace}.) }
 
\subsubsection*{General rules about MathML content in the Arrays Package} 
\printValid{Wherever MathML content appears in an SBML document, the MathML
content must be placed within a math element, and that math element
must be either explicitly or implicitly declared tobe in the XML
namespace   \textls[-25]{\uri {http://www.w3.org/1998/Math/MathML”}}.
 (References:SBML Level~3 Version~1 Core, Section~3.4.)}

\printValid{The following is a list of the only MathML 2.0 elements
  permitted in the Arrays package:  \token{vector},
  \token{selector},\token{lowlimit}, \token{uplimit}
  \token{condition},\token{sum}, \token{product},\token{forall},
  \token{exists},\token{mean}, \token{sdev}, \token{variance},
  \token{median}, \token{mode}, \token{moment}, and \token{momentabout}.
  (References: SBML Level~3 Package Specification for Arrays,
  Version~1, \sec{math-formulas}.) }

\printValid{The first argument of a MathML \token{selector} must be a
valid identifier to a MathML \token{vector} object or an extended
\SBase with a \Dimension object. }

\printValid{The arguments of a MathML \token{selector} other than the
  first argument must be evaluated to non-negative integers.}

\printValid{The arguments of a MathML \token{vector} must agree in
  size.}

%Reference? MathML or arrays?
\printValid{For MathML unary operations involving a MathML \token{vector} or a \Dimension object, the operator is performed element-wise on each entry of the array.}

\printValid{For MathML operations with two or more operands involving a MathML \token{vector} or a \Dimension object, the operand is performed element-wise.}

\printValid{For MathML operations with two or more operands involving a MathML \token{vector}, the size of the arguments should match.}

\printValid{For MathML operations with two or more operands involving a \Dimension object, the size and the array dimension of the arguments should match.}

\subsubsection*{Rules for the extended \class{SBase} abstract class} 
\setcounter{arraysCtr}{20101} 
\printValid{Any object derived from the extended \SBase class
  (defined in the Arrays package) may contain
  at most one instance of a \ListOfDimensions subobject.
  (References: SBML Level~3 Package Specification for Arrays, Version~1, \sect{sec:dimension}.) }

\printValid{Apart from the general notes and annotation
  subobjects permitted on all SBML objects, a \ListOfDimensions
  container object may only contain \Dimension objects.
  (References: SBML Level~3 Package Specification for Arrays,
  Version~1, \sec{sec:dimension}.) }

\printValid{The \ListOfDimensions in an \SBase object must have a
    \Dimension object with \\ \token{arrays:arrayDimension} attribute set
    to 0 before adding a  \Dimension object with \\ \token{arrays:arrayDimension} attribute set
    to 1. Similarly, the \ListOfDimensions in an \SBase object must have
    \Dimension objects, where one of them has \token{arrays:arrayDimension} attribute set
    to 0 and the other set to 1 before adding a  \Dimension object with  \token{arrays:arrayDimension} attribute set
    to 2. (References:
  \sbmlthreearrays, \sec{sec:dimension}.) }

\printValid{The \ListOfDimensions in an \SBase object must not have multiple
    \Dimension objects with the same \token{arrays:arrayDimension} attribute. (References:
  \sbmlthreearrays, \sec{sec:dimension}.) }


\printValid{A \ListOfDimensions object may have the optional SBML core attributes \token{metaid} and \token{sboTerm}.  No other attributes from the SBML Level~3 Core namespace or the Arrays namespace are permitted on a \ListOfDimensions object.  (References: \sbmlthreearrays, \sec{sec:dimension}.) }


\printValid{Any object derived from the extended \SBase class
  (defined in the Arrays package) may contain
  at most one instance of a \ListOfIndices subobject.
  (References: SBML Level~3 Package Specification for Arrays, Version~1, \sect{sec:index}.) }

\printValid{Apart from the general notes and annotation
  subobjects permitted on all SBML objects, a \ListOfIndices
  container object may only contain \Index objects.
  (References: SBML Level~3 Package Specification for Arrays,
  Version~1, \sec{sec:index}.) }

\printValid{The \ListOfIndices in an \SBase object must not have multiple
    \Index objects with the same \token{arrays:arrayDimension} attribute. (References:
  \sbmlthreearrays, \sec{sec:index}.) }


\printValid{A \ListOfIndices object may have the optional SBML core attributes \token{metaid} and \token{sboTerm}.  No other attributes from the SBML Level~3 Core namespace or the Arrays namespace are permitted on a \ListOfIndices object.  (References: \sbmlthreearrays, \sec{sec:index}.) }

\printValid{The following are the elements that are permitted to have
  a \ListOfDimensions: \token{Parameters},
\token{Compartments},
\token{Species},
\token{Reactions},
\token{Species references},
\token{Rules},
\token{Initial assignments},
\token{Events},
\token{Event assignments}, and
\token{Constraints}.  (References: \sbmlthreearrays, \sec{sec:dimensionUsage}.)}

\printValid{The following are the elements that are permitted to have
  a \ListOfIndices:
\token{Model} to reference a \token{conversionFactor} element,
\token{Species} to reference a \token{compartment} or a
\token{conversionFactor} element,
\token{Reactions} to reference a \token{compartment},
\token{Initial assignments}  to reference a \token{symbol},
\token{Rules}  to reference a \token{variable},
\token{Species references}  to reference a \token{species}, and
\token{Events assignments}  to reference a \token{variable}.  (References: \sbmlthreearrays, \sec{sec:index}.)}

\subsubsection*{Rules for \class{Dimension} objects}
\setcounter{arraysCtr}{20201} 

\printValid{An \Dimension object may have the optional SBML
  Level~3 Core attributes \token{metaid} and \token{sboTerm}.  No
  other attributes from the SBML Level~3 Core namespace are permitted
  on a \Dimension object.  (References: SBML Level~3 Version~1 Core,
  Section~3.2.) }

\printValid{A \Dimension object must have a value
  for the attributes
  \token{arrays:arrayDimension} and \\
  \token{arrays:size}, and may additionally have the
  attributes \token{arrays:id} and \token{arrays:name}. (References: \sbmlthreearrays,
  \sec{sec:dimension}.) }


\printValid{The value of the
  \token{arrays:arrayDimension} attribute, if set on a given
  \Dimension object, must be either 0, 1 or 2.  (References:
  \sbmlthreearrays, \sec{sec:dimension}.) }

\printValid{The value of the
  \token{arrays:size} attribute, if set on a given
  \Dimension object, must be an valid \token{SIdRef} that of type \token{Parameter}.  (References:
  \sbmlthreearrays, \sec{sec:dimension}.) }

\printValid{The value of the \token{Parameter} referenced by the
  \token{arrays:size} attribute must be both a non-negative scalar and
  constant. (References:
  \sbmlthreearrays, \sec{sec:dimension}.) }

\subsubsection*{Rules for \class{Index} objects} 
\setcounter{arraysCtr}{20301}

\printValid{An \Index object may have the optional SBML
  Level~3 Core attributes \token{metaid} and \token{sboTerm}.  No
  other attributes from the SBML Level~3 Core namespace are permitted
  on a \Index object.  (References: SBML Level~3 Version~1 Core,
  Section~3.2.) }

\printValid{An \Index object must have a value
  for the attributes
  \token{arrays:arrayDimension}, and \\
  \token{arrays:referencedAttribute} . (References: \sbmlthreearrays,
  \sec{sec:index}.) }

\printValid{The value of the
  \token{arrays:referencedAttribute} attribute, if set on a given
  \Index object, must be an existing attribute that references a
  valid \token{SIdRef} with a value.  (References:
  \sbmlthreearrays, \sec{sec:index}.) }

\printValid{The value of the
  \token{arrays:arrayDimension} attribute, if set on a given
  \Index object, must be either 0, 1 or 2.  (References:
  \sbmlthreearrays, \sec{sec:index}.) }

\printValid{An \Index object must have exactly one MathML math element . (References: \sbmlthreearrays,
  \sec{sec:index}.) }

\printValid{The MathML math element in an \Index object must only reference \Dimension objects with \token{arrays:size} attribute value that is a non-negative and constant integer, and scalar values. }

\printValid{The MathML math element in an \Index object must be
  evaluated to a non-negative integer value that is less than the value of the
  \token{arrays:size} attribute. }

% If you look at the apdx-validation.tex file for any of the accepted packages you will see the macros in use.

% You can also probably cut and paste the opening paragraphs :-)

% Note with the numbers for arrays they will be

% arrays-nnnnn

% and then the first three numbers should be consistent with the categories from core for the more general 10xxx types rules; so
% 101xx for general xml rules
% 102xx for mathml rules
% etc

% Specific component type rules would be 20nxx where n is a new number for each component.

% So Dimension related rules would be 201xx

% and Index ones would be 202xx

%% -------

%  Yes indeed.  I assume you are using the sbmlpkgspec latex class.  In the documentation for that class (which is in a PDF file with the sbmlpkgclass style file kit, https://sourceforge.net/projects/sbml/files/specifications/tex/) see section 2.9.

\section{Example Algorithm for Producing a ``flattened'' Model}
\label{flatten}
