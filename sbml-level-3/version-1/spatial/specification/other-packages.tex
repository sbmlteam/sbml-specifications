\section{Interaction with other packages}
\begin{blockChanged}
The SBML Spatial package was designed to be orthogonal to other SBML packages, and as such should interact freely with all of them.  However, there are no current implementations that take use of these interactions, so the following descriptions should all be taken as guidelines for future development, and do not represent established protocols at this time.

\subsection{SBML Level~3 Version~2}
This package may be used with either SBML Level~3 Version~1 Core, or SBML Level~3 Version~2 Core, with no essential change in functionality.  SBML L3v2 adds an 'id' and 'name' to all constructs, so a few more Spatial elements that don't currently have an id would gain one.  Those ids would fall in the SId namespace and not interact with the rest of the Spatial package.

Even though Level~3 Version~2 elements may reference package elements in the SId namespace with mathematical meaning, it is impossible to do this with Spatial elements, as the ids of all elements with mathematical meaning are in the SpId namespace instead.  Both versions of SBML Level~3 should handle this the same way.


\subsection{The \token{comp} package}
The Hierarchical Model Composition package allows models to be composed hierarchically, with submodels that become part of the containing model, and connections that define how to integrate the submodels into the containing model.  Spatial models may be composed in this way just like other SBML constructs, with the caveat that because Spatial ids are in the SpId namespace, and because comp does not come with an element of type SpIdRef, metaids must be used instead.

Also, since there may only be a single \Geometry object child of the \Model, the \Geometry of any child submodel must be integrated into the parent model's \Geometry:  the \token{id}, \token{name}, and \token{coordinateSystem} of the parent remains unchanged, and the children of the \Geometry should be merged into the parent \Geometry in much the same way that the children of the child \Model are merged into the parent \Model.


\subsection{The \token{distrib} package}
The Distributions package allows modelers to include new MathML \token{csymbol} constructs that define draws from distributions, and the ability to define the uncertainty in any element with mathematical meaning.  Both could be employed as-is by users of the Spatial package:  the \token{csymbol} constructs can be used in spatial MathML, and in core MathML to affect spatial elements through the \SpatialSymbolReference link.  Any Spatial element with mathematical meaning could additionally be given an \class{Uncertainty} child defining its uncertainty.


\subsection{The \token{multi} package}
The Multistate and Multicomponent package allows modelers to define template species and reactions that are only realized upon actually simulating the model.  In principle, the extended \Species and \Reaction objects defined in that package should be able to be placed in a spatial context.

The work involved in interpreting and simulating such a model would be substantial, however, and more research needs to be done to identify best practices in using the two packages together.


\subsection{The \token{qual} package}
The Qualitative Models  package allows modelers to define interaction networks that are defined by element state instead of element levels.  In principle, such state transition modeling could be modeled in a spatially-distinguished context, but no effort has been made to do so at this time.  Mechanically, new extensions would have to be created to allow the \class{QualitativeSpecies} and \class{Transition} elements from the \token{qual} package to be spatially defined before any such model definitions could be created.


\subsection{The \token{layout} and \token{render} packages}
The Layout Package and the Render Package allow the visualization of SBML models.  In principle, both could be used as-is to display the underlying reaction networks being used in a spatial model.  However, there are no special constructs available to visualize the geometry defined in the spatial package: simulators typically display this geometry to the user with species levels superimposed, but nothing in \token{layout} or \token{render} would help the simulator displaying this information.  Thus, at present, the \Geometry itself is used for visualization of spatial simulations, while \token{layout} and \token{render} are used separately to visualize the reaction network.
\end{blockChanged}
