% -*- TeX-master: "multi" -*-

\section{Background and context}
\label{def:Background}

Rule-based modeling (more specific: ``Domain-detailed reaction rule modeling'') approaches (\BioNetGen[\cite{ref:bionetgen2009}], \Kappa[\cite{ref:kappa2004}], and \Simmune[\cite{ref:simmune2012, ref:simmune2006}]) define rules for interactions between pairs of molecule domains, specifying how the interactions depend on particular states of the molecules (pattern) and their locations in specific compartments. In order to generate networks of biochemical reactions these rules are applied to the molecular components of the systems to be modeled, either at the beginning of the modeling (simulation) process or ``on the fly'' (as molecule complexes emerge from the interaction rules). Expressing such rule-based reaction networks using the concepts of \Species and \Compartment in SBML (L3 core and L2) can be difficult for rules and molecule sets that lead to large numbers of resulting molecular complexes. It would therefore be desirable to have an SBML standard for encoding rule-based \smodels\ using their ``native'' concepts for describing reactions instead of having to apply the rules and unfold the networks prior to encoding in an SBML format.

We proposed a revised proposal of the \multi: ``Multistate, Multicomponent and Multicompartment Species Package for SBML Level 3''  (abbreviated as Multi)[\cite{ref:revisedMulti} and \cite{ref:multiproposal280}] which takes the scopes and some data structures developed in \multiOneProposalWC\ and addresses main issues arising from a rule-based modeling point of view with the data structures consistent with that used in the available rule-based modeling tools. This specification documentation has been developed from the most recent release of the Multi specification in June 2015 [\cite{ref:multispecV104}] with the following modifications based on the discussion during and after COMBINE 2015 [\cite{ref:combine2015}]: 

\begin{itemize}
 \item Drop the \occurAtt\ attribute in the class of \SpeciesTypeInstance.
 \item Drop the \occurAtt\ attribute in the class of \SpeciesTypeComponentIndex.
 \item Drop the class of \DenotedSpeciesTypeComponentIndex.
 \item Revise the scope of \PossibleSpeciesFeatureValue ids to be global.
\end{itemize}


%-------------------------------------------------------
\subsection{Past work on this problem or similar topics}
\label{def:Past_work}

\begin{itemize}
 \item 
  Nicolas Le Nov\`ere and Anika Oellrich proposed the previous version of the Multi proposal[\cite{ref:multi1}]. The development became stalled after 2010. 
 
 \item In August 2012, Fengkai Zhang from the \Simmune\ group presented `` Draft for discussion SBML Proposals for Revised Multi, Simple Spatial and Multi-Spatial Extensions'' at COMBINE 2012[\cite{ref:revisedMulti}]. The three proposals cover the goals and scope of \multiOneProposal, revise it and add some new features that improve usage of the proposal for rule-based approaches.
 
 \item Based on the discussions and suggestions received during COMBINE 2012 as well as on feedback from the SBML discussion forum, \multiTwoProposalVerTwo\ was released to the SBML-Multi community, which integrates and covers most of the features in the three previous proposals of Aug 2012. 
 
 \item In May 2013, a new reversion of the Multi proposal [\cite{ref:multiproposal280}] was released before the meeting of HARMONY 2013. The extended \ExCompartment class and its related classes have been reorganized. All optional boolean attributes have been removed/replaced. A new optional Multi attribute, \val{whichValue}, was added to the \token{ci} elements in \class{KineticLaw} to identify  the sources of \species. (Lucian Smith gave many comments/suggestions about this proposal and William Hlavacek gave thoughtful feedback about the \BioNetGen\ example in this proposal). This revision 280 was presented at HARMONY 2013  [\cite{ref:harmony2013}] with new features to configure multiple occurrences of \SpeciesFeatureType. Several new or revised features were discussed during and after HARMONY 2013, including multiple occurrences of \SpeciesFeatureType, multiple copies of \SpeciesTypeInstance, the \numericValueAtt\ attribute for \PossibleSpeciesFeatureValue and concentration summation of pattern \species. These features are covered or updated in this specification.
 
 
 \item A draft specification V1.0.1 was released in Sep 2013 [\cite{ref:multispecV101}] and was presented in COMBINE 2013 [\cite{ref:combine2013}]. This version of the specification addresses the scenario of multiple occurrences of identical components and/or identical features.
 
 \item In COMBINE 2014, a discussion of the \multi\ package focused on how to facilitate tools to export and import \smodels\ encoded in the Multi format[\cite{ref:combine2014}].
 
 \item The drafted specification v1.0.4[\cite{ref:multispecV104}] became relatively stable after COMBINE 2014. Since then effort had been put on development of the validation rules and implementation of the features described in the specification into libsbml library. 
 
\end{itemize}

\clearpage

