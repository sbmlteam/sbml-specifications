% -*- TeX-master: "main"; fill-column: 72 -*-

\section{Transformations}
\label{apdx-transformations}

There are four basic transformation operations that can be combined in a affine transformation matrix. 

\subsection{Translation}
Translating something means moving it some distance along one or more of the axes. The corresponding 2D transformation matrix is

\hspace*{0.4cm}
\begin{center}
\begin{math}\left[ \begin{array}{ccc} 1 & 0 & tx \\ 0 & 1 & ty \\ 0 & 0 & 1\end{array}\right]\end{math}
\end{center}
\hspace*{0.4cm}

where tx and ty are the distance along the x and y axes by which the object shall be moved.

\subsection{Scaling}
Scaling means to multiply all coordinate components of an object by a certain value.
The corresponding 2D transformation matrix is

\hspace*{0.4cm}
\begin{center}
\begin{math}\left[ \begin{array}{ccc} sx & 0 & 0 \\ 0 & sy & 0 \\ 0 & 0 & 1\end{array}\right]\end{math}
\end{center}
\hspace*{0.4cm}

where sx and sy are the scaling factors along the x and y axis respectively.

\subsection{Rotation}
With a rotation, an object can be rotated around the origin of the coordinate system.
The corresponding 2D transformation matrix is

\hspace*{0.4cm}
\begin{center}
\begin{math}\left[ \begin{array}{ccc} cos(\alpha) & -sin(\alpha) & 0 \\ sin(\alpha) & cos(\alpha) & 0 \\ 0 & 0 & 1\end{array}\right]\end{math}
\end{center}
\hspace*{0.4cm}

where $\alpha$ is the angle of rotation around the origin.

\subsection{Skewing}
Skewing is the least used operation and we have to distinguish between skewing along the x or the y axis.
The corresponding 2D transformation matrices are

\hspace*{0.4cm}
\begin{center}
\begin{math}\left[ \begin{array}{ccc} 1 & tan(\alpha) & 0 \\ 0 & 1 & 0 \\ 0 & 0 & 1\end{array}\right]\end{math}
\end{center}
\hspace*{0.4cm}


\hspace*{0.4cm}
\begin{center}
\begin{math}\left[ \begin{array}{ccc} 1 & 0 & 0 \\ tab(\beta) & 1 & 0 \\ 0 & 0 & 1\end{array}\right]\end{math}
\end{center}
\hspace*{0.4cm}

where $\alpha$ is the skewing angle of skewing along the x axis and $\beta$ is the angle for skewing along the y axis.

Combining several of the operations above means multiplying the transformation matrices that belong to the individual operations.
Depending on the matrices that are multiplied, the order of the operations matter, e.g. it makes a difference if an object is translated before it is rotated or if it is rotated first.

If an object specifies a transformation, this transformation is to be applied to the object prior to any other coordinate properties of the object. E.g. if a rectangle specifies a position of $x=10$ and $y=20$ and it also specifies a rotation by 45 degrees, the rotation is applied before the object is placed at $P(10,20)$.
The transformation for an object is always in relation to the objects view port. For most render objects, this would be the bounding box of the corresponding layout object. For layout curves, e.g. in reaction glyphs or species reference glyphs, the view port is the complete diagram.
For objects defined in line endings, the view port is the bounding box of the line ending before it is applied to the line.

\vspace*{0.25cm}
{\large
{\bf
example:
}
}

{\footnotesize
\begin{example}
 <g ...>
   <text x="50%" y="50%" text-anchor="middle" stroke="#FF0000"
        font-family="serif" font-size="20.0" 
        transform="1.0, 3.0, 2.5, 1.4, 4.0, 5.0">This is a Text</text>
      ...
</g> 
\end{example}
}

