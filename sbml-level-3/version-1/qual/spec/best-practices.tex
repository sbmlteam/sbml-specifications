% -*- TeX-master: "main"; fill-column: 72 -*-

\section{Best practices}
\label{best-practices}

\ALL To be valid, the SBML root element must express the requirement of this package: \texttt{<sbml $\dots$ qual:required="true" $\dots$ >}.

\medskip
\PN In Petri nets the initial conditions are part of the model, the \attr{initialLevel} must be defined.
To represent unbounded places, the \attr{maxLevel} should be not specified.

\medskip
\LRG Discussions are still ongoing about the possible (but some times convenient to avoid cumbersome descriptions) incoherency of the \qual{FunctionTerm} elements. For the moment, here are the guidelines to ensure coherent definitions:
\begin{itemize}
\item The \qual{FunctionTerm} elements of all the transitions targeting the same output should be "coherent": the conditions of two \qual{FunctionTerm} elements, leading to different effects on the level/symbol of the output, should not be fulfilled at the same time( i.e. they should be exclusive).
\item If several \qual{FunctionTerm} elements lead to the same effect on the level/symbol of the same output, then the importing tool should consider the disjunction (OR) on the conditions of the terms. 
\end{itemize}

\medskip
\LRG To declare external nodes (ones that have no Boolean expression/truth table associated with them), one should set the attribute \attr{boundaryCondition} of the  \qual{QualitativeSpecies} to TRUE: 
\begin{verbatim}
<qualitativeSpecies id="EGF" maxLevel="1" boundaryCoundition="true" 
                                                  compartment="extracellular"/> 
\end{verbatim}

%Thus, the  \qual{QualitativeSpecies} is defined as being in the boundary on the  system, and cannot be used as an output in any transition."
\medskip
\LRG To declare a "delay" node, which is specified to delay its state update for $k$ iterations, one should set, for all the \qual{Transition} elements having this node as their (unique) output, the attribute \attr{temporisationType} to the value \const{timer} and the \attr{temporisationValue} to $k$. 

\medskip
\LRG To declare a "sustain" node, which is specified to sustain (i.e., to remain in) its latest state for the next $k$ iterations, one should set, for all the \qual{Transition} elements having this node as their (unique) output, the attribute \attr{temporisationType} to the value \const{sustain} and the \attr{temporisationValue} to $k$. 


% section recommended Practices (end)
