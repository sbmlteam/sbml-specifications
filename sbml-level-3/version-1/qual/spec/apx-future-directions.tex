% -*- TeX-master: "main"; fill-column: 72 -*-

\section{Future directions}
\label{apdx-future}

To account for qualitative models where parameters are not (all) instantiated, as well as models for which timing constraints are specified, an extension of the current specification was contemplated. Here, we briefly recapitulate the elements and attributes that have been discarded in the current specification, with the intent to fuel discussions on future extensions. \hl{We also point out other extensions that may be considered in the near future, since they were evoked while discussing the current specification.}

Finally, we briefly comment on the possible use of this qual package to represent Petri net  models. 

%\begin{figure}
%\includegraphics{figs/qual_future_directions.pdf}
%\caption {The definitions of the classe \listOf{SymbolicValues}, \qual{SymbolicValue} and \qual{TemporisationMath} and additional related attributes for existing classes.}
%  \label{qual_future_directions}
%\end{figure}


\subsection{Symbols}

\subsection*{Definition of \qualt{SymbolicValue}} % (fold)

\begin{figure}[hb]
  \includegraphics{figs/qual-qualitative-species-future-uml.pdf}
  \caption{Possible future extensions of the \QualitativeSpecies class.}
  \label{qual_future_directions}
\end{figure}

The \qual{QualitativeSpecies} element may contain at most one \listOf{SymbolicValues} that contains zero or more \qual{SymbolicValue}s. An empty list is allowed, and useful for e.g. adding annotations. 
The \qual{SymbolicValue} element defines a non instantiated parameter. Such symbols may represent the different solutions of piecewise linear differential equations, along with different thresholds.

\paragraph{The \attr{id} and \attr{name} attributes}
These attributes are used according to the SBML L3.1 Section 3.3. The attribute \attr{id} is mandatory and \attr{name} is optional. 

\paragraph{The \attr{rank} attribute}
The \attr{rank} is an \type{integer} that defines the position of the symbol in the \listOf{SymbolicValues}. This attribute is optional.


\paragraph{The \attr{thresholdLevel} and \attr{thresholdSymbol} attributes:} %% of Input
The \attr{thresholdLevel} is an \type{integer} and \attr{thresholdSymbol} is a \type{SIdRef}. They are optional and exclusive.

\paragraph{The \attr{resultLevel} and \attr{resultSymbol} attributes:} %% in FunctionTerm
The result of the term is described by a \attr{resultLevel} or a \attr{resultSymbol}. Both are optional, but one of them must be defined.


\const{assignmentSymbol}: The symbol associated to the \attr{qualitativeSpecies} is set to the \attr{resultSymbol} of the selected term.

\subsection{Temporisation}

\begin{figure}[hb]
  \includegraphics{figs/qual-transition-future-uml.pdf}
  \caption{Possible future extensions of the \Transition class.}
  \label{qual_future_directions}
\end{figure}

\paragraph{The \attr{temporisationType} attribute:} %% attribute of Transition
The \attr{temporisationType} is an \type{enumeration} the ``temporisation'' of the \qual{Transition}, that is the updating policy associated with the \qual{Transition}. It can be set to \const{timer}, \const{priority}, \const{sustain}, \const{proportion} or \const{rate}.
This attribute is optional. 



\paragraph{The \attr{temporisationValue} attribute and the \sbml{TemporisationMath} element:}
The attribute \attr{temporisationValue} and the element \sbml{TemporisationMath} allow the specification of the ``temporisation'' of the \qual{Transition} under the corresponding \qual{FunctionTerm}. Both are optional. Depending on the value of the \attr{temporisationType}, either one or both could be used.

The \attr{temporisationValue} is a \type{double}. The element \sbml{TemporisationMath} holds a MathML function returning a \type{double}. 

\hl{
\subsection{Classes of models and random models}}
\hl{
Several comments indicate that a future extension could support the representation of classes of models ({\it i.e.} models that are not fully parametrised, meaning that e.g. the logical rule of a component is incomplete), or random models, e.g. where several logical rules are associated to a component, the choice of a rule in the course of the dynamical evolution being arbitrary.
This might be done by revising the requirement of certain elements as well as the current semantics of {\em function terms}.}

\hl{\subsection{Interaction with SBML Core concepts}}
\hl{
At the time at which this Qualitative Modelling specification was developed, the policy and process for interacting with SBML Level 3 Core constructs was undefined. Thus, this particular specification does not facilitate the use of Core constructs. It is anticipated that in the future the specification will be extended to allow the use of these constructs; in particular Parameters and Events.
}

\subsection{Petri net models}

%The current specification should cover the needs for standard Petri nets. However,  no group developing PN tools was involved in the definition of the qual package so far. 
\hl{Although this package was designed to apply for PN models, no implementation of this specification for Petri net models has been done so far. Moreover, as most PN models currently refer to metabolic or other reaction networks, PN tools dedicated to this modelling framework often provides support for the SBML core format (see e.g. \cite{snoopy10}). }

\textcolor{blue}{It is worth mentioning that High Level Petri Nets (also referred to as Coloured Petri Nets) might be targeted by this package. There are issues, already mentioned, regarding the use of multiple \Output elements with \token{transitionEffect} \val{assignmentLevel} which need to be considered. However, it seems likely that these types of models can be accommodated in future versions.}

Finally, the consideration of further elements and attributes to define temporized transitions (as described above), might be used to represent timed PN.
