% -*- TeX-master: "main"; fill-column: 72 -*-

\section{Introduction}
\label{intro}

In the context of SBML, ``hierarchical model composition'' refers to the ability to include models as submodels inside another model. The goal is to support the ability of modelers and software tools to do such things as (1) decompose larger models into smaller ones, as a way to manage complexity; (2) incorporate multiple instances of a given model within one or more enclosing models, to avoid literal duplication of repeated elements; and (3) create libraries of reusable, tested models, much as is done in software development and other engineering fields.

SBML Level 3 Version 1 Core~\citep{l3v1c}, by itself, has no direct support for allowing a model to include other models as submodels. Software tools either have to implement their own schemes outside of SBML, or (in principle) could use annotations to augment a plain SBML Level~3 model with the necessary information to allow a software tool to compose a model out of submodels.  However, such solutions would be proprietary and tool-specific, and not conducive to interoperability. There is a clear need for an official SBML language facility for hierarchical model composition.

This document describes a specification for an SBML Level~3 package that provides exactly such a facility.  \fig{fig1} illustrates some of the scenarios targeted by this package. 

\begin{figure}[hb]
  \includegraphics{figs/figure1}
  \caption{Three different examples of model composition scenarios. From left to right: (1) a model composed of multiple instances of a single, internally-defined submodel definition; (2) a model composed of a submodel that is itself composed of submodels; and (3) a model composed of submodels, one of which is defined in an external file.}
  \label{fig1}
\end{figure}

The effort to create a hierarchical model composition mechanism in SBML has a long history, which we summarize in \sec{background}.  It has also been known by different names.  In the beginning, it was called \emph{modularity} because it allows a model to be divided into structural and conceptual modules.  It was renamed \emph{model   composition} when it became apparent that the name ``modularity'' was easily confused with other notions modularity, particularly XHTML~1.1~\citep{xhtml} modularity, which concerns decomposition into separate files.  To make clear that the purpose is structural \emph{model} composition, regardless of whether the components are stored in separate files, the SBML community adopted the name SBML \emph{Hierarchical Model Composition}.

To support a variety of composition scenarios, this package provides for optional black-box encapsulation by means of defined data communication interfaces (here called \emph{ports}).  In addition, it also separates model \emph{definitions} (i.e., blueprints, or templates) from \emph{instances} of those definitions, it supports optional external file storage, and it allows recursive model decomposition with arbitrary submodel nesting.


\subsection{Proposal corresponding to this package specification}

This specification for Hierarchical Model Composition in SBML \changed{was originally} based on the proposal by the same authors, located at the following URL:

\begin{center}
  \vspace*{1ex}\small
  \url{https://sbml.svn.sf.net/svnroot/sbml/trunk/specifications/sbml-level-3/version-1/comp/proposal}
  \vspace*{1ex}
\end{center}

\changed{Version 2 of the specification was developed for use with \sbmlthreecore, to account for the changes made in that version of the core specification.  Any issues or proposals having to do with this specification should be reported at \url{https://sourceforge.net/p/sbml/sbml-specifications/}.}


\subsection{Package dependencies}

The Hierarchical Model Composition package has no dependencies on other SBML Level~3 packages.  It is also designed to work seamlessly with other SBML Level~3 packages.  For example, one can create a set of hierarchical models that also use Groups or Spatial Processes features. (If you find incompatibilities with other packages, please contact the Package Working Group for the Hierarchical Model Composition effort. Contact information is shown on the front page of this document.)


\subsection{Document conventions}
\label{conventions}

Following the precedent set by the SBML Level~3 Core specification document, we use UML~1.0 (Unified Modeling Language; \citealt{eriksson:1998,oestereich:1999}) class diagram notation to define the constructs provided by this package.  We also use color in the diagrams to carry additional information for the benefit of those viewing the document on media that can display color.  The following are the colors we use and what they represent:

\begin{itemize}

\item[\raisebox{2.75pt}{\colorbox{black}{\rule{0.8pt}{0.8pt}}}] \emph{Black}: Items colored black in the UML diagrams are components taken unchanged from their definition in the SBML Level~3 Core specification document.

\item[\raisebox{2.75pt}{\colorbox{mediumgreen}{\rule{0.8pt}{0.8pt}}}] \emph{\textcolor{mediumgreen}{Green}}: Items colored green are components that exist in SBML Level~3 Core, but are extended by this package.  Class boxes are also drawn with dashed lines to further distinguish them.

\item[\raisebox{2.75pt}{\colorbox{darkblue}{\rule{0.8pt}{0.8pt}}}] \emph{\textcolor{darkblue}{Blue}}: Items colored blue are new components introduced in this package specification.  They have no equivalent in the SBML Level~3 Core specification.

\end{itemize}

We also use the following typographical conventions to distinguish the names of objects and data types from other entities; these conventions are identical to the conventions used in the SBML Level~3 Core specification document:

\begin{description}
  
\item \abstractclass{AbstractClass}: Abstract classes are never instantiated directly, but rather serve as parents of other classes. Their names begin with a capital letter and they are printed in a slanted, bold, sans-serif typeface.  In electronic document formats, the class names defined within this document are also hyperlinked to their definitions; clicking on these items will, given appropriate software, switch the view to the section in this document containing the definition of that class.  (However, for classes that are unchanged from their definitions in SBML Level~3 Core, the class names are not hyperlinked because they are not defined within this document.)

\item \class{Class}: Names of ordinary (concrete) classes begin with a capital letter and are printed in an upright, bold, sans-serif typeface.  In electronic document formats, the class names are also hyperlinked to their definitions in this specification document. (However, as in the previous case, class names are not hyperlinked if they are for classes that are unchanged from their definitions in the SBML Level~3 Core specification.)

\item \token{SomeThing}, \token{otherThing}: Attributes of classes, data type names, literal XML, and tokens \emph{other} than SBML class names, are printed in an upright typewriter typeface.  Primitive types defined by SBML begin with a capital letter; SBML also makes use of primitive types defined by XML Schema~1.0~\citep{biron:2000,fallside:2000,thompson:2000}, but unfortunately, XML~Schema does not follow any capitalization convention and primitive types drawn from the XML~Schema language may or may not start with a capital letter.

\end{description}

For other matters involving the use of UML and XML, we follow the conventions used in the SBML Level~3 Core specification document.  




