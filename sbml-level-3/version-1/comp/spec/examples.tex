% -*- TeX-master: "main"; fill-column: 72 -*-

\section{Examples}
\label{examples}

This section contains a variety of examples of SBML Level~3 Version~1
documents employing the Hierarchical Model Composition package.

\subsection{Simple aggregate model}

The following is a simple aggregate model, with one defined model being
instantiated twice:

\exampleFile{figs/eg-simple-aggregate.xml}

In the model above, we defined a two-step enzymatic process, with
species \val{S} and \val{E} forming a complex, then dissociating to
\val{E} and \val{D}.  The aggregate model instantiates it twice, so the
resulting model (the one with the identifier \val{aggregate}) has two
parallel processes in two parallel compartments performing the same
reaction.  The compartments and processes are independent, because there
is nothing in the model that links them together.


\subsection{Example of importing definitions from external files}

Now suppose that we have saved the above SBML content to a file named
\val{enzyme\_model.xml}.  The following example imports the model
\val{enzyme} from that file into a new model:

\exampleFile{figs/eg-import-external.xml}

In the model above, we import \val{enzyme} twice to create submodels 
\val{A} and \val{B}.  We then create a
compartment and species local to the parent model, but use that
compartment and species to replace \val{comp} and \val{S}, respectively,
from the two instantiations of \val{enzyme}.  The result is a model with
a single compartment and two reactions; the species \val{S} can either
bind with enzyme \val{E} (from submodel \val{A}) to form \val{D} (from
submodel \val{A}), or with enzyme \val{E} (from submodel \val{B}) to
form \val{D} (from submodel \val{B}).


\subsection{Example of using ports}

In the following, we define one model (\val{simple}) with a single
reaction involving species \val{S} and \val{D}, and ports, and we
again import model \val{enzyme}:

\exampleFile{figs/eg-ports.xml}

In model \val{simple} above, we give ports to the compartment, the two
species, and the reaction.  Then, in model \val{com\-plex\-i\-fied}, we import
both \val{simple} and the model \val{enzyme} from the file
\val{enzyme\_model.xml}, and replace the simple reaction with the more
complex two-step reaction by deleting reaction \val{J0} from model
\val{simple} and replacing \val{S} and \val{D} from both models with
local replacements.  Note that it is model \val{simple} that defines the
initial concentrations of \val{S} and \val{D}, so our modeler set the 
version in the containing model to be replaced by those elements, instead of 
having the external version replacing those elements.  Also note that since
\val{simple} defines ports, the \token{port} attribute is used for the
subelements that referenced \val{simple} model elements, but
\val{idRef} still had to be used for subelements referencing
\val{enzyme}.

In the resulting model, \val{S} is converted to \val{D} by a two-step
enzymatic reaction defined wholly in model \val{enzyme}, with the
initial concentrations of \val{S} and \val{D} set in
\val{simple}.  If \val{simple} had other reactions that created
\val{S} and destroyed \val{D}, \val{S} would be created, would bind with
\val{E} to form \val{D}, and \val{D} would then be destroyed, even
though those reaction steps were defined in separate models.


\subsection{Example of deletion replacement}

In reference to the previous example, what if we would like to annotate
that the deleted reaction had been \emph{replaced} by the two-step
enzymatic process?  To do that, we must include those reactions as 
references in the parent model.  However, because we want to leave the complexity of the 
\val{E} and \val{ES} species to the \val{com\-plex\-i\-fied} model,
the two reactions in the containing model will contain almost no information
and serve only as placeholders for the express purpose of being replaced by
the fuller version in the \val{com\-plex\-i\-fied} submodel:  The first (\val{J0}) has
only \val{S} as a reactant, and is set to be replaced by \val{J0} from
\val{com\-plex\-i\-fied}, while the second (\val{J1}) has only \val{D} as a 
product, and is set to be replaced by \val{J1} from \val{com\-plex\-i\-fied}.

The following SBML fragment demonstrates one way of doing that.

\exampleFile{figs/eg-replacement.xml}

In the example above, we have recreated \val{enzyme} so as to provide it
with ports, then created dummy versions of the reactions from
\val{com\-plex\-i\-fied} so we can reference the deleted \val{oldrxn} in the
replacements lists.  Note that we list the deletion of \val{oldrxn}
for both of the two new reactions, since the full reactions still live in
\val{com\-plex\-i\-fied}.

The net result is a model where \val{com\-plex\-i\-fied} supplies the mechanism
for the conversion of \val{S} to \val{D}, while the initial conditions of
both species is set in \val{simple}.
