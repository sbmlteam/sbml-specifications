% -*- TeX-master: "main" -*-

\section{Background and context}
\label{background}

At the heart of many SBML models is mathematics. When present, the mathematics is considered to be of principle importance to understanding the model. Many SBML Level~3 packages replace the mathematical interpretation of elements present in \sbmlthreecore.  Models with these package namespaces defined must add a \token{required}=\val{true} flag on the definition of that namespace.  However, for software tools that wish to manipulate and augment SBML models with extended mathematics which they do not understand, a blanket \token{required}=\val{true} flag on the namespace is not granular enough.  The tools need to know which \emph{specific} model elements have had their mathematics changed (if any), so that they can understand which parts of the model are safe to manipulate freely and which must be treated with care.  In addition, some software tools add custom annotations to SBML models that are intended to change the mathematical meaning of the element.  Any model consumer that receives such a model but doesn't understand that annotation won't even have a \token{required} flag to let them know that the math is supposed to be different.

The Required Elements package is an exceedingly small package that allows model writers to declare specifically which components of the model have had their mathematics changed, by which package or annotation, and whether interpretation of those components in the absence of any package information results in a workable model.  It accomplishes this by defining a class that can be added as an optional child of any SBML element with mathematical meaning.
