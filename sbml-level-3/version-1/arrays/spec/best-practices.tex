% -*- TeX-master: "main"; fill-column: 72 -*-

\section{Best practices}
\label{best-practices}

In this section, we recommend a number of practices for using and
interpreting various constructs in the arrays package.
These recommendations are non-normative, but we advocate them strongly;
ignoring them will not render a model invalid, but may reduce
interoperability between software and models.

% \subsection{Examples contrasting the current \SBML L2 encoding with L3 and \FBC}
% These examples contrast some elements of an existing model, iJR904 from the \textsf{BiGG} Database encoded in the \textsf{COBRA} format \cite{ijr904, bigg, cobra} that have been translated into SBML Level~3 Version~1 using the \textsf{CBMPy} implementation of the \FBC package \cite{pysces, cbmpy} and \textsf{libSBML} experimental ver.~5.6.0 \cite{libsbml}.

% \subsubsection*{Objective function definition}
% \paragraph{Old style \SBML Level 2 objective}
% \exampleFile{examples/ex_objf_bigg.txt}

% \paragraph{New \SBML Level 3 style objective}
% \protect\exampleFile{examples/ex_objf_l3.txt}

% \newpage
% \subsubsection*{Species definition}
% It is particularly useful to contrast the differences in \Species definition as used in constraint based, genome scale models.

% \paragraph{Old \SBML Level 2 style species}
% To begin with we let's examine the \SBML Level~2 Version~1 species definition used by the BiGG database and \textsf{COBRA} \cite{bigg, cobra}. Note how the \token{name} attribute is overloaded with the chemical formula.
% %
% \exampleFile{examples/ex_spec_bigg.txt}

% \paragraph{An alternate \SBML Level 2 style annotation}
% A newer variation of the above, probably necessitated by the discontinuation of the \token{charge} attribute in \SBML and \textsf{libSBML}
% %
% \exampleFile{examples/ex_spec_cobra.txt}

% \paragraph{New \SBML Level 3 style species}
% Hopefully, with the adoption of \SBML \FBC these species properties can be unified into a common format.
% %
% \exampleFile{examples/ex_spec_l3.txt}


% \newpage
% \subsubsection*{Reaction definition and flux bounds}
% \paragraph{Old \SBML Level 2 style reaction}
% \exampleFile{examples/ex_reaction_bigg.txt}
% \newpage
% \paragraph{New \SBML Level 3 style reaction and flux bound}
% Please note that in order to maintain all the annotation encoded in the \SBML L2 reaction \token{notes} an additional (tool specific) annotation is introduced i.e.~`KeyValueData'. This should be considered a transitional step until a generally accepted annotation system is adopted by the constraint based modelling community. Nevertheless, where (unambiguously) possible, the L2 annotation has been converted into a MIRIAM compliant form e.g.~the `EC number'.
% \exampleFile{examples/ex_reaction_l3.txt}
% \exampleFile{examples/ex_fb_l3.txt}

