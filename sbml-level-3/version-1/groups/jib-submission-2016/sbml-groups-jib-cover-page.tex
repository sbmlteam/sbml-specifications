\documentclass{jib}
\newlength{\platz}
\setlength{\platz}{15pt}
\RequirePackage{listings}
\lstset{%
  basicstyle=\ttfamily,
  fontadjust,
  flexiblecolumns=true,
  frame=L,
  xleftmargin=15pt,
  framesep=5pt,
  emphstyle=\rmfamily\itshape}

\usepackage{pdfpages}

%%%%%%%%%%%%%%%%%%%%%%%%%%%%%%%%%%%%%%%%%%%%%%%%%%%%%%%%%%
% JIB Header/Footer
%%%%%%%%%%%%%%%%%%%%%%%%%%%%%%%%%%%%%%%%%%%%%%%%%%%%%%%%%%
\jibvolume{XX} % insert volume
\jibissue{X}   % insert issue
\jibpages{XXX} % insert article ID
\jibyear{XXXX} % insert year
\makeHeaderFooter{} % leave as is
%%%%%%%%%%%%%%%%%%%%%%%%%%%%%%%%%%%%%%%%%%%%%%%%%%%%%%%%%%

\begin{document}

%%%%%%%%%%%%%%%%%%%%%%%%%%%%%%%%%%%%%%%%%%%%%%%%%%%%%%%%%%
%
% Title Page
%
%%%%%%%%%%%%%%%%%%%%%%%%%%%%%%%%%%%%%%%%%%%%%%%%%%%%%%%%%%

\begin{jibtitlepage}

\jibtitle{SBML Level 3 package: Groups, Version 1 Release 1}

\jibauthor{%
  Michael Hucka\iref{caltech},
  Lucian P. Smith\iref{uw}
}

\addjibinstitution{caltech}{Department of Computing and Mathematical Sciences,\\California Institute of Technology, Pasadena, CA, USA}
\addjibinstitution{uw}{Department of Bioengineering,\\University of Washington, Seattle, WA, USA}

\end{jibtitlepage}

% The abstract
\begin{abstract}
Biological models often contain components that have relationships with each other, or that modelers want to treat as belonging to groups with common characteristics or shared metadata.  The \emph{SBML Level~3 Version~1 Core} specification does not provide an explicit mechanism for expressing such relationships, but it does provide a mechanism for SBML \emph{packages} to extend the Core specification and add additional syntactical constructs.  The SBML \emph{Groups} package for SBML Level~3 adds the necessary features to SBML to allow grouping of model components to be expressed.  Such groups do not affect the mathematical interpretation of a model, but they do provide a way to add information that can be useful for modelers and software tools.  The SBML Groups package enables a modeler to include definitions of groups and nested groups, each of which may be annotated to convey why that group was created, and what it represents.
\end{abstract}

% Include your PDF document
%\includepdf[pages=-]{../spec/sbml-level-3-version-1-core.pdf}

\end{document}
