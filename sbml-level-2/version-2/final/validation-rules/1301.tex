If a \Compartment definition has a \token{spatialDimensions} value of
\val{0}, then its \token{units} must be set to \val{dimensionless} or else
to the identifier of a \UnitDefinition based on \token{dimensionless}.
(References: L2V1 Section 4.5.4; Section~\ref{sec:compartment-units}.)
