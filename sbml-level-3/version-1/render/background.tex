% -*- TeX-master: "main"; fill-column: 72 -*-

\section{Background}
\label{background}

In 2003 the authors proposed an extension to the SBML file format that allowed programs 
to include layout and render information in SBML files to store one or more 
graphical representations of the SBML model. During the discussions on the
SBML mailing list, it soon became evident that a consensus for both layout and render 
information would not be reached easily, therefore the layout specification was 
separated from the render part of the specification and concentrated on the inclusion of 
layout information into SBML files. The Layout Specification has since been publicly 
accepted as SBML Level~3 Package. 

This document describes now an extension to the SBML Layout Package that describes 
the precise rendering of elements. Where the Layout package only describes the size and 
location if objects, the \RenderPackage complements this description by detailing 
precisely how they are to be rendered. 

\subsection{Design decisions}

The render extension is based on the existing layout extension. Secondly, the render extension 
is made as flexible as possible in order to not impose any artificial limits on how 
programs can display their reaction networks.

The render extension is independent of the underlying SBML model as well as of the layout extension, 
thus the render information will be stored as one or more separate blocks. There can be one 
block of render information that applies to all layouts and an additional block for each 
layout. For SBML Level~2 this render information will be stored in the annotation of the 
\texttt{listOfLayouts} element or the annotation of a \texttt{layout} element 
respectively.

The render information consists of a set of styles that are associated with 
objects from the Layout Package either by a list of \token{id}s of layout objects or 
by roles of layout objects or \token{id}s of their corresponding model elements. For example, 
a style can be defined that applies to all \SpeciesReference 
objects or to all objects that have the \token{role} \texttt{product}. 

Global render information included in the annotation of 
the \texttt{listOf\-Layouts} element will only be able to define styles that 
associate render information with roles of elements, it cannot associate 
styles with individual objects from a layout via their \token{id}s.

Many of the elements used in the current render specification are based on 
corresponding elements from the SVG specification. This allows us to easily convert a combination 
of layout information and render information into a SVG drawing. At the same time we profit 
from the work that has already been done while creating the SVG specification.
