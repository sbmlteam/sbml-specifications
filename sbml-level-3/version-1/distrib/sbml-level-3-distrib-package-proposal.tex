\documentclass[draftspec]{sbmlpkgspec}
\usepackage{tabularx}
\usepackage{tabu}
\usepackage{longtable}
\usepackage{booktabs}
\usepackage{microtype}

%macros:
\newcommand{\fixttspace}{\hspace*{1pt}}

\newcommand{\sbmlthreecore}{SBML Level~3 Version~1 Core\xspace}
\newcommand{\sbmlthreedistrib}{SBML Level~3 Package Specification for Distributions, Version~1\xspace}
\newcommand{\sbmlverone}{SBML Level 3 Version 1\xspace}

\newcommand{\DrawFromDistribution}{\defRef{DrawFromDistribution}{drawFromDistribution-class}}
%\newcommand{\ExplicitPDF}{\defRef{ExplicitPDF}{explicitPDF-class}}
\newcommand{\ListOfDistribInputs}{\defRef{ListOfDistribInputs}{drawFromDistribution-class}}
\newcommand{\DistribInput}{\defRef{DistribInput}{distribInput-class}}
\newcommand{\Distribution}{\defRef{Distribution}{distributionOrSample-class}}
\newcommand{\AbstractUncertainty}{\defRef{AbstractUncertainty}{abstractuncertainty-class}}
%\newcommand{\UncertMath}{\defRef{Math}{uncertmath-class}}
\newcommand{\Uncertainty}{\defRef{Uncertainty}{uncertainty-class}}

\newcommand{\ExplicitPMF}{\textbf{\class{ExplicitPMF}}\xspace}
\newcommand{\FluxBound}{\textbf{\class{FluxBound}}\xspace}
\newcommand{\FunctionTerm}{\textbf{\class{FunctionTerm}}\xspace}
\newcommand{\LambdaClass}{\textbf{\class{Lambda}}\xspace}
\newcommand{\ChangedMath}{\textbf{\class{ChangedMath}}\xspace}
\newcommand{\Math}{\textbf{\class{Math}}\xspace}

\newcommand{\arraysshort}{arrays\xspace}
\newcommand{\arrays}{Arrays\xspace}
\newcommand{\distribshort}{\emph{distrib}\xspace}
\newcommand{\distrib}{Distributions\xspace}
\newcommand{\mathml}{MathML\xspace}
\newcommand{\req}{Required Elements\xspace}
\newcommand{\uncertml}{UncertML\xspace}

\reversemarginpar  % Want "\watchout" to be put on the left, not the right.
\newcommand{\watchout}{\marginpar{\hspace*{34pt}\raisebox{-0.5ex}{\Large\ding{43}}}}
\newcommand{\controversial}{\marginpar{\hspace*{34pt}\raisebox{-0.5ex}{\Large?}}}

\begin{document}

\packageTitle{The Distributions Package\\for SBML Level 3}
\packageVersion{Version \changed{0.16} (Draft)}
\packageVersionDate{April, 2015}
%\packageGeneralURL{http://sbml.org/Community/Wiki/SBML_Level_3_Proposals/Distributions_and_Ranges}
%\packageThisVersionURL{}

\author{%
  \begin{tabular}{c>{\hspace{20pt}}c}
    \multicolumn{2}{c}{\Large\bf{Authors}}\\\\
    Stuart L Moodie                     & Lucian P Smith\\
    \mailto{moodie@ebi.ac.uk}           & \mailto{lpsmith@uw.edu}\\
    EMBL-EBI                            & California Institute of Technology\\
    Hinxton, UK                         & Seattle, WA, USA\\
  \\
    \multicolumn{2}{c}{\Large\bf{Contributors}}\\\\
    Nicolas Le Nov\`{e}re               & Darren Wilkinson\\
    \mailto{lenov@babraham.ac.uk}       & \mailto{darren.wilkinson@ncl.ac.uk}\\
    Babraham Institute                  & University of Newcastle\\
    Babraham, UK                        & Newcastle, UK\\
  \\
    Maciej Swat                         & Sarah Keating\\
    \mailto{mjswat@ebi.ac.uk}           & \mailto{skeating@ebi.ac.uk}\\
    EMBL-EBI                            & EMBL-EBI\\
    Hinxton, UK                         & Hinxton, UK\\
\\
     \multicolumn{2}{c}{ Colin Gillespie}\\
     \multicolumn{2}{c}{\mailto{c.gillespie@ncl.ac.uk}}\\
     \multicolumn{2}{c}{University of Newcastle}\\
     \multicolumn{2}{c}{ Newcastle, UK}\\
\end{tabular}
}

\frontNotice{Disclaimer: This is a working draft of the SBML Level 3
  ``distib'' package proposal. It is not a normative document.  Please
  send comments and other feedback to the mailing list:
  \mailto{sbml-distrib@lists.sourceforge.net.}}

\maketitlepage
\maketableofcontents

\section*{Revision History}

\begin{edtable}{tabularx}{\linewidth}{c c c X }\toprule
\textbf{Version} & \textbf{Date} & \textbf{Author} & \textbf{Comments} \\ \midrule
0.1 (Draft) & 15 Oct 2011 & Stuart Moodie & First draft \\ \midrule
0.2 (Draft) & 16 Oct 2011 & Stuart Moodie & Added introductory text
and background info. Other minor changes etc. \\ \midrule
0.3 (Draft) & 16 Oct 2011 & Stuart Moodie & Filled empty invocation
semantics section.\\ \midrule
0.4 (Draft) & 4 Jan 2012 & Stuart Moodie & Incorporated comments from
NlN, MS and SK. Some minor revisions and corrections.\\  \midrule
0.5 (Draft) & 6 Jan 2012 & Stuart Moodie & Incorporated addition
comments on aim of package from NlN.\\ \midrule
0.6 (Draft) & 19 Jul 2012 & Stuart Moodie & Incorporated revisions
discussed and agreed at HARMONY 2012.\\ \midrule
0.7 (Draft) & 6 Aug 2012 & Stuart Moodie & Incorporated review
comments from Maciej Swat and Sarah Keating.\\ \midrule
0.8 (Draft) & 21 Dec 2012 & Stuart Moodie & Incorporated changes
suggested at combine and subsequently through list discussions.\\ \midrule
0.9 (Draft) & 9 Jan 2013 & Stuart Moodie & Incorporated corrections
and comments from Maciej Swat and Sarah Keating.\\ \midrule
0.10 (Draft) & 10 Jan 2013 & Stuart Moodie & Modified based on comments
from MS.\\ \midrule
0.11 (Draft) & 17 May 2013 & Lucian Smith & Modified based on Stuart's proposals and PWG discussion.\\ \midrule
0.12 (Draft) & June 2013 & Lucian Smith and Stuart Moodie & Modified based on HARMONY 2013 discussion.\\ \midrule
0.13 (Draft) & July 2013 & Lucian Smith and Stuart Moodie & Modified based PWG discussion, particularly with respect to UncertML.\\ \midrule
0.14 (Draft) & March 2015 & Lucian Smith  & Modified to match UncertML 3.0.\\\midrule
0.15 (Draft) & March 2015 & Lucian Smith and Sarah Keating & Modified to match UncertML 3.0 for real this time.\\ \midrule
0.16 (Draft) & March 2015 & Lucian Smith & Added information about UncertML 3.0 distributions, and the distributions custom annotations.\\
%\midrule
\bottomrule
\end{edtable}

\section{Introduction and motivation}

\subsection{What is it?}

The \distrib package (also affectionately known as \distribshort for
short) provides an extension to SBML Level 3 that enables a model to encode and sample from
both discrete and continuous probability distributions, and provide
the ability to annotate elements with information about the distribution their
values were drawn from. 
Applications of the package include for instance descriptions of
population based models: an important subset of which are
pharmacokinetic/pharmacodynamic (PK/PD) models\footnote{for more
  information see: \url{http://www.pharmpk.com/}.}, which are used to
model the action of drugs.

Note that originally the package was called Distributions and Ranges,
but Ranges and the use of probability distributions to describe
statistical uncertainly are no longer in the scope, hence the name change.

\subsection{Scope}

The \distrib package adds support to SBML for sampling from a
probability distribution. In particular the following are in scope:

\begin{itemize}
\item Sampling from a continuous distribution.
\item Sampling from a discrete distribution.
\item Sampling from user-defined discrete probability density function.
\item The specification of descriptive statistics (mean, standard
  deviation, standard error, etc.).
\end{itemize}

At one point the following were considered for inclusion in this
package but are now \textbf{out of scope}:

\begin{itemize}
\item Sampling from user-defined probability density function.
\item Stochastic differential equations.
\item Other functions used to characterise a probability distribution,
  such as cumulative distribution functions (CDF) or survival functions, etc.
\end{itemize}

\subsection{This Document}

This proposal describes the consensus view of workshop participants
and subscribers to the sbml-distrib mailing list. Although it was
written by the listed authors it does not soley reflect their views nor is
it their proposal. Rather, it is their understanding of the consensus
view of what the \distrib package should do and how it should do
it. The contributors listed have made significant contributions to the
development and writing of this specification and are credited
accordingly, but a more comprehensive attribution is provided in the
acknowledgements (\sec{sec:acknowledgements}).

Finally, the authors would encourage the
reader to consider them and contribute their ideas or comments ---
indeed any feedback about this proposal --- to the \distribshort
discussion list\footnote{\mailto{sbml-distrib@lists.sourceforge.net}}.

Once the proposal is finalised this will be the first step towards the
formal adoption of the \distribshort as a package in SBML Level
3. After this, two implementations based on this proposal are required
and then the SBML editors must agree that the implementations and specification are complete. The proposal
will then provide the basis for a future package specification
document. More details of the SBML package adoption process can be
found at: \url{http://sbml.org/Documents/SBML_Development_Process}.


\subsection{Conventions used in this document}

As we are early in the package proposal process there will be some
parts of this proposal where there is no clear consensus on the
correct solution or only recent agreement or agreement by a group
which may not be representative of the SBML community as a
whole. These cases are indicated by the \controversial question mark
in the left margin (illustrated). The reader should pay particular
attention to these points and ideally provide feedback, especially if
they disagree with what is proposed. Similarly there will be points
--- especially as the proposal is consolidated --- which are agreed,
but which the reader should take note of and perhaps read again. These
points \watchout are emphasised by the hand pointer in the left margin
(illustrated).

\section{Background}

\subsection{Problems with current SBML approaches}

SBML Level 3 Core has no direct support for encoding random values
within a model. Currently there is no workaround within the core
language itself, although it is possible to define such information
using annotations within SBML itself. Frank Bergmann had proposed such
an semi-formalised extension for use with SBML L2 [REF?].

\subsection{Past work on this problem or similar topics}

\subsubsection{The Newcastle Proposal}
\label{sec:newcastle proposal}

In 2005 there was a proposal from Colin Gillespie and others
\footnote{\url{http://sbml.org/Community/Wiki/SBML\_Leve\l_3\_Proposals/Distributions\_and\_Ranges}}
to introduce support for probability distributions in the SBML core specification. This
was based on their need to use such distributions to represent the
models they were creating as part of the BASIS project
(\url{http://www.basis.ncl.ac.uk}).

They proposed that distributions could be referred to in SBML using
the \class{csymbol} element in the \mathml subset used by
the SBML Core specification. An example is below:

\begin{example}
<xmlns=''http://www.w3.org/1998/Math/MathML''>
  <apply>
    <csymbol encoding=''text''
        definitionURL=''http://www.sbml.org/sbml/symbols/uniformRandom''>
      uniformRandom
    </csymbol>
    <ci>mu</ci>
    <ci>sigma</ci>
  </apply>
</math>
\end{example}

This required that a library of definitions be maintained as part of
the SBML standard and in their proposal they defined an initial small
set of commonly used distributions. The proposal was never
implemented.

\subsubsection{Seattle 2010}

The ``distrib'' package was discussed at the Seattle SBML Hackathon%
\footnote{\url{http://sbml.org/Events/Hackathons/The_2010_SBML-BioModels.net_Hackathon}}
and this section is an almost verbatim reproduction of Darren
Wilkinson's report on the
meeting\footnote{\url{http://sbml.org/Forums/index.php?t=tree\&goto=6141\&rid=0}}. There
Darren presented an overview of the problem%
\footnote{Slides: \url{http://sbml.org/images/3/3b/Djw-sbml-hackathon-2010-05-04.pdf}}%
\footnote{Audio: \url{http://sbml.org/images/6/67/Wilkinson-distributions-2010-05-04.mov}},
building on the old proposal from the Newcastle group (see above:
\ref{sec:newcastle proposal}).  There was broad support at the meeting
for development of such a package, and for the proposed feature
set. Discussion following the presentation led to a consensus on the
following points:

\begin{itemize}
\item There is an urgent need for such a package.
\item It is important to make a distinction between a description of
  uncertainty regarding a model parameter and the mechanistic process
  of selecting a random number from a probability distribution, for
  applications such as parameter scans and experimental design
\item It is probably worth including the definition of PMFs, PDFs and CDFs in the package
\item It is worth including the definition of random distributions using particle representations within such a package, though some work
 still needs to be done on the precise representation
\item It could be worth exploring the use of xinclude to point at particle
representations held in a separate file
\item Random numbers must not be used in rate laws or anywhere else that
 is continuously evaluated, as then simulation behaviour is not
 defined
\item Although there is a need for a package for describing extrinsic
 noise via stochastic differential equations in SBML, such mechanisms
 should not be included in this package due to the considerable
 implications for simulator developers
\item We probably don't want to layer on top of \uncertml
 (www.uncertml.org), as this spec is fairly heavy-weight, and
 somewhat tangential to our requirements
\item A random number seed is not part of a model and should not be
 included in the package
\item The definition of truncated distributions and the specification of
 hard upper and lower bounds on random quantities should be
 considered.
\end{itemize}

It was suggested that new constructs should be introduced into SBML by
the package embedded as user-defined functions using the following
syntax:

\begin{example}
<listOfFunctionDefinitions>
  <functionDefinition id="myNormRand">
    <distrib:####>
      #### distrib binding information here ####
    </distrib:####>
    <math>
      <lambda>
        <bvar>
          <ci>mu</ci>
          <ci>sigma</ci>
        </bvar>
        <ci>mu</ci>
      </lambda>
    </math>
  </functionDefinition>
</listOfFunctionDefinitions>
\end{example}

which allows the use of a "default value" by simulators which do not
understand the package (but simulators which do will ignore the <math>
element). The package would nevertheless be "required", as it will not
be simulated correctly by software which does not understand the
package.

Informal discussions following the break-out covered topics such as:

\begin{itemize}
\item how to work with vector random quantities in the absence of the vector
element in the MathML subset used by SBML
\item how care must be taken with the semantics of random variables
  and the need to both:
\begin{itemize}
\item reference multiple independent random quantities at a given
  time
\item make multiple references to the same random quantity at a given
time.
\end{itemize}
\end{itemize}

\subsubsection{Hinxton 2011}

Detailed discussion was continued at the Statistical Models Workshop
in Hinxton in June 2011%
\footnote{\url{http://sbml.org/Events/Other_Events/statistical_models_workshop_2011}}. There
those interested in representing Statistical Models in SBML came
together to work out the details of how this package would work in
detail. Dan Cornford from the \uncertml
project\footnote{\url{http://www.uncertml.org/}} attended the meeting
and described how that resource could be used to describe uncertainty
and in particular probability distributions. Perhaps the most
significant decision at this meeting was to adopt the \uncertml
resource as a controlled vocabulary that is referenced by the \distrib package.

Much has changed since this meeting, but the output from this meeting
was the basis for the first version of this proposal.


\subsubsection{HARMONY 2012: Maastricht}

Two sessions were dedicated to discussion of \distrib at HARMONY based
around the proposals described in version 0.5 of this document. In
addition there was discussion about the \arrays proposal which was
very helpful in solving the problem of multivariate distributions in
\distrib. The following were the agreed outcomes of the meeting:

\begin{itemize}
\item The original proposal included UncertML markup directly in the
  function definition. This proved unwieldy and confusing and has been
  replaced by a more elegant solution that eliminates the UncertML
  markup and integrates well with the fallback function (see details
  below).
\item Multivariate distributions can be supported using the \arrays
  package to define a covariance matrix.
\item User defined continuous distributions would define a PDF in
  \mathml.
\item Usage semantics were clarified so that invokation of a function
  definition implied a value was sampled from the specified
  distribution.
\item It was agreed from which sections of an SBML model a
  distribution could be invoked.
\item Statistical descriptors of variables (for
  example mean and standard deviation) would be separated from
  \distrib and either provided in a new package or in a later version
  of SBML L3 core.
\end{itemize}

\subsubsection{COMBINE 2012: Toronto}

The August proposal was reviewed and an improvement was agreed to
the user-defined PMF part of the proposal. In particular is was agreed
that the categories should be defined by \distribshort classes rather
than by passing in the information as an array. Questions were also raised
about whether \uncertml was suitably well defined to be used as an
external definition for probability distributions. This was resolved
subsequent to the meeting with a teleconference to Dan Cornford and
colleagues. These changes are incorporated here. Finally, there was
considerable debate about whether to keep the dependence of
\distribshort on the Arrays package in order to support multi-variate
distributions. The outcome was an agreement that we would review this
at the end of 2012, based on the results of an investigation
into how feasible it would be to implement \arrays as a package.

\subsubsection{2013 Package Working Group discussions}

Early 2013 saw a good amount of discussion on the \distribshort Package Working Group mailing list, spurred by proposals by Stuart Moodie\footnote{\url{http://thestupott.wordpress.com/2013/03/12/an-improved-distrib-proposal/}}.  While not all of his suggestions ended up being fully accepted by the group, several changes were accepted, including:

\begin{itemize}
\item To use UncertML as actual XML, instead of as a set of reference definitions.
\item To use UncertML to encode descriptive statistics of SBML elements such as mean, standard deviation, standard error, etc.) bringing this capability back in scope for this package.
\end{itemize}


\subsubsection{HARMONY 2013: Connecticut}

At HARMONY at UConn in Connecticut, further discussions revealed the importance of distinguishing the ability to describe an element as a distributed variable vs. a function call within the model performing a draw from a distribution.

We also decided to discard the encoding of explicit PDFs for now, as
support for it is remarkably complicated, and there no demand for
it. The current design could be extended to support this feature so if
there is demand for it in the future support for explicit PDFs could
be reintroduced.

\section{Proposed syntax and semantics}

\subsection{Overview}

Following the precedent set by the SBML Level~3 Core specification
document, we use UML~1.0 (Unified Modeling Language;
\citealt{eriksson:1998,oestereich:1999}) class diagram notation to
define the constructs provided by this package.  We also use color in
the diagrams to carry additional information for the benefit of those
viewing the document on media that can display color.  The following are
the colors we use and what they represent:

\begin{itemize}

\item[\raisebox{2.75pt}{\colorbox{black}{\rule{0.8pt}{0.8pt}}}]
  \emph{Black}: Items colored black in the UML diagrams are components
  taken unchanged from their definition in the SBML Level~3 Core
  specification document.

\item[\raisebox{2.75pt}{\colorbox{mediumgreen}{\rule{0.8pt}{0.8pt}}}]
  \emph{\textcolor{mediumgreen}{Green}}: Items colored green are
  components that exist in SBML Level~3 Core, but are extended by this
  package.  Class boxes are also drawn with dashed lines to further
  distinguish them.

\item[\raisebox{2.75pt}{\colorbox{darkblue}{\rule{0.8pt}{0.8pt}}}]
  \emph{\textcolor{darkblue}{Blue}}: Items colored blue are new
  components introduced in this package specification.  They have no
  equivalent in the SBML Level~3 Core specification.

\end{itemize}

We also use the following typographical conventions to distinguish the
names of objects and data types from other entities; these conventions
are identical to the conventions used in the SBML Level~3 Core specification
document:

\begin{description}
  
\item \abstractclass{AbstractClass}: Abstract classes are never
  instantiated directly, but rather serve as parents of other classes.
  Their names begin with a capital letter and they are printed in a
  slanted, bold, sans-serif typeface.  In electronic document formats,
  the class names defined within this document are also hyperlinked to
  their definitions; clicking on these items will, given appropriate
  software, switch the view to the section in this document containing
  the definition of that class.  (However, for classes that are
  unchanged from their definitions in SBML Level~3 Core, the class names
  are not hyperlinked because they are not defined within this
  document.)
  
\item \class{Class}: Names of ordinary (concrete) classes begin with a
  capital letter and are printed in an upright, bold, sans-serif
  typeface.  In electronic document formats, the class names are also
  hyperlinked to their definitions in this specification document.
  (However, as in the previous case, class names are not hyperlinked if
  they are for classes that are unchanged from their definitions in the
  SBML Level~3 Core specification.)

\item \token{SomeThing}, \token{otherThing}: Attributes of classes, data
  type names, literal XML, and tokens \emph{other} than SBML class
  names, are printed in an upright typewriter typeface.  Primitive types
  defined by SBML begin with a capital letter; SBML also makes use of
  primitive types defined by XML
  Schema~1.0~\citep{biron:2000,fallside:2000,thompson:2000}, but
  unfortunately, XML~Schema does not follow any capitalization
  convention and primitive types drawn from the XML~Schema language may
  or may not start with a capital letter.

\item \token{[elementName]}:  In some cases, an element may contain a child of any class inheriting from an abstract base class.  In this case, the name of the element is indicated by giving the abstract base class name in brackets, meaning that the actual name of the element depends on whichever subclass is used, with capitalization following the capitalization of the name in brackets.

\end{description}

For other matters involving the use of UML and XML, we follow the
conventions used in the SBML Level~3 Core specification document.  


\subsection{Namespace URI and other declarations necessary for using this package}
\label{xml-namespace}

Every SBML Level~3 package is identified uniquely by an XML namespace URI.  For an SBML document to be able to use a given Level~3 package, it must declare the use of that package by referencing its URI.  The following is the namespace URI for this version of the \distrib package for \sbmlthreecore:
\begin{center}
\uri{http://www.sbml.org/sbml/level3/version1/distrib/version1}
\end{center}

In addition, SBML documents using a given package must indicate whether the package may be used to change the mathematical meaning of \sbmlthreecore elements.  This is done using the attribute \token{required} on the \token{<sbml>} element in the SBML document.  For the \distrib package, the value of this attribute must be \val{true}, as the \DrawFromDistribution element overrides the core definition of a \FunctionDefinition.  Note that the value of this attribute must \emph{always} be set to \val{true}, even if the particular model does not contain any \DrawFromDistribution elements.

The following fragment illustrates the beginning of a typical SBML model using \sbmlthreecore and this version of the \distrib package:

\begin{example}
<?xml version="1.0" encoding="UTF-8"?>
<sbml xmlns="http://www.sbml.org/sbml/level3/version1/core" level="3" version="1"
      xmlns:distrib="http://www.sbml.org/sbml/level3/version1/distrib/version1"
      distrib:required="true">
\end{example}


\subsection{Primitive data types}
\label{new-primitive-types}

The \distrib package uses the \val{string} primitive data type described in Section~3.1 of the \sbmlthreecore specification, and adds two additional primitive types described below.


\subsubsection{Type \fixttspace\primtypeNC{UncertId}}
\label{primtype-portid}

The type \primtype{UncertId} is derived from \primtype{SId} (\sbmlthreecore specification Section~3.1.7) and has identical syntax.  The \primtype{UncertId} type is used to create local IDs that can be used in the extended \FunctionDefinition objects to refer to the arguments of the function, in much the same way that the identities of the \token{bvar} elements are used in MathML \token{lambda} elements.  Each \primtype{UncertId} has a scope local to the \DrawFromDistribution in which it is found.  The
equality of \primtype{UncertId} values is determined by an exact
character sequence match; i.e., comparisons of these identifiers must be
performed in a case-sensitive manner.


\subsubsection{Type \fixttspace\primtypeNC{UncertIdRef}}
\label{primtype-portidref}

Type \primtype{UncertIdRef} is used for references to
identifiers of type \primtype{UncertId}.  This type is derived from
\primtype{UncertId}, but with the restriction that the value of an
attribute having type \primtype{UncertIdRef} must match the value of a
\primtype{UncertId} found in the same parent \DrawFromDistribution as the
\primtype{UncertIdRef}.  As with \primtype{UncertId}, the equality of
\primtype{UncertIdRef} values is determined by exact character sequence
match; i.e., comparisons of these identifiers must be performed in a
case-sensitive manner.



\subsection{Defining Distributions}

\subsubsection{The approach}

The \distrib package has two very simple purposes. First, it provides a
mechanism for sampling a random value from a probability
distribution. This implies that it must define the probability distribution and then must sample a
random value from that distribution.

Secondly, it provides a mechanism for describing elements with information about their uncertainty.  One common use case for this will be to provide the standard deviation for a value.  Another may be describing a parameter's distribution, so that a better search can be performed in a parameter scan experiment.

Both purposes are achieved by using \uncertml.  Probability density functions (PDFs) are defined in the \val{Distri\-butions} branch of \uncertml, probability mass functions (PMFs) are defined in the \val{Samples} branch, and summary statistics in the \val{Statistics} branch.

%The second way is to define
%a probability density function (PDF) using MathML.

It is technically possible to provide an explicit PDF in MathML instead of using the pre-defined PDFs from \uncertml.  However, one advantage of using the UncertML pre-defined
distributions is that software can easily recognise the distribution
and use an optimised built-in implementation rather than interpreting
the distribution from the PDF definitions. For some
applications such optimisations make important performance
differences.  Another advantage is that some software may only support certain types of distributions, and having them predefined makes it simpler for the software to inform a user that a particular distribution is not supported.

It is hoped that if users find the need to define distributions not covered by UncertML, that they will either be able to encode those distributions as combinations of other predefined distributions, or that they will be able to persuade \uncertml to add the new distribution to the list.

When a distribution is defined in a \FunctionDefinition, it is sampled when it is invoked. To reuse a
sampled value, the value must be assigned to a parameter first, such as through the use of an \InitialAssignment or \EventAssignment.  When a distribution is defined elsewhere, that information may be used outside of the model, using whatever methodology is appropriate to answer the question being pursued.


\subsection{The extended \class{FunctionDefinition} class}

To model random processes, this package extends the \FunctionDefinition class as
can be seen in the UML representation in \fig{fig:funcdef}. The redefined \FunctionDefinition optionally
contains a single \token{drawFromDistribution} child.

The \FunctionDefinition class must still contain the
\mathml block containing the standard SBML function definition, because that element is required in SBML Level~3 Version~1 (in SBML Level~3 Version~2, this requirement has been dropped).  This also
ensures a degree
of backwards compatibility for SBML readers and validators that do not
understand the \distribshort package.

\begin{figure}[htb]
\includegraphics[width=0.75\linewidth]{distrib-functionDefinition.pdf}
\caption{The definition of the extended \FunctionDefinition class, plus the \DrawFromDistribution, \ListOfDistribInputs, \DistribInput, and \Distribution classes.  A \DrawFromDistribution element must have exactly one \Distribution child.  Together, these classes provide a way to transform a \FunctionDefinition to sample from a distribution.}
\label{fig:funcdef}
\end{figure}


As outlined above, the \FunctionDefinition class is extended to contain a \DrawFromDistribution child.  Because a \FunctionDefinition must have a \LambdaClass child according to \sbmlthreecore, a valid function will still have one, but the \token{drawFromDistribution} child, if present, will override any definition found there.  However, software that does not support \distribshort could potentially invoke the function found in the \LambdaClass element (see \ref{sec:fallbackfunc}).

In the \Model, an extended \FunctionDefinition may be used in any \mathml to perform a draw from a distribution.  This draw will be unique for every use of the \FunctionDefinition, whether or not the draw is performed at the same simulation time as a different draw (for example, if used in two different \InitialAssignment elements).

%In the \Model, an extended \FunctionDefinition can have two different uses.  First, it may be used in any \mathml to perform a draw from a distribution.  This draw will be unique for every use of the \FunctionDefinition, whether or not the draw is performed at the same simulation time as a different draw (for example, if used in two different \InitialAssignment elements).

%Secondly, it may be used in an \Uncertainty element to denote that a particular element has the uncertainty defined by the extended function.  See \sec{uncertmath-class} for more details.



\subsection{The \class{DrawFromDistribution} class}
\label{drawFromDistribution-class}

As illustrated in \fig{fig:funcdef}, the \DrawFromDistribution class may have a \ListOfDistribInputs child, which must in turn contain one or more \DistribInput children, which act as the arguments to the function--they serve the same role as the \token{bvar} elements of the \LambdaClass child of a \FunctionDefinition.  The order of arguments is determined by the \token{index} attribute:  the first argument (if any) must have an index of \val{0}, the second of \val{1}, etc.

It must also have a \Distribution child, representing a probability density function (PDF) or probability mass function (PMF) defined by \uncertml.  Within the \uncertml, the \primtype{UncertId}s defined by the \DistribInput objects are used as the variables within the distribution.


\subsection{The \class{DistribInput} class}
\label{distribInput-class}

The \DistribInput class mimics the \token{bvar} elements of MathML lambda functions.  It must have an \token{id} attribute of type \primtype{UncertId} and an \token{index} attribute of type \primtype{non-negative integer}.  It may additionally have a \token{name} attribute of type \token{string}, which may be used in the same manner as other \token{name} attributes on \sbmlthreecore objects; please see Section~3.3.2 of the \sbmlthreecore specification for more information.

Each \DistribInput element represents an argument to the function, and serves as a local identifier, referenced only by the \uncertml in the sibling \Distribution class. See the examples in \ref{sec:fd-examples} for more details.

Because the \LambdaClass child of the \FunctionDefinition is required, it must have the same number of \token{bvar} children as the \DrawFromDistribution has \DistribInput children.  They do not, however, have to have the same IDs:  the \token{bvar} ids are defined as being local to the \LambdaClass function in much the same way that the \DistribInput IDs are defined as being local to the \DrawFromDistribution object.

Each \token{index} attribute on a \DistribInput within a \ListOfDistribInputs element must have a unique value, numbered consecutively from \val{0}:  if one \DistribInput is present, its \token{index} value must be \val{0}; if there are two, they must have \token{index} values of \val{0} and \val{1}, etc.

%The value \val{x} may not be used as an \primtype{UncertId} in an \DistribInput element; that symbol is reserved for use by the \ExplicitPDF element (see \ref{explicitPDF-class}).

\subsection{The \class{Distribution} class}
\label{distributionOrSample-class}

The \Distribution class is an \token{UncertML} element that contains a single element from the \token{Distributions} branch of UncertML.  There are 29 'Distribution' elements derived from this class defined at \url{http://uncertml.org/dictionary}, with the namespace \val{http://uncertml.org/3.0}.  Note that as of this writing, UncertML 3.0 is only defined with an XML schema.  However, the \distribshort Package Working Group has been in communication with the developers of UncertML, and feel certain that the change needed in UncertML to accomodate its use in this package (namely, the possible substitution of IDs for numbers) will be made for UncertML 3.0.  The actual name of this element within an SBML document will be identical to the name of the derived class itself, i.e. \val{NormalDistribution}, \val{LogisticDistribution}, etc.

When a \Distribution is encountered, its parent \FunctionDefinition is defined as sampling from the defined distribution, and returning that sample.  It may contain any number of \token{UncertIdRef} strings, each of which must correspond to an \token{UncertId} defined in a \DistribInput in the same function.

The full list of the 29 distributions and how they can be used is provided in \ref{sec:uncertml-distributions}.  Four of these distributions (Dirichlet, Multinomial, Multivariate Normal, and Multivariate Student T) use vectors as both input and output.  It is possible, if tedious, to provide vector input to these distributions by simply defining each element of the required vector as a numeric value or as an \primtype{UncertId}.  However, it is not possible in \sbmlthreecore to take a single vector and simultaneously assign its values to different elements, or even to use a vector within MathML.  While it would be theoretically possible to define new elements in this specification to work around this limitation, such capabilities are more obviously the domain of the Arrays package within SBML, or of SED-ML\footnote{\url{http://sed-ml.org/}} (to set a suite of element values).  Unfortunately, as of this writing, the Arrays package has not yet been finalized, and many aspects of it have not been set.  Therefore, it is left to the future finalized Arrays package to define how to utilize a \FunctionDefinition that returns a vector, and how to define a \FunctionDefinition that takes a vector as input.  It is also possible for other individuals or groups to come up with custom annotations that define how to do this, and in fact, this is encouraged for any group that requires the use of any of these four distributions for their models.  If no such definitions exist, however, any numerical results from the use of these functions remain undefined, and models using this technique are unsimulatable.  (Such models may still be useful descriptions of certain situations, however.)  A final possibility is that SED-ML could be extended to extract the function definition, perform the sampling itself, and use the resulting vector to assign initial values to certain elements.

%All three Sample elements in UncertML are logistically identical, and only semantically different.  The RandomSample describes a set of realizations known to come from a randomly-distributed source.  The SystematicSample describes a set of deliberately chosen realizations, such as those resulting from unscented sampling methods for Gaussian random variables.  The UnknownSample describes a set of realizations for which the source distribution is unknown.  All three describe a set of realizations, each with a weight and one or more values. The weights of all realizations in a single Sample must sum to 1.0.

%UncertML also allows the definition of realizations with \val{categories} instead of \val{values} (i.e. \primtype{strings} instead of \primtype{doubles}).  This data type is unsupported in\sbmlthreecore, but if a package supports this data type in the future (as the Qualitative Models package might), \val{categories} could be used.

%When a \Distribution is encountered containing a Sample, its parent \FunctionDefinition is defined as randomly choosing a single realization from the the \uncertml according to its \token{weight}, and, if the realization contains a \token{values} child, returning that value (or vector of values) when the function is called.  If the realization contains a \token{categories} child, that string (or vector of strings) is returned instead.  (\uncertml requires that a single realization contain either a \token{values} or \token{categories} child, but may not have both.)

%As happens with some of the UncertML distributions above, then, if the chosen realization has multiple values or categories, the result of the draw is a vector, which cannot, in \sbmlthreecore, be used either in MathML or as an assignment to an SBML element.  The Arrays package or some form of custom annotation must be used to define what happens when a multi-value realization is chosen.

%Similarly, there is no \token{string} data type for SBML elements defined in \sbmlthreecore.  Without a package that defines such a data type, then, any call to a \FunctionDefinition that returns a string or vector of strings from a \token{categories} \uncertml element will be undefined and therefore unsimulatable, but may, again, be a useful description of a particular type of model.

Note \notice that the \ExplicitPMF class in previous versions of this specification did not use \uncertml, and instead defined its own way of listing samples.  The functionality (with the addition of IDs instead of numbers in \uncertml 3.0) is replicated in the CategoricalDistribution class of \uncertml 3.0.


%\subsection{The \class{ExplicitPDF} class}
%\label{explicitPDF-class}

%The \ExplicitPDF class is a container for MathML XML.  It must define a function, using the same MathML allowed in the rest of the SBML Document, for the value \val{x} that integrates to 1.0 over all \token{x}, and which never evaluates to a negative value for any \token{x}.  It may also use the \primtype{UncertIDRef} identifiers defined by its sibling \DistribInput elements, and the \primtype{SId}s of any other \FunctionDefinition, but no other identifier.  For this reason, \val{x} is illegal to use as an \primtype{UncertId} in any \DistribInput element in the model:  it is reserved for use by the \ExplicitPDF class.  As is the case in \sbmlthreecore, circular function definitions are illegal: the \primtype{SId} of a \FunctionDefinition may not be used in its own child \ExplicitPDF, nor may it use the \primtype{SId} of a \FunctionDefinition that references itself, etc.


\subsection{Discrete vs. continuous sampling}
\label{discrete-continuous}

The \primtype{SId}s of \FunctionDefinition elements can be used in \sbmlthreecore in both discrete and continuous contexts:  \InitialAssignment, \EventAssignment, \Priority, and \Delay elements are all discrete, while \Rule, \KineticLaw, and \Trigger elements are all continuous in time.  For discrete contexts, the behavior of \distribshort-extended \FunctionDefinition elements is well-defined:  one or more random values are sampled from the distribution each time the function definition is invoked. Each invocation implies one sampling operation.  In continuous contexts, however, their behavior is ill-defined.  More information than is defined in this package (such as autocorrelation values or full conditional probabilities) would be required to make random sampling tractable in continuous contexts, and is beyond the scope of this version of the package.  If some package is defined in the future that adds this information, or if custom annotations are provided that add this information, such models may become simulatable.  However, this package does not define how to handle sampling in continuous contexts, and recommends against it: a warning may be produced by any software encountering the use of a \distribshort-extended \FunctionDefinition in a continuous context.  Assuming such models are desirable, and the information is not provided in a separate package, this information may be incorporated into a future version of this specification.

Any other package that defines new contexts for MathML will also either be discrete or continuous.  Discrete situations (such as those defined in the Qualitative Models package) are, as above, well-defined.  Continuous situations (as might arise within the Spatial Processes package, over space instead of over time) will most likely be ill-defined.  Those packages must therefore either define for themselves how to handle \distribshort-extended \FunctionDefinition elements, or leave it to some other package/annotation scheme to define how to handle the situation.


\subsection{Truncation}
\label{truncation}

In order to perform truncation, one or both ends of the distribution are cut off, and the remaining function is re-scaled so the area under the curve is once again 1.0.  This capability is provided in UncertML directly, through the use of the \val{truncationLowerInclusiveBound} and \val{truncationUpperInclusiveBound} elements.  As those element names imply, the first is used to truncate the distribution at a lower bound, and the second at an upper bound.  The fact that both are inclusive mean that the value at that boundary may possibly be returned by the function.  Care should thus be taken to ensure that (for example) if zero is a \emph{non-inclusive} bound for one's model, that a value slightly larger than zero be used as the distribution's inclusive lower bound.


\subsection{Examples using the extended \FunctionDefinition}
\label{sec:fd-examples}

Several examples are given below that illustrate various uses of an extended \FunctionDefinition.

\subsubsection{Defining and using a normal distribution with UncertML}
In the following example, a \FunctionDefinition is extended to define a draw from an UncertML-defined normal distribution:

\begin{example}
...
  <listOfFunctionDefinitions>
    <functionDefinition id="normal">
      <math xmlns="http://www.w3.org/1998/Math/MathML">
        <!-- Overridden MathML -->
      </math>
      <distrib:drawFromDistribution>
         <distrib:listOfDistribInputs>
           <distrib:distribInput distrib:id="avg" distrib:index="0"/>
           <distrib:distribInput distrib:id="sd" distrib:index="1"/>
         </distrib:listOfDistribInputs>
         <UncertML xmlns="http://www.uncertml.org/3.0">
           <NormalDistribution definition="http://www.uncertml.org/distributions">
             <mean>
               <var varId="avg"/>
             </mean>
             <stddev>
               <var varId="sd"/>
             </stddev>
           </NormalDistribution>
         </UncertML>
      </distrib:drawFromDistribution>
    </functionDefinition>
  </listOfFunctionDefinitions>
...
\end{example}

Here, the \DistribInput children of \DrawFromDistribution define the local \primtype{UncertId}s \val{avg} and \val{sd}, which are then used by the \Distribution as the \token{mean} and \token{stddev} of a normal distribution, as defined by UncertML.  This function could then be used anywhere the \FunctionDefinition id \val{normal} can be used, as for example in an \InitialAssignment:

\begin{example}
...
  <listOfInitialAssignments>
    <initialAssignment symbol="y">
      <math xmlns="http://www.w3.org/1998/Math/MathML">
        <apply>
          <ci> normal </ci>
          <ci> z </ci>
          <cn> 10 </cn>
        </apply>
      </math>
    </initialAssignment>
  </listOfInitialAssignments>
...
\end{example}

This use would apply a draw from a normal distribution with mean \val{z} and standard deviation \val{10} to the SBML element \val{y}.

\subsubsection{Defining a 'die roll' PMF with UncertML}
In the following example, a \FunctionDefinition is extended to define a draw from an UncertML-defined set of explicit PMFs:

\clearpage
\begin{example}
...
  <listOfFunctionDefinitions>
    <functionDefinition id="rolld4">
      <math xmlns="http://www.w3.org/1998/Math/MathML">
        <!-- Overridden MathML -->
      </math>
        <distrib:drawFromDistribution>
          <UncertML xmlns="http://www.uncertml.org/3.0">
            <CategoricalDistribution definition="http://www.uncertml.org/distributions">
              <categoryProb>
                <name>1</name>
                <prob>
                  <pVal>0.25</pVal>
                </prob>
              </categoryProb>
              <categoryProb>
                <name>2</name>
                <prob>
                  <pVal>0.25</pVal>
                </prob>
              </categoryProb>
              <categoryProb>
                <name>3</name>
                <prob>
                  <pVal>0.25</pVal>
                </prob>
              </categoryProb>
              <categoryProb>
                <name>4</name>
                <prob>
                  <pVal>0.25</pVal>
                </prob>
              </categoryProb>
            </CategoricalDistribution>
          </UncertML>
        </distrib:drawFromDistribution>
    </functionDefinition>
  </listOfFunctionDefinitions>
...
\end{example}

No inputs are provided.  The four \token{categoryProb} children of the \token{CategoricalDistribution} all have equal values for their \token{prob} children, and sum to 1.0, as they must.  Each \token{name} is therefore equally likely to be chosen, resulting in this function returning \val{1}, \val{2}, \val{3}, or \val{4}, each with equal probability.



\subsubsection{Defining a 'pick one' sample with UncertML}
In the following example, a \FunctionDefinition is extended to define a draw from an UncertML-defined set of samples:

\begin{example}
...
<?xml version="1.0" encoding="UTF-8"?>
<sbml xmlns="http://www.sbml.org/sbml/level3/version1/core"
      xmlns:distrib="http://www.sbml.org/sbml/level3/version1/distrib/version1"
      level="3" version="1" distrib:required="true">
  <model>
    <listOfFunctionDefinitions>
      <functionDefinition id="pickone">
        <math xmlns="http://www.w3.org/1998/Math/MathML">
          <lambda>
            <bvar>
              <ci> A </ci>
            </bvar>
            <bvar>
              <ci> B </ci>
            </bvar>
            <bvar>
              <ci> C </ci>
            </bvar>
            <bvar>
              <ci> D </ci>
            </bvar>
            <notanumber/>
          </lambda>
        </math>
        <distrib:drawFromDistribution>
          <distrib:listOfDistribInputs>
            <distrib:distribInput distrib:id="A" distrib:index="0"/>
            <distrib:distribInput distrib:id="B" distrib:index="1"/>
            <distrib:distribInput distrib:id="C" distrib:index="2"/>
            <distrib:distribInput distrib:id="D" distrib:index="3"/>
          </distrib:listOfDistribInputs>
          <UncertML xmlns="http://www.uncertml.org/3.0">
            <CategoricalDistribution definition="http://www.uncertml.org/distributions">
              <categoryProb>
                <name>A</name>
                <prob>
                  <pVal>0.25</pVal>
                </prob>
              </categoryProb>
              <categoryProb>
                <name>B</name>
                <prob>
                  <pVal>0.25</pVal>
                </prob>
              </categoryProb>
              <categoryProb>
                <name>C</name>
                <prob>
                  <pVal>0.25</pVal>
                </prob>
              </categoryProb>
              <categoryProb>
                <name>D</name>
                <prob>
                  <pVal>0.25</pVal>
                </prob>
              </categoryProb>
            </CategoricalDistribution>
          </UncertML>
        </distrib:drawFromDistribution>
      </functionDefinition>
    </listOfFunctionDefinitions>
    <listOfParameters>
      <parameter id="y"/>
    </listOfParameters>
    <listOfInitialAssignments>
      <initialAssignment symbol="y">
        <math xmlns="http://www.w3.org/1998/Math/MathML">
          <apply>
            <ci> pickone </ci>
            <cn type="integer"> 1 </cn>
            <cn type="integer"> 3 </cn>
            <cn type="integer"> 4 </cn>
            <cn type="integer"> 7 </cn>
          </apply>
        </math>
      </initialAssignment>
    </listOfInitialAssignments>
  </model>
</sbml>
...
\end{example}

In this example, the function 'pickone' is defined, with four arguments, \val{A}, \val{B}, \val{C}, and \val{D}.  When called, each argument has an equal chance of being chosen as the return value.  The parameter 'x' is initialized by calling this function with the arguments 1, 3, 4, and 7, each of which has an equally likely chance of being chosen.

{\color{red} Stuart: \controversial This seems a bit idiosyncratic to me. Are you trying to get
  round the fact that SBML doesn't support strings? If so I'd leave
  that up to the modeller. They can map numeric values to their
  categories.  }

{\color{red} Lucian: \controversial It may be idiosyncratic; I was trying to illustrate a case where variables were used in the values instead of the probabilities.  Is there a better example you can think of?  I've added an initial assignment that uses the function by way of illustration.}

\subsection{Equivalence with Fallback Function}
\label{sec:fallbackfunc}

The \mathml definition directly contained by the
\class{functionDefinition} is not used, but is required by \sbmlthreecore.  To ensure the continued validity of the model, the following rules must be followed:

\begin{itemize}
\item the lambda function should have the same number of arguments as
  its equivalent distribution (defined by \distribshort).
\item Each argument should match the type of the equivalent argument
  in the external function.
\item The lambda function should have the same return type as the
  \emph{sampled} distribution. For example, if a predefined PDF when
  sampled returns a scalar value, the dummy function should also do so.
\end{itemize}

Clearly, these rules can only be enforced by a \distribshort-aware validator.

In the following example, the fallback function is coded to simply return \val{mean}, the first argument of the function.  Note that the arguments have been given different local IDs (\val{mean} and \val{s} instead of \val{avg} and \val{sd}); their equivalence is based on order, not string matching.

\begin{example}
      <functionDefinition id="normal">
        <math xmlns="http://www.w3.org/1998/Math/MathML">
          <lambda>
            <bvar>
              <ci> mean </ci>
            </bvar>
            <bvar>
              <ci> s </ci>
            </bvar>
            <ci> mean </ci>
          </lambda>
        </math>
        <distrib:drawFromDistribution>
          <distrib:listOfDistribInputs>
            <distrib:distribInput distrib:id="avg" distrib:index="0"/>
            <distrib:distribInput distrib:id="sd" distrib:index="1"/>
          </distrib:listOfDistribInputs>
          <UncertML xmlns="http://www.uncertml.org/3.0">
            <NormalDistribution definition="http://www.uncertml.org/distributions">
              <mean>
                <var varId="avg"/>
              </mean>
              <stddev>
                <var varId="sd"/>
              </stddev>
            </NormalDistribution>
          </UncertML>
        </distrib:drawFromDistribution>
      </functionDefinition>
\end{example}

\begin{figure}[bh]
  \includegraphics{extended-sbase}
  \caption{The definition of the extended \SBase class to include a new optional \Uncertainty child, which in turn has an \AbstractUncertainty child in the \uncertml namespace.  Intended for use with any element with mathematical meaning, or with a \Math child.}
  \label{extended-sbase-uml}
\end{figure}

\subsection{The extended \SBase class}
\label{sec:extended-sbase-class}

As can be seen in \fig{extended-sbase-uml}, the SBML base class \SBase is extended to include an optional \Uncertainty child, which must contain an \AbstractUncertainty child containing what information about the uncertainty of its parent element.  In \sbmlthreecore, one should only extend those \SBase elements with mathematical meaning (so, \Compartment, \Parameter, \Reaction, \Species, and \SpeciesReference), or those \SBase elements with \Math children (so, \Constraint, \Delay, \EventAssignment, \FunctionDefinition, \InitialAssignment, \KineticLaw, \Priority, \Rule, and \Trigger).  This is added here to \SBase instead of to each of these the various SBML elements so that other packages inherit the ability to extend their own elements in the same fashion:   the \FluxBound class from the Flux Balance Constraints package has mathematical meaning, for example, and can be given information about the distribution or set of samples from which it was drawn.  Similarly, the \FunctionTerm class from the Qualitative Models package has a \Math child, which could be similarly extended.

A few SBML elements can interact in interesting ways that can confuse the semantics here.  A \Reaction element and its \KineticLaw child, for example, both reference the exact same mathematics, and should therefore have the same \Uncertainty.  Similarly, if an \InitialAssignment assigns to a constant element (\Parameter, \Species, etc.), the uncertainty for both should be the same, or only one should be provided.

Other elements not listed above should probably not be given an \Uncertainty child, as it would normally not make sense to talk about the uncertainty of something that doesn't have a corresponding mathematical meaning.  However, because packages or annotations can theoretically give new meaning (including mathematical meaning) to elements that previously did not have them, this is not a requirement.

It is important to note that the uncertainty described by the \Uncertainty class is defined as being the uncertainty at the moment the element's mathematical meaning is calculated, and does not describe the uncertainty of how that element changes over time.  For a \Species, \Parameter, \Compartment, and \SpeciesReference, this means that it is the uncertainty of their initial values, and does not describe the uncertainty in how those values evolve in time.  The reason for this is that other SBML constructs all describe how (or if) the values change in time, and it is those other constructs that should be used to describe a symbol's time-based uncertainty.  For example, a \Species whose initial value had uncertainty due to instrument precision could have an \Uncertainty child describing this.  A \Species whose value was known to change over time due to unknown processes, but which had a known average and standard deviation could be given an \AssignmentRule that set that \Species amount to the known average, and the \AssignmentRule itself could be given an \Uncertainty child describing the standard deviation of the variability.


\subsection{The \class{Uncertainty} class}
\label{uncertainty-class}

The \Uncertainty class is a container for \AbstractUncertainty (from \uncertml) that descibes the uncertainty in the parent element's mathematical meaning.

The optional \token{id} attribute on the \Uncertainty object class serves to provide a way to identify the uncertainty.  The attribute takes a value of type \primtype{SId}.  Note that the identifier of a the uncertainty carries no mathematical interpretation and cannot be used in mathematical formulas in a model.  \Uncertainty also has an optional \token{name} attribute, of type \primtype{string}.  The \token{name} attribute may be used in the same manner as other \token{name} attributes on \sbmlthreecore objects; please see Section~3.3.2 of the \sbmlthreecore specification for more information.


%It also contains an optional \token{index} attribute:  this is to assist with function definitions that return a vector, and indicate the position in the vector that corresponds to the parent element's value.  For example, a \FunctionDefinition might define an explicit PMF where each realization contains three values, one for each of three different parameters in the model (see example \sec{uncertmath-index-example}).  Each of the three parameters could be given an \Uncertainty child containg \UncertMath referencing that \FunctionDefinition, each with the index of the vector corresponding to its value.  Note that the index is zero-indexed:  the first position in the vector will have an \token{index} of \val{0}, the second of \val{1}, etc.
%
%With the release of the Arrays package, presumably the need for this attribute will go away, but it seems a simple enough addition in the absence of that package, and would provide a tangible benefit to many.  Even after the release of the Arrays package, some software might not support arrays, but will still want to annotate this information:  this attribute will allow them to do so.  See \sec{extended-sbase-examples} for examples of how this might work.


\subsection{The \class{AbstractUncertainty} class}
\label{abstractuncertainty-class}

The \AbstractUncertainty class is defined as describing the uncertainty in its parent element's mathematics.  It is the abstract base class, in \uncertml, for that language's \token{Distribution}, \token{Sample}, and \token{Statistics} elements.  This means it has the capability to provide anything from a full distribution to a \token{Sample} element to summary statistics such as \token{StandardDeviation} and \token{Mean}.

For convenience, \uncertml provides a \token{StatisticsCollection}, which may be used to collect multiple other elements.  For clarity, if multiple \uncertml elements are present in a \token{StatisticsCollection}, they should not conflict with each other.

The namespace for IDs used in the \uncertml is the SId namespace of elements with mathematical meaning in the model, including from other packages.  If that other package is not understood by an interpreter, the \AbstractUncertainty element may be ignored.  If an interpreter does not understand an ID and cannot tell whether that ID came from a not-understood package, it may issue a warning.

Note that the described uncertainty for elements that change in value over time apply only to the element's uncertainty at a snapshot in time, and not the uncertainty in how it changes in time.  For typical simulations, this means the element's initial condition.  Note too that the description of the uncertainty of a \Species should describe the uncertainty of its \token{amount}, not the uncertainty of its \token{concentration}.  The 'primary' mathematical meaning of a \Species in SBML is always the amount; the concentration may be used, but is considered to be derived.


\subsection{Examples using extended \SBase}
\label{extended-sbase-examples}

Several examples are given to illustrate the use of the \Uncertainty class:


\subsubsection{Basic \AbstractUncertainty example}

In this examples, a species is given an \Uncertainty child to describe its standard deviation:

\begin{example}
...
      <species id="S1" compartment="C" initialAmount="3.22" hasOnlySubstanceUnits="true"
               boundaryCondition="false" constant="false">
        <distrib:uncertainty>
          <UncertML xmlns="http://www.uncertml.org/3.0">
            <StandardDeviation definition="http://www.uncertml.org/distributions">
              <value>
                <prVal>0.3</prVal>
              </value>
            </StandardDeviation>
          </UncertML>
        </distrib:uncertainty>
      </species>
...
\end{example}

Here, the species with an initial amount of 3.22 is described as having a standard deviation of 0.3, a value that might be written as \val{3.22 $\pm$ 0.3}.  This is probably the simplest way to use the package to introduce facts about the uncertainty of the measurements of the values present in the model.

It is also possible to include additional information about the species, should more be known:

\begin{example}
...
      <species id="S1" compartment="C" initialAmount="3.22" hasOnlySubstanceUnits="true"
               boundaryCondition="false" constant="false">
        <distrib:uncertainty>
          <UncertML xmlns="http://www.uncertml.org/3.0">
            <StatisticsCollection definition="http://www.uncertml.org/statistics">
              <Mean definition="http://www.uncertml.org/statistics">
                <value>
                  <rVal>3.2</rVal>
                </value>
              </Mean>
              <StandardDeviation definition="http://www.uncertml.org/statistics">
                <value>
                  <prVal>0.3</prVal>
                </value>
              </StandardDeviation>
              <Variance definition="http://www.uncertml.org/statistics">
                <value>
                  <prVal>0.09</prVal>
                </value>
              </Variance>
            </StatisticsCollection>
          </UncertML>
        </distrib:uncertainty>
      </species>
...
\end{example}

In this example, the initial amount of 3.22 is noted as having a mean of 3.2, a standard deviation of 0.3, and a variance of 0.09.  Since there is more than one piece of information included, a \token{StatisticsCollection} \uncertml element is used.  Note that the standard deviation can be calculated from the variance (or visa versa), but the modeler has chosen to include both here for convenience.  Note too that this use of the \Uncertainty element does not imply that the species amount comes from a normal distribution with a mean of 3.2 and standard deviation of 0.3, but rather that the species amount comes from an unknown distribution with those qualities.  If it is known that the value was drawn from a particular distribution, that distribution should be used, rather than the \token{Mean} and \token{StandardDeviation} statistical values.

Note also that 3.22 (the \token{initialAmount}) is different from 3.2 (the \token{Mean}):  evidently, this model was constructed as a realization of the underlying uncertainty, instead of trying to capture the single most likely model of the underlying process.


\subsubsection{Defining a Random Variable}

In addition to describing the uncertainty about an experimental
observation one can also use this mechanism to describe a parameter as
a random variable. In the example below the parameter, $Z$, is defined
as following a normal distribution, with a given mean and variance. No
value is given for the parameter so it is then up the modeller to
decide how to use this random variable. For example they may choose to
simulate the model in which case they may provide values for $mu\_Z$
and $var\_Z$ and then sample a random value from the
simulation. Alternatively they may choose to carry out a parameter
estimation and use experimental observations to estimate $mu\_Z$ and
$var\_Z$.

\begin{example}
    <listOfParameters>
      <parameter id="mu_Z" value="10" constant="true"/>
      <parameter id="var_Z" value="0.1" constant="true"/>
      <parameter id="Z" constant="true">
        <distrib:uncertainty>
          <UncertML xmlns="http://www.uncertml.org/3.0">
            <NormalDistribution definition="http://www.uncertml.org/distributions">
              <mean>
                <var varId="mu_Z"/>
              </mean>
              <variance>
                <var varId="var_Z"/>
              </variance>
            </NormalDistribution>
          </UncertML>
        </distrib:uncertainty>
      </parameter>
    </listOfParameters>
\end{example}

{\color{red}%
  Stuart:\controversial This example illustrates the kind of use cases
  I was describing at HARMONY. I hope this isn't controversial, but I
  want us to be clear that we all happy using this interpretation of
  uncertainty too. If so then it makes sense to have an example. }


%\subsubsection{An \AbstractUncertainty example with a PMF}
%\label{uncertmath-index-example}

%In the following example, a species and two parameters are all given \AbstractUncertainty children indicating that they were sampled from a PMF.  Because all three have the same \token{names} IDs, the assumption can be made that the three values are correlated, and are drawn from \val{patient1}, \val{patient2}, or \val{patient3}.

%\begin{example}
%...
%  <listOfSpecies>
%    <species id="S1" compartment="C" boundaryCondition="false" initialAmount="2.24"
%             hasOnlySubstanceUnits="true" constant="false">
%      <distrib:uncertainty>
%         <CategoricalDistribution xmlns="http://uncertml.org/3.0">
%           <values>1.01 2.24 1.72</values>
%           <names>patient1 patient2 patient3</names>
%           <probabilities> 0.5 0.25 0.25 </probabilities>
%         </CategoricalDistribution>
%      </distrib:uncertainty>
%    </species>
%  </listOfSpecies>
%  <listOfParameters>
%    <parameter id="y" constant="true" value="45.9">
%      <distrib:uncertainty>
%        <CategoricalDistribution xmlns="http://uncertml.org/3.0">
%          <values>42.3 45.9 50.0</values>
%          <names>patient1 patient2 patient3</names>
%          <probabilities> 0.5 0.25 0.25 </probabilities>
%        </CategoricalDistribution>
%      </distrib:uncertainty>
%    </parameter>
%    <parameter id="z" constant="true" value="0.002">
%      <distrib:uncertainty>
%        <CategoricalDistribution xmlns="http://uncertml.org/3.0">
%          <values>0.004 0.002 0.0033</values>
%          <names>patient1 patient2 patient3</names>
%          <probabilities> 0.5 0.25 0.25 </probabilities>
%        </CategoricalDistribution>
%      </distrib:uncertainty>
%    </parameter>
%  </listOfParameters>
%...
%\end{example}

%In this example, the chosen value for all three elements is the second position in the \token{values} vector, corresponding to \val{patient2}, and presumably comes from the same external data source.

%In a version of this model that used Arrays package constructs, a single vector could be created with all three values in it, and given an \AbstractUncertainty child with \token{value} elements that contained vectors instead of single values.

%{\color{red} Stuart:\controversial I'm not at all sure about this. I
%  don't think that's a correct use of realisations to interpret that the same ID implies correlation. However, I have asked Dan
%  for clarification on the use of realisations so we can get the
%  definitive view on this. Aside from the above example it may be good
%  to start off with a simple example. How about this?  
%  We could even be clever and add a third gender!}

%{\color{red} Lucian:\controversial Updated based on Dan's reply to my emails!  I've tweaked your example below to match:}

%{\color{red}
%\begin{example}
%...
%  <parameter id="gender">
%    <distrib:uncertainty>
%      <CategoricalDistribution xmlns="http://uncertml.org/3.0">
%        <!-- M=0 F=1 -->
%        <values>0 1</values>
%        <probabilities>0.6 0.4</probabilities>
%      </CategoricalDistribution>
%    </distrib:uncertainty>
%  </parameter>
%...
%\end{example}
%}


\section{Interaction with other packages}

\begin{blockChanged}
\subsection{Custom annotations for function definitions}
Before this package was available, a collection of SBML simulator authors developed an \emph{ad-hoc} convention for exchanging annotated \FunctionDefinition objects that represented draws from distributions.  This convention is described by Frank T. Bergmann at \url{https://docs.google.com/file/d/0B_wMqVOQLkZ3TVZHblNNRWgzNTg/}, and represents a basic starting point for any modeler interested in exchanging SBML models containing draws from distributions.

When implementing \distrib support, it would be possible to include 'backwards' support for this annotation convention by annotating any \FunctionDefinition using UncertML to also include these annotations, where appropriate.

The following table is taken from the above document by Frank Bergmann, and can be used to annotate \FunctionDefinition elements that have been extended by \distrib to perform the same functions, providing the arguments are presented in the same order.  The suggested fallback function returns the mean of the distribution.

\begin{longtabu} to \linewidth {
    X[2,c]
    X[3,c]
    X[12,c]
    X[4,l]}
\textbf{Id} & \textbf{Name} & \textbf{URL} & \textbf{Fallback} \\ \midrule
uniform & Uniform distribution & \footnotesize{\url{http://en.wikipedia.org/wiki/Uniform_distribution_(continuous)}} &\small{$lambda(a,b,\frac{a+b}{2})$}
\\ \midrule
normal & Normal distribution & \footnotesize{\url{http://en.wikipedia.org/wiki/Normal_distribution}} &\small{$lambda(m,s,m)$}
\\ \midrule
exponential & Exponential distribution & \footnotesize{\url{http://en.wikipedia.org/wiki/Exponential_distribution}} &\small{$lambda(l,\frac{1}{l})$}
\\ \midrule
gamma & Gamma distribution & \footnotesize{\url{http://en.wikipedia.org/wiki/Gamma_distribution}} &\small{$lambda(a,b,a*b)$}
\\ \midrule
poisson & Poisson distribution & \footnotesize{\url{http://en.wikipedia.org/wiki/Poisson_distribution}} &\small{$lambda(mu,mu)$}
\\ \midrule
lognormal & Lognormal distribution & \footnotesize{\url{http://en.wikipedia.org/wiki/Log-normal_distribution}} &\small{$lambda(z,s,e^{z+\frac{s^2}{2}})$}
\\ \midrule
chisq & Chi-squared distribution & \footnotesize{\url{http://en.wikipedia.org/wiki/Chi-squared_distribution}} &\small{$lambda(nu,nu)$}
\\ \midrule
laplace & Laplace distribution & \footnotesize{\url{http://en.wikipedia.org/wiki/Laplace_distribution}} &\small{$lambda(a,0)$}
\\ \midrule
cauchy & Cauchy distribution & \footnotesize{\url{http://en.wikipedia.org/wiki/Cauchy_distribution}} &\small{$lambda(a,a)$}
\\ \midrule
rayleigh & Rayleigh distribution & \footnotesize{\url{http://en.wikipedia.org/wiki/Rayleigh_distribution}} &\small{$lambda(s,s*\sqrt{\pi/2})$}
\\ \midrule
binomial & Binomial distribution & \footnotesize{\url{http://en.wikipedia.org/wiki/Binomial_distribution}} &\small{$lambda(p,n,p*n)$}
\\ \midrule
bernoulli & Bernoulli distribution & \footnotesize{\url{http://en.wikipedia.org/wiki/Bernoulli_distribution}} &\small{$lambda(p,p)$}
\\
\bottomrule
\end{longtabu}

\clearpage
As an example, here is a complete (if small) model that uses both the above 'custom annotation' scheme and the \distrib extensions of a \FunctionDefinition:

\begin{example}
<?xml version="1.0" encoding="UTF-8"?>
<sbml xmlns="http://www.sbml.org/sbml/level3/version1/core"
      xmlns:distrib="http://www.sbml.org/sbml/level3/version1/distrib/version1"
      level="3" version="1" distrib:required="true">
  <model id="__main" name="__main">
    <listOfFunctionDefinitions>
      <functionDefinition id="normal">
        <annotation>
          <distribution xmlns="http://sbml.org/annotations/distribution"
                   definition="http://en.wikipedia.org/wiki/Normal_distribution"/>
        </annotation>
        <math xmlns="http://www.w3.org/1998/Math/MathML">
          <lambda>
            <bvar>
              <ci> mean </ci>
            </bvar>
            <bvar>
              <ci> stddev </ci>
            </bvar>
            <ci> mean </ci>
          </lambda>
        </math>
        <distrib:drawFromDistribution>
          <distrib:listOfDistribInputs>
            <distrib:distribInput distrib:id="mean" distrib:index="0"/>
            <distrib:distribInput distrib:id="stddev" distrib:index="1"/>
          </distrib:listOfDistribInputs>
          <UncertML xmlns="http://www.uncertml.org/3.0">
            <NormalDistribution definition="http://www.uncertml.org/distributions">
              <mean>
                <var varId="mean"/>
              </mean>
              <stddev>
                <var varId="stddev"/>
              </stddev>
            </NormalDistribution>
          </UncertML>
        </distrib:drawFromDistribution>
      </functionDefinition>
    </listOfFunctionDefinitions>
    <listOfParameters>
      <parameter id="x" constant="true"/>
    </listOfParameters>
    <listOfInitialAssignments>
      <initialAssignment symbol="x">
        <math xmlns="http://www.w3.org/1998/Math/MathML">
          <apply>
            <ci> normal </ci>
            <cn> 3 </cn>
            <cn> 0.2 </cn>
          </apply>
        </math>
      </initialAssignment>
    </listOfInitialAssignments>
  </model>
</sbml>

\end{example}


\end{blockChanged}

\subsection{The \arrays package}
This package is dependent on no other package, but relies on the \arrays package
to provide vector and matrix structures if those are desired/used.

\subsection{The \req package}
If the \req package is used, any \FunctionDefinition that has been given an \DrawFromDistribution child must be given a \ChangedMath child referencing this package's namespace.  If the fallback function provides a complete \LambdaClass function, its \token{viableWithoutChange} atribute \emph{may} be set \val{true} if the modeler considers that function an acceptable alternative to the draw from the distribution, otherwise it must be set \val{false}.  The \Uncertainty class does not affect the mathematics of any element, so no other element may be given a \ChangedMath child referencing this package's namespace.


\section{Use-cases and examples}

The following examples are more fleshed out than the ones in the main text, and/or illustrate features of this package that were not previously illustrated.

\subsection{Sampling from a distribution: PK/PD Model}

This is a very straightforward use of an \uncertml-defined
distribution. The key point to note is that a value is sampled from
the distribution and assigned to a variable when it is invoked in the
initialAssignments element in this example. Later use of the variable
does not result in re-sampling from the distribution. This is
consistent with current SBML semantics.

{\color{red}Stuart:\controversial I'd like to add another example here
  that defines the model without sampling. We can have 2 versions of
  the same model. I'll come back to it...}

\exampleFile{examples/pkpd.xml}


\subsection{Truncated distribution}
\label{sec: truncated-eg}

To encode a truncated distribution we may use the optional \token{truncationLowerInclusiveBound} as well as the \token{truncationUpperInclusiveBound} child elements of the \token{NormalDistribution} element.  Many other \uncertml distributions have these optional child elements as well.  Note that the values for truncation are here provided as arguments to the function definition, but could instead be hard-coded.

\exampleFile{examples/truncated_distn.xml}


\subsection{Multivariate distribution}

In this example two correlated parameters are sampled from a
multivariate distribution. The correlation is defined using a
covariance matrix and the sampled values are returned as a vector of 2
values, and assigned to the variable \val{correlated\_params}.  This vector is then used to assign values to \val{V} and \val{C1}, thereby associating those two values with the same draw from the multivariate normal distribution.  The use of various array and matrix \mathml here is speculative:  in the absence of a finalized Arrays package, it is impossible to tell exactly what form that will take.  However, all of the functionality expressed here will need to be incorported in some form into the Arrays package, and much if not all of it may take the form illustrated here.

\exampleFile{examples/mutivariate_example.xml}


%\subsection{Multiple uses of distributions }

%In this example, an \uncertml Normal distribution is used in three places:  to denote the uncertainty in the parameter \val{V}, the uncertainty in the initial assignment to \val{V}, and to construct the initial assignment itself through the use of an extended function definition.  Note that strictly speaking, since \val{V} is constant, one could assume that the uncertainty in the parameter itself was identical to the uncertainty in its initial assigment; both are given here by way of illustration.

%\exampleFile{examples/user-defined.xml}


%\subsection{User-defined discrete distribution}
%\label{sec:userDefinedDiscrete}

%In this example, a \Distribution is used where the weights themselves are the arguments to the function instead of the values.

%\exampleFile{examples/user-defined-pmf.xml}


\section{Prototype implementations}

As of this writing (March 2015), libsbml has full support for elements defined by \distrib, as well as limited support for \uncertml.  Antimony (\url{http://antimony.sf.net/}) has support for a limited number of \uncertml function (normal, uniform, exponential, gamma, poisson, and truncated versions of these) for model creation only (no simulation).  LibRoadRunner (\url{http://libroadrunner.org}) also supports the normal and uniform functions (though not their truncated forms), and is a full simulator.  Neither Antimony nor LibRoadRunner support the \token{uncertainty} child of \SBase, and support the extended \FunctionDefinition only.

%\section{Unresolved issues}
%\label{sec:unresolved}


\section{Acknowledgements}
\label{sec:acknowledgements}

Much of the initial concrete work leading to this proposal document
was carried out at the Statistical Models Workshop in Hinxton in 2011,
which was organised by Nicolas le Nov\`{e}re. A list of participants
and recordings of the discussion is available from
\url{http://sbml.org/Events/Other_Events/statistical_models_workshop_2011}.
Before that a lot of the ground work was carried out by Darren
Wilkinson who led the discussion on \distribshort at the Seattle SBML
Hackathon and before that Colin Gillespie who wrote an initial
proposal back in 2005. The author would also like to thank the
participants of the \distribshort sessions during HARMONY 2012 and
COMBINE 2012 for their excellent contributions in helping revising
this proposal; Sarah Keating, Maciej Swat and Nicolas le Nov\`{e}re
for useful discussions, corrections and review comments; and Mike
Hucka for \LaTeX{} advice and the beautiful template upon which this
document is based.

\appendix
\section{Changes anticipated in UncertML 3.0}

\changed{Now that the UncertML 3.0 xsd is out, there are probably not going to be many more changes, if any.}

\section{UncertML Distributions}
\label{sec:uncertml-distributions}

\begin{blockChanged}
In this table, all UncertML 3.0 distributions are listed, along with their types (Continuous, Categorical, or Discrete), whether they're univariate or multivariate, and a brief description.  The UncertML element name is the name of the distribution with spaces removed and \val{Distribution} appended, so, for example, the \val{Exponential} distribution becomes \val{<ExponentialDistribution>}, and the \val{Student T} distribution becomes \val{<StudentTDistribution>}.  The exception is that all of the mixture models (ones that end with \val{Mixture Model}), only have spaces removed but nothing appended:  the Continuous Univariate Mixture Model becomes \val{<ContinuousUnivariateMixtureModel>}, etc.

All of these distributions inherit from the abstract UncertML \emph{AbstractDistribution} class.  Additionally, the appropriate distributions inherit from the \emph{AbstractContinuousDistribution}, \emph{AbstractCategoricalDistribution}, and \emph{AbstractDiscreteDistribution} abstract classes, and further from the \emph{AbstractContinuousUnivariateDistribution}, \emph{AbstractContinuousMultivariateDistribution}, etc. classes, which are related to one another as one would expect.

All descriptions are copied from \url{http://www.uncertml.org/}, where more information can be found.  Do note that any XML content there based on UncertML 2.0 will not be applicable to the \distrib package, which uses UncertML 3.0 instead.  The arguments and definitions will, however, still be relevant.

Distributions are listed grouped by category (type and univarite/multivariate), and alphabetical within those categories.

\begin{longtabu} to \linewidth {
    X[2,c]
    X[2,c]
    X[2,c]
    X[10,l]}
\textbf{Distribution} & \textbf{Type} & \textbf{Variables} & \textbf{Description} \\ \midrule
Beta & Continuous & Univariate 
  & A random variable x is Beta distributed if the probability density function (pdf) is of the form $\frac{1}{B\left(\alpha,\beta\right)}x^{\alpha-1}\left(1-x\right)^{\beta-1}$, where $B\left(\alpha,\beta\right)$ = $\frac{\Gamma\left(\alpha\right)\Gamma\left(\beta\right)}{\Gamma\left(\alpha+\beta\right)}$. The distribution is usually denoted as $x\sim Be\left(\alpha,\beta\right)$ with parameters $\alpha$ and $\beta$, both positive real values. As the domain of the random variable is defined to be $[0,1]$ the Beta distribution is normally used to describe the distribution of a probability value. \\ \midrule
Cauchy & Continuous & Univariate 
  & A random variable x follows a Cauchy distribution if the probability density function (pdf) is of the form $\frac{1}{\pi\gamma}\left[1+\left(\frac{x-\theta}{\gamma}\right)^2\right]^{-1}$. The Cauchy distribution is equivalent to a Student-T distribution with 1 degree of freedom. It is widely used in physics, optics and astronomy. It is also known as the Lorenz or the Breit-Wigner distribution.\\ \midrule
Chi Square & Continuous & Univariate 
  & A random variable x is Chi-square distributed if the probability density function (pdf) is of the form $\frac{1}{\Gamma(\nu/2)2^{\nu/2}}x^{\nu/2-1}exp(-x/2)$. The distribution is usually denoted as $x\sim\chi_\nu$ where $\nu$ is known as the degrees of freedom parameter. $\nu$ has to be positive and $x$ has to be non-negative for the density to be defined. The Chi-square distribution is a special case of the Gamma distribution where $\chi \sim\Gamma(k=\nu/2,\theta=2)$. \\ \midrule
Continuous Univariate Mixture Model & Continuous & Univariate 
  & A mixture model is a linear combination of base distributions. A widely used case is where the base distributions are Gaussian in which case the model is known as the Gaussian Mixture Model. \\ \midrule
Dirac Delta & Continuous & Univariate 
  & [New to UncertML 3.0--no definition yet.] \\ \midrule
Exponential & Continuous & Univariate
  & A random variable x follows an exponential distribution if the probability density function (pdf) is of the form $\lambda e^{-\lambda x}$. It is often represented as $x \sim$ Exp$(\lambda)$. It is used to model the time between events for a Poisson process and is used in simulation of stochastic systems. \\ \midrule
F & Continuous & Univariate 
  & A random variable x follows an F distribution if the probability density function (pdf) is of the form $\frac{ 1 } {B(\nu_1/2, \nu_2/2)} \left( \frac{\nu_1}{\nu_2}\right)^{\nu_1/2} x^{\nu_1/2 - 1} \left(1 + \frac{\nu_1}{\nu_2}x \right)^{-\frac{\nu_1+\nu_2}{2} }$ where $B(.)$ is the Beta function. It often arises as the ratio of two random variables that are identically Chi-Square distributed. \\ \midrule
Gamma & Continuous & Univariate 
  & A random variable x is Gamma distributed if the probability density function (pdf) is of the form $\frac{1}{\gamma(k)\theta^{k}}x^{k-1}$. The distribution is usually denoted as $x \sim \mathcal{N}(\mu, \sigma^2)$ where $k$ is known as the shape parameter and $\theta$ the scale parameter. Both parameters have be positive and x has to be non-negative for the density to be defined. In practice the Gamma distribution is often use to model the distribution of non-negative quantities such as variances. \\ \midrule
Inverse Gamma & Continuous & Univariate 
  & A random variable x is Inverse Gamma distributed if the probability density function (pdf) is of the form $\frac{\beta^\alpha}{\gamma(\alpha)}x^{-\alpha-1}exp(-\beta/x)$. If variable $x$ is Inverse Gamma distributed, 1/$x$ is gamma distributed. The Inverse Gamma distribution function can be obtained from the Gamma distribution by a transformation of variables. \\ \midrule
Laplace & Continuous & Univariate 
  & A random variable x is Laplace distributed if the probability density function (pdf) is of the form $\frac{1}{2b}exp(-\frac{abs(x-\mu)}{b})$ where $abs$ denotes the absolute value. It can be thought of as a combination of two exponential distributions. \\ \midrule
Log Normal & Continuous & Univariate 
  & A random variable x is Log Normal distributed if the probability density function (pdf) is of the form $\frac{1}{x\sqrt{2\pi\sigma^2}}exp(-\frac{(ln(x)-\mu)^2}{2\mu^2})$. If variable x is normally distributed, exp(x) is Log Normal distributed. The Log Normal distribution function can be obtained from the normal distribution by a transformation of variables. It is often used for variables that must be positive. \\ \midrule
Logistic & Continuous & Univariate 
  &  A random variable x is Logistic distributed if the probability density function (pdf) is of the form $\frac{exp(-(x-\mu)/s)}{s(1+exp(-(x-\mu)/s))^2}$.\\ \midrule
Normal & Continuous & Univariate
  & A random variable x is normally distributed if the probability density function (pdf) is of the form $\frac{1}{\sqrt{2\pi\sigma^2}}exp(-\frac{(x-\mu)^2}{2\sigma^2})$. The distribution is usually denoted as $x\sim \mathcal{N}(\mu,\sigma^2)$ where $\mu$ is known as the mean parameter and $\sigma^2$ the variance parameter. If the random variable x is a vector of length greater than one, the normal distribution can be generalised to the Multivariate normal.  A reason for the widespread usage of the normal distribution is the Central limit theorem which states that the distribution of the mean of a large number of independent identically distributed random variables tends to a normal distributions as the number of random variables increases.\\ \midrule
Pareto & Continuous & Univariate 
  & A random variable x follows a Pareto distribution if the probability density function is of the form $\frac{\alpha x_m^\alpha}{x^{\alpha+1}}$. The distribution allows for the specification of a minimum value below which the density is 0. It is a skewed heavy-tailed distribution. \\ \midrule
Student T & Continuous & Univariate 
  & A random variable x follows a Student-t distribution if the probability density function (pdf) is of the form $\frac{\Gamma(\nu/2+1/2)}{\Gamma(\nu/2)(\pi\nu\sigma^2)^{1/2}}[1+\frac{(x-\mu)^2}{\nu\sigma^2}]^{-\nu/2-1/2}$. The distribution is usually denoted as $x\sim St(\mu,\lambda,\nu)$. This distribution corresponds to integrating out the variance of a normal distribution using a inverse Gamma prior. It can therefore be interpreted as an infinite mixture of normal distributions having the same mean but different variances. The three parameters are the mean ($\mu$), degrees of freedom ($\nu$) and variance ($\sigma^2$). Setting the variance to 1 and the mean to 0 we obtain the Student-t form found in standard statistics references such as Wikipedia. Setting the d.f. to 1 the Cauchy distribution is obtained. Setting the d.f. to infinity the normal distribution is obtained. The student-t distribution is commonly used in likelihood inference as the maximum likelihood parameter estimates are more robust to outlier observations compared to the normal distribution. \\ \midrule
Uniform & Continuous & Univariate 
  & A random variable x follows a uniform distribution if the probability density function (pdf) is of the form $\frac{1}{b-a}$. The distribution assigns equal probability to all events within the chosen domain between the minimum ($a$) and the maximum ($b$). \\ \midrule
Weibull & Continuous & Univariate 
  & A random variable x follows an Weibull distribution if the probability density function (pdf) is of the form $\frac{k}{\lambda}\left(\frac{x}{\lambda}\right)^{k-1}exp(-x/\lambda)^k$. It includes the exponential distribution as a special case. It is often used in engineering and finance. \\ \midrule
\midrule
Continuous Multivariate Mixture Model & Continuous & Multivariate 
  & A mixture model is a linear combination of base distributions. A widely used case is where the base distributions are Gaussian in which case the model is known as the Gaussian Mixture Model. \\ \midrule
Dirichlet & Continuous & Multivariate 
  & A $K$ dimensional random variable x follows a Dirichlet distribution if the probability density function (pdf) is of the form $\frac{1}{B(\mathbf{a})} \prod_{i=1}^K x_i^{\alpha_i - 1}$ where $B(\mathbf{a}) = \frac{ \prod^k_{i=1} \Gamma(\alpha_i) } { \Gamma( \sum_{i=1}^K \alpha_i ) }$ and $\Gamma(.)$ is the Gamma function. It is the multivariate extension of the beta distribution to higher dimensions with K a positive integer greater than or equal to 2. \\ \midrule
Multivariate Normal & Continuous & Multivariate 
  & The Multivariate Normal is an extension of the univariate normal distribution to higher dimensional vector spaces. A random vector variable of dimension $ k $  denoted $ \mathbf{x} $ is normally distributed if the probability density function (pdf) is of the form $(2 \pi)^{-k/2} \mathrm{det}(\Sigma)^{-1/2} exp (-\frac{1}{2} (\mathbf{x} - \mathbf{\mu})^T \Sigma^{-1} (\mathbf{x} - \mathbf{\mu}) ) $ where $ \mathrm{det}(.) $ denotes the determinant and $ (.)^T $ the matrix transpose. The distribution is usually denoted as $ \mathbf{x} \sim \mathcal{N}(\mathbf{\mu}, \Sigma) $ where $ \mathbf{\mu} $ is known as the mean vector parameter and $ \Sigma $ the covariance matrix parameter. \\ \midrule
Multivariate Student T & Continuous & Multivariate 
  & A random variable $ \mathbf{x} $ follows a multivariate Student-t distribution if the probability density function (pdf) is of the form $\frac{ \Gamma(\nu/2 + k/2)} {\Gamma(\nu/2) (\pi \nu)^{k/2} \mathrm{det} (\Sigma)^{1/2} } \left[1 + \frac{ \Delta^2 } { \nu } \right]^{-\nu/2 - k/2}  $ where $ \Delta^2 = (\mathbf{x} - \mathbf{\mu})^T \Sigma^{-1}  (\mathbf{x} - \mathbf{\mu}) $ is the squared Mahalanobis distance. The distribution is usually denoted as $ x \sim St(\mathbf{\mu},\Sigma,\nu) $. It is the extension of the univariate student-t distribution to higher dimensions. Student-t distributions are often used when tails are expected to be heavier than Gaussian or Normal, and can result from applying Bayesian inference. \\ \midrule
Normal Inverse Gamma & Continuous & Multivariate 
  & A Normal Inverse Gamma distribution is the conjugate prior of a normal distribution with unknown mean and variance. It is the coupled product of an Inverse Gamma distribution and a normal distribution. In particular if $p(\mathbf{X} ; \mu, \sigma^2) $ is the likelihood function of a Normally distributed set of random variables with mean $ \mu $ and variance $ \sigma^2 $ and if both the mean and variance are considered unknown, the conjugate prior is $  p(\mu,\sigma^2) = p(\mu ; \sigma^2) p(\sigma^2) $ where $  p(\mu ; \sigma^2) = \mathcal{N}(\mu ; \mu_0, \sigma^2/\nu) $ a Normal prior on the mean and $ p(\sigma^2) = \mathrm{IG}(\sigma^2 ; \alpha, \beta) $ an Inverse Gamma prior on the variance. Note that the priors are not independent as the prior variance of the mean is a linear function of of the variance $ \sigma^2 $. It is also common to use a Normal-Gamma distribution where a conjugate prior is placed on the unknown mean and precision (i.e. inverse variance) of the Normal likelihood in which case the prior is a product of a Normal and Gamma distributions. In the case of a Multivariate normal likelihood, the corresponding conjugate prior is a Normal-Wishart distribution. \\ \midrule
\midrule
Bernoulli & Categorical & Univariate 
  & A random variable $ x $ follows a Bernoulli distribution if the probability mass function (pmf) is of the form $\mu^x (1-\mu)^{1-x} $. It describes the distribution of a single binary variable $ x $. \\ \midrule
Categorial Univariate Mixture Model & Categorical & Univariate 
  & A mixture model is a linear combination of base distributions. A widely used case is where the base distributions are Gaussian in which case the model is known as the Gaussian Mixture Model. \\ \midrule
\midrule
Categorical & Categorical & Multivariate 
  & A Categorical distribution is a generalisation of the Bernoulli distribution to $ K $ discrete outcomes, giving the $ K $ probabilities $ p_i $, $ i=1,..,K $ for each outcome. There is no ordering in the $ K $ outcomes. \\ \midrule
Categorial Multivariate Mixture Model & Categorical & Multivariate 
  & A mixture model is a linear combination of base distributions. A widely used case is where the base distributions are Gaussian in which case the model is known as the Gaussian Mixture Model. \\ \midrule
\midrule
Binomial & Discrete & Univariate 
  & A random variable $ x $ follows a Binomial distribution if the probability mass function (pmf) is of the form ${n \choose x} \theta^x (1-\theta)^{n-x} $, where $ {n \choose x} $ denotes $ n $ choose $ x $. The distribution is usually denoted as $ x \sim b(n,\theta) $. The distribution describes the probability of getting $ x $ successes in $n$ trials of independent experiments that have the same probability of success. \\ \midrule
Discrete Univariate Mixture Model & Discrete & Univariate 
  & A mixture model is a linear combination of base distributions. A widely used case is where the base distributions are Gaussian in which case the model is known as the Gaussian Mixture Model. \\ \midrule
Geometric & Discrete & Univariate 
  & A random variable $ x $ follows a geometric distribution if the probability mass function (pmf) is of the form $(1-p)^{x-1} p$. It is often represented as $x \sim \mathrm{Geom}(p)$. It is the discrete analogue of the exponential distribution. It is used to model distribution of the number of binary (Bernoulli) trials needed to get one success, with parameter, probability $p$. \\ \midrule
Hypergeometric & Discrete & Univariate 
  & A random variable $ x $ follows a hypergeometric distribution if the probability mass function (pmf) is of the form $\frac{ {m \choose k} { N-m \choose n-k } } { {N \choose n}}$, probability of getting $x$ successes.  It describes the number of successes in a sequence of draws without replacement. \\ \midrule
Negative Binomial & Discrete & Univariate 
  & A random variable $ x $ follows a Negative Binomial distribution if the probability mass function (pmf) is of the form ${x + r - 1 \choose x} p^x (1-p)^r $. The distribution describes the probability of getting $ x $ successes in trials of independent experiments that have the same probability of success, and are run until we observe $ r $ failures. \\ \midrule
Poisson & Discrete & Univariate 
  & A random variable $ x $ follows a Poisson distribution if the probability mass function (pmf) is of the form $\frac{\lambda^x}{x!} \mathrm{exp}(-\lambda)$. The Poisson distribution can be used to model the number of events occurring within fixed time period of time. \\ \midrule
\midrule
Discrete Multivariate Mixture Model & Discrete & Multivariate 
  & A mixture model is a linear combination of base distributions. A widely used case is where the base distributions are Gaussian in which case the model is known as the Gaussian Mixture Model. \\ \midrule
Multinomial & Discrete & Multivariate 
  & A random variable $ x $ follows a Multinomial distribution if the probability mass function (pmf) is of the form $\frac{N!}{x_1! \dots x_k!} \prod_{i=1}^K p_i^{x_i}$. The Multinomial distribution is a multivariate generalisation of the Binomial distribution for a $ K $ state variable to be in state $ k $ given $N$ observations. This can be confused with the Categorical distribution (added at version 3.0 of UncertML) which can be considered a Multinomial distribution when a 1 of K encoding is used. \\ \midrule
Wishart & Discrete & Multivariate 
  & A random matrix variable $ \mathbf{X} $ of size $ D \times D $ follows a Wishart distribution if the probability density function is of the form $\mathrm{det}(\mathbf{W})^{-\nu/2} \left( 2^{\nu D/2} \pi^{D(D-1)/4} \prod_{i=1}^D \Gamma\left(\frac{\nu+1-i}{2}\right) \right)^{-1}$ $\mathrm{det}(\mathbf{X})^{(\nu-D-1)/2} \exp\left(-\frac{1}{2} \mathrm{Tr} (\mathbf{W}^{-1} \mathbf{X}) \right)$ where $ \mathrm{det} $ denotes the determinant, $ \mathrm{Tr} $ the matrix trace. The Wishart distributon is the conjugate prior for the inverse of a covariance matrix of a Multivariate Normal distribution. It is a generalistion of the gamma distribution to higher dimensions. In one dimesion the Wishart distribution is equivalent to a gamma distribution with parameters $ k=\nu/2 $ and $ \theta = 1/2\mathbf{W} $. \\ \midrule

\bottomrule
\end{longtabu}

\end{blockChanged}

\bibliography{sbml-level-3-distrib-package-proposal}


\end{document}

