% -*- TeX-master: "sbml-level-3-version-1-core"; fill-column: 66 -*-
% $Id$
% $HeadURL$
% ----------------------------------------------------------------

\section{Discussion}
\label{sec:discussion}

The volume of data now emerging from molecular biotechnology leave
little doubt that extensive computer-based modeling, simulation
and analysis will be critical to understanding and interpreting
the data~\citep{abbott:1999,gilman:2000,popel:1998,smaglik:2000}.
This has lead to an explosion in the development of computer tools
by many research groups across the world.  The explosive rate of
progress is exciting, but the rapid growth of the field is
accompanied by problems and pressing needs.

One problem is that simulation models and results often cannot be
directly compared, shared or re-used, because the tools developed
by different groups often are not compatible with each other.  As
the field of systems biology matures, researchers increasingly
need to communicate their results as computational models rather
than box-and-arrow diagrams.  They also need to reuse published
and curated models as library components in order to succeed with
large-scale efforts~\citep[e.g., the Alliance for Cellular
Signaling;][]{gilman:2000,smaglik:2000}.  These needs require that
models implemented in one software package be portable to other
software packages, to maximize public understanding and to allow
building up libraries of curated computational models.

We offer SBML to the systems biology community as a suggested
format for exchanging models between simulation/analysis tools.
SBML is an open model representation language oriented
specifically towards representing systems of biochemical
reactions.

Our vision for SBML is to create an open standard that will enable
different software tools to exchange computational models.  SBML
is not static; we continue to develop and experiment with it, and
we interact with other groups who seek to develop similar markup
languages.  We plan on continuing to evolve SBML with the help of
the systems biology community to make SBML increasingly more
powerful, flexible and useful.


Many people have expressed a desire to see additional capabilities
added to SBML.  Section~\ref{sec:sbml} describes the modular approach 
used in SBML Level~3 to allow optional packages to include additional
features.  Several packages have been proposed and are at various 
stages of the development process.  Up to date information about existing
proposals, together with information on how to submit a proposal can be
found at \url{http://sbml.org/specifications/sbml-level-3/}.
 





