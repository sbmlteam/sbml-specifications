% -*- TeX-master: "main"; fill-column: 72 -*-

\section{Introduction}
\label{intro}

\subsection{Motivation}
Quantitative methods for modelling biological networks require an in-depth knowledge of the biochemical reactions and their stoichiometric and kinetic parameters. In many cases, this knowledge is missing. This has led to the development of several qualitative modelling methods using information such as gene expression data coming from functional genomic experiments. Qualitative models are typically based on the definition of \emph{regulatory} or \emph{influence graph}. The components of these models differ from species and reactions used in current SBML models. For example, qualitative models typically associate discrete levels of activities with entity pools; the processes involving them cannot be described as reactions per se but rather as transitions between states. Boolean networks, logical models and some Petri nets are the most used qualitative formalisms in biology. Despite differences from traditional SBML models, it is desirable to bring these classes of models under a common format scheme. The purpose of this Qualitative Models package for SBML Level 3 is to support qualitative models into SBML.



