% -*- TeX-master: "sbml-level-2-version-3"; fill-column: 66 -*-
% $Id$
% $Source$
% ----------------------------------------------------------------

\section{Overview of SBML}
\label{sec:overview}

The following is an example of a simple network of
biochemical reactions that can be represented in SBML:
\begin{linenomath}
  \begin{equation*}
    \begin{array}{@{}ccl@{}}
      S_1 & \overset{\underrightarrow{k_1 [S_1] / ([S_1] + k_2)}}{} & S_2 \\ \\[-5pt]
      S_2 & \overset{\underrightarrow{\rule{0.26in}{0pt}k_3 [S_2]\rule{0.26in}{0pt}}}{} & S_3 + S_4
    \end{array}
  \end{equation*}
\end{linenomath}
In this particular set of chemical equations above, the symbols in
square brackets (e.g., ``$[S_1]$'') represent concentrations of molecular
species, the arrows represent reactions, and the formulas above
the arrows represent the rates at which the reactions take place.
(And while this example uses concentrations, it could equally have used
other measures such as molecular counts.)
Broken down into its constituents, this model contains a number of
components: reactant species, product species, reactions,
reaction rates, and parameters in the rate expressions.  To analyze or
simulate this network, additional components must be made
explicit, including compartments for the species, and units on the
various quantities.

SBML allows models of arbitrary complexity to be represented.
Each type of component in a model is described using a specific
type of data structure that organizes the relevant information.
The top level of an SBML model definition consists of lists of
these components, with every list being optional:
\begin{center}
  \slshape
  \begin{edtable}{tabular}{c}
    \begin{minipage}{3in}
      \begin{tabbing}
        xxxx\=\kill
        beginning of model definition\\[-1pt]
        \>list of function definitions (optional)\\[-1pt]
        \>list of unit definitions (optional)\\[-1pt]
        \>list of compartment types (optional)\\[-1pt]
        \>list of species types (optional)\\[-1pt]
        \>list of compartments (optional)\\[-1pt]
        \>list of species (optional)\\[-1pt]
        \>list of parameters (optional)\\[-1pt]
        \>list of initial assignments (optional)\\[-1pt]
        \>list of rules (optional)\\[-1pt]
        \>list of constraints (optional)\\[-1pt]
        \>list of reactions (optional)\\[-1pt]
        \>list of events (optional)\\[-1pt]
        end of model definition
      \end{tabbing}
    \end{minipage}
  \end{edtable}
\end{center}

The meaning of each component is as follows:

\begin{description}
  
\item \emph{Function definition}: A named mathematical function
  that may be used throughout the rest of a model.

\item \emph{Unit definition}: A named definition of a new
  unit of \changed{measurement}, or a redefinition of an existing SBML default
  unit.  Named units can be used in the expression of quantities
  in a model.

\item \emph{Compartment Type}: A type of location where
  reacting entities such as chemical substances may be located.

\item \emph{Species type}: A type of entity
  that can participate in reactions.  \changed{Typical} examples of species types
  include ions such as $\text{Ca}^{2+}$, molecules such as
  glucose or ATP, and more.

\item \emph{Compartment}: A well-stirred container of a
  particular type and finite size where species may be located.
  A model may contain multiple compartments of the same
  compartment type.  Every species in a model must be located in
  a compartment.

\item \emph{Species}: A pool of entities of the same
  \emph{species type} located in a specific \emph{compartment}.

\item \emph{Parameter}: A quantity with a symbolic name.
  In SBML, the term \emph{parameter} is used in a generic sense
  to refer to named quantities regardless of whether they are
  constants or variables in a model.  \sbmltwo provides the
  ability to define parameters that are global to a model as
  well as parameters that are local to a single reaction.
  
\item \emph{Initial Assignment}: A mathematical
  expression used to determine the initial conditions of a
  model.  This type of structure can only be used to define how
  the value of a variable can be calculated from other values
  and variables at the start of simulated time.
  
\item \emph{Rule}: A mathematical expression added to the set of
  equations constructed based on the reactions defined in a model.
  Rules can be used to define how a variable's value can
  be calculated from other variables, or used to define the rate
  of change of a variable.  The set of rules in a model can be
  used with the reaction rate equations to determine the
  behavior of the model with respect to time.  The set of rules
  constrains the model for the entire duration of simulated
  time.

\begin{blockChanged}

\item \emph{Constraint}: A means of detecting out-of-bounds
  conditions during a dynamical simulation and optionally issuing
  diagnostic messages.  Constraints are defined by an arbitrary
  mathematical expression computing a true/false value from model
  variables, parameters and constants.  An SBML constraint applies
  at all instants of simulated time; however, the set of
  constraints in model should not be used to \emph{determine} the
  behavior of the model with respect to time.

\end{blockChanged}
  
\item \emph{Reaction}: A statement describing some transformation,
  transport or binding process that can change the amount of one
  or more species.  For example, a reaction may describe how
  certain entities (reactants) are transformed into certain other
  entities (products).  Reactions have associated kinetic rate
  expressions describing how quickly they take place.
  
\item \emph{Event}: A statement describing an instantaneous,
  discontinuous change in a set of variables of any type (species
  quantity, compartment size or parameter value) when a
  triggering condition is satisfied.

\end{description}

Table~\ref{tab:sections} provides a summary of the sections in this
document where each of these components is described in more detail.

\begin{table}[b]
  \small
  \centering
  \begin{tabular}{lllc}
    \toprule
                        &                               &                               & \textbf{New in}\\
    \textbf{Component}  & \textbf{Section}              & \textbf{Starting page}        & \textbf{Version 2?}\\
    \midrule
    Function definition & \ref{sec:functiondefinition}  & \pageref{sec:functiondefinition}\\
    Unit definition     & \ref{sec:unitdefinitions}     & \pageref{sec:unitdefinitions}\\
    Compartment type    & \ref{sec:compartmentType}     & \pageref{sec:compartmentType} & Yes\\
    Species type        & \ref{sec:speciesType}         & \pageref{sec:speciesType}     & Yes\\
    Compartment         & \ref{sec:compartments}        & \pageref{sec:compartments}\\
    Species             & \ref{sec:species}             & \pageref{sec:species}\\
    Parameter           & \ref{sec:parameters}          & \pageref{sec:parameters}\\
    Initial assignment  & \ref{sec:initialAssignment}   & \pageref{sec:initialAssignment} & Yes\\
    Rule                & \ref{sec:rules}               & \pageref{sec:rules}\\
    Constraint          & \ref{sec:constraints}         & \pageref{sec:constraints}     & Yes\\
    Reaction            & \ref{sec:reactions}           & \pageref{sec:reactions}\\
    Event               & \ref{sec:events}              & \pageref{sec:events}\\
    \bottomrule
  \end{tabular}
  \caption{A guide to the sections, and their starting page numbers, where
    each major SBML component is described in this specification document.
    The ``New?'' column indicates whether a given component is new as of
    \sbmltwotwo.}
  \label{tab:sections}
\end{table}

A software package can read an SBML model description and
translate it into its own internal format for model analysis.  For
example, a package might provide the ability to simulate the model
by constructing differential equations representing the network
and then perform numerical time integration on the equations to
explore the model's dynamic behavior.  By supporting SBML as an
input and output format, different software tools can all operate
on an identical external representation of a model, removing
opportunities for errors in translation and assuring a common
starting point for analyses and simulations.
