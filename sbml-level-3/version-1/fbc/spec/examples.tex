% -*- TeX-master: "main"; fill-column: 72 -*-

\section{Illustrative examples of the \FBC syntax}
\label{examples}

This section contains a worked example showing the encoding of a model suitable for Flux Balance Analysis using the \FBCPackage.

\subsection{Example one: the basic \FBC syntax}
\label{examples1}
\subsubsection{Kinetic model description}
\begin{figure}[h]
  \centering
  % Requires \usepackage{graphicx}
  \includegraphics[width=8cm]{examples/spec-example1.pdf}\\
  \caption{\FBC syntax example: a simple four reaction pathway. The
  reactions are \textit{R1}, \textit{R2}, \textit{X1}, \textit{X2} with
  fixed species \textit{IN}, \textit{OUT}, \textit{ATP}, \textit{NADH} and
  variable species \textit{A}, \textit{B}.}
  \label{fig:example1}
\end{figure}

As shown in \ref{fig:example1} this example is a simple four reaction 
pathway that transforms metabolite \textit{IN} to \textit{OUT}. The 
model was created and analyzed using the \textsf{SBW Flux Balance} \FBC 
implementation \cite{sbwfba, sbw}. In \SBML each reaction is represented 
as a chemical process transforming reactants to products, e.g. reaction 
\textit{R1} is encoded in XML as (see also the complete example provided 
at the end of this section): 


%
\exampleFile{examples/spec-example1-reaction.txt}
%
Using the reagent identity and stoichiometry it is possible to compactly 
describe this network in terms of its reaction stoichiometry as shown in 
\ref{tble:ex1nmat} where each reaction is represented as a column. 

\begin{table}[h]
  \centering
    \begin{tabular}{c|cccc}
          & R1 & R2 & X1 & X2 \\ \hline
        A & 1 &  0 & -1 & -1 \\
        B & 0 & -1 &  1 &  1 \\
    \end{tabular}
  \caption{Example one: stoichiometric matrix, \Nmat}
  \label{tble:ex1nmat}
\end{table}
%
While the stoichiometry contains the structural properties of the 
reaction network the full description of a biological model can be 
described as a set of ordinary differential equations (ODE's). Of course 
other formalisms do exist, but here we will concentrate exclusively on 
kinetic models where the change in concentration of each variable 
component in the system ($\frac{ds}{dt}$) is a non-linear function of 
the rates of the reactions which either create or consume it (the 
product of the stoichiometric matrix, \Nmat\ and the vector of reaction 
rates, \vvec). 


%
\begin{equation}\label{eqn:kinmod}
  \frac{ds}{dt} = \textbf{Nv}
\end{equation}
%
The formulation of the kinetic model, as shown in 
Equation~\ref{eqn:kinmod} is typical of the kind that can already be 
described using \sbmlthreecore where the vector \vvec\ would contain 
rate equations as a function of parameters and variable species. In a 
steady-state, constraint based model these rates are considered unknowns 
and the system of equations can be rewritten as a set of linear 
constraints (see Equation~\ref{eqn:kinmodsteady}): 


%
\begin{equation}\label{eqn:kinmodsteady}
  \textbf{NJ} = 0
\end{equation}
%
Note that the rate vector \vvec\ is now represented as the steady-state 
flux vector \Jvec. However, in order to perform a typical steady-state 
analysis such as flux balance analysis (FBA) we need to include more 
information into the model description. \sbmlthreecore does not have an 
unambiguous way of encoding either a capacity constraint or an objective 
target and for this we need to use the additional constructs provided by 
the \FBCPackage. In the following sections the same model data is shown 
encoded as XML and as a Linear Program (LP) in the common format used by 
IBM \textsf{CPLEX}. 


\subsubsection{Capacity constraints}
\label{examples1:fluxbound}
A capacity constraint: in this example the maximum limit (upper bound) 
of the flux through reaction \textit{R1} is set to be one (with an 
arbitrary unit of flux). In LP format this can be written as: 

%
\exampleFile{examples/spec-example1-bnd.lp}
%
the same information encoded as XML:
%
\exampleFile{examples/v2harmony-spec-example1-bnd.txt}

\subsubsection{Objective function}
\label{examples1:objfunc}
This described a target which can be maximized or minimized: in this 
example the flux through reaction \textit{R2} will be 
\textit{maximized}. 

%
\exampleFile{examples/spec-example1-obj.lp}
%
the same information encoded as XML:
%
\exampleFile{examples/spec-example1-obj.txt}

\subsubsection{Complete worked example}
\label{examples1:complete}
To conclude we show how the complete model described in 
\ref{fig:example1} encoded as both an LP and as XML. Formulated as an LP 
the problem can be written as: 

%
\exampleFile{examples/spec-example1.lp}
%
Solving this we find that maximization of flux through \textit{R2}
gives an optimal solution $R2 = 1$, shown in Equation~\ref{egn:ex1sol1}, with one possible solution
for \Jvec.
\begin{equation}\label{egn:ex1sol1}
  \left(
    \begin{array}{cccc}
        1 &  0 & -1 & -1 \\
        0 & -1 &  1 &  1 \\
    \end{array}
  \right)
  \left(
    \begin{array}{c}
        1.0 \\
        \textbf{1.0} \\
        0.0 \\
        1.0 \\
    \end{array}
  \right)
  = \textbf{0}
\end{equation}

Finally we provide the complete model, described above, encoded using the \FBCPackage:
%
\exampleFile{examples/v2harmony-spec-example1.txt}
